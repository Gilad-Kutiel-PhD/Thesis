In this thesis we develop and analyze approximation algorithms, while focusing on two families of problems: network design and submodular optimization.
For the former family we consider the problems of \textsc{Convex Recoloring} and the \textsc{Service Chain Placement in SDNs}.
For the latter family we consider the problems of maximizing a monotone submodular function given a knapsack constraint and the \textsc{Maximum Carpool Matching} problem.

In a typical instance of a \emph{network design} problem we are given a collection of resources and our goal is to construct a desired network that satisfies some requirements while utilizing the minimum possible amount of resources.
One of the classical network design problems (and probably the most well known) is the \textsc{Minimum Spanning Tree} problem where, given an undirected weighted graph, our goal is to find the minimum cost set of edges that induces a connected graph.

Let $\mathcal{C}$ be a set of colors, a \emph{coloring} of an undirected graph, $G(V, E)$, is a function $\chi:V \to \mathcal{C}$.
We say that a coloring is \emph{convex} if the vertices of each color induce a connected graph of $G$.
In the \textsc{Convex Recoloring} problem we are given an undirected graph $G$ and a (non-convex) coloring of this graph and our goal is to find a convex coloring of the graph that recolors the minimum number of vertices.
In this thesis we consider a special case of this problem, namely the \textsc{2-Convex Recoloring} problem, in which the original coloring uses each color in $\mathcal{C}$ to color at most two vertices.
For this special case we develop a greedy $3/2$-approximation algorithm.
We show that if the input graph is a path then this is in fact a $5/4$-approximation algorithm.

In the \textsc{Service Chain Placement in SDNs} problem our goal is to find an optimal resource allocation in a software defined networks that support network function virtualization.
We model the problem and give a fully polynomial time approximation scheme for the special case where the physical network is a directed acyclic graph. 
For general graphs we give a parameterized algorithm. 
This algorithm finds an optimal solution efficiently for cactus network.

Submodularity is a fundamental mathematical notion that captures the concept of economy of scale and is prevalent in many areas of science and technology.
Given a ground set $U$ a set function $f:2^U \to \mathbb{R}_+$ over $U$ is called \emph{submodular} if it has the \emph{diminishing returns} property:
$f(A \cup \{a\}) - f(A) \geq f(B \cup \{a\}) - f(B)$ for every $A \subseteq B \subseteq U$ and $a \in U \setminus B$.
Submodular functions naturally arise in different disciplines such as combinatorics, graph theory, probability, game theory, and economics.

In this thesis we consider the problem of maximizing a monotone, submodular function given a knapsack constraint and develop a fast and simple approximation algorithm for this problem.
We also consider the \textsc{Maximum Carpool Matching} problem, where we seek for an optimal way a group of people should share their ride, and show that this problem can be formalized as a non-monotone, unconstrained, submodular function maximization, thus admits a $1/2$-approximation.