In this section we give a local search $(\half-\eps)$-approximation
algorithm for the \carpool problem.
This algorithm repeatedly checks whether the current carpool matching $M$
can be improved by means of a star centered at a vertex $v$.
The profit from this star is the total weight of the arcs in the star,
and the cost is the total weight of lost arcs
(e.g., arcs from passengers to drivers that became passengers of $v$).
If the profit is larger than the cost, then an improvement step is performed.
The algorithm terminates when such a step is not possible.
We remind the reader that this algorithm will be extended to \gcp in
Section~\ref{sec:group}.

We need a few definitions before presenting our algorithm.
%
Given a directed graphs $G = (V,A)$, define 
$\nin[] \eqdf \set{u : (u,v) \in A}$ 
and 
$$\nout[] \eqdf \set{u : (v,u) \in A}$$
%
Let $M$ be a feasible carpool matching.  The weight $w_M(v)$ of a
vertex $v$ with respect to $M$ is the sum of the weights of the arcs
in $M$ that are incident on $v$, namely
\[
\textstyle
w_M(v)
\eqdf w(M \cap \nin[]) + w(M \cap \nout[])
=     \sum_{(u, v) \in M} w(u, v) + \sum_{(v, u) \in M} w(v, u)
\]
For a subset of vertices $U \subseteq V$ we define
$w_M(U) \eqdf \sum_{v \in U} w_M(v)$.

We now argue that, with respect to any carpool matching $M$, the total
weight of all the vertices is equal to twice the weight of the matching.

\begin{observation}
\label{lm:val-twice}
$w_M(V) = 2 w(M)$.
%\sum_{v \in V} w_M(v) = 2 \sum_{e \in M} w(e)$.
\end{observation}
\begin{proof}
\(
\displaystyle
\sum_{v \in V} w_M(v)
= \sum_{v \in V} \sum_{(u, v) \in M} w(u, v) +
    \sum_{v \in V} \sum_{(v, u) \in M} w(v, u) 
= 2 \sum_{e \in M} w(e)
\).
\end{proof}

Denote by $\delta(u,v)$ the difference between the weight of the arc
and the weight of its source vertex, that is:
\[
\delta_M(u, v) \eqdf w(u, v) - w_M(u)
\]
For a subset $S \subseteq A$ of arcs define
$\delta(S) \eqdf \sum_{(u,v) \in S} \delta(u,v)$.

A subset $S_v$ of arcs entering a vertex $v$, whose size is not
greater than the capacity of $v$, is called an \emph{improvement} to
vertex $v$ if $\delta(S_v)$ is greater than the value of $v$.  More
formally, 

\begin{definition}
A subset $S_v \subseteq A \cap (V \times \{v\})$ is
an \emph{improvement} with respect to a carpool matching $M$, if%
%the following conditions hold:%
\begin{inparaenum}[(i)]
\item $\abs{S_v} \leq c(v)$, and
\item $\delta_M(S_v) > w_M(v)$.
\end{inparaenum}
Furthermore, if there exists an improvement for a vertex $v$, we say
that vertex $v$ can be \emph{improved}.
\end{definition}

Given an arc $(u,v) \in A$, let $\inc(u,v)$ be the set of arcs that
incident $(u, v)$, namely define
\(
\inc(u,v) \eqdf (\nin[](u) \times \{u\}) \cup (\{v\} \times \nout[](v))
%~.
\).
If $S$ is a set of arcs, then $\inc(S) \eqdf \bigcup_{(u,v) \in
S} \inc(u,v)$.
%
Figure~\ref{fig:defs} depicts all the above definitions.

\begin{figure}
\centering
\add{fig-val-delta}
\caption{In this example $M$ is the set of the blue, dashed arcs.
In this case $w_M(2) = 7$, $w_M(5) = 2$, and $w_M(6) = 0$.  Also,
$\delta_M(2, 3) = 1$ and $\delta_M(6, 3) = 2$.  The set $\set{(2,3),
(6,3)}$ is an \emph{improvement} to vertex 3 and $\inc(2,3)
= \set{(1,2),(4,2),(3,5),(6,3)}$.}
\label{fig:defs}
\end{figure}

We are now ready to describe our local search algorithm, which is
called \textbf{StarImprove} (Algorithm~\ref{alg:grd}).  It starts with
an empty carpool matching $M$, and in every iteration it looks for a
vertex that can be improved.  If there exists such a vertex $v$, then
the algorithm removes the arcs that are incident on it from $M$, and
adds the arcs in $S_v$.  The algorithm terminates when no vertex can
be improved.  Figure~\ref{fig:improvement} depicts an improvement
step.

\begin{algorithm}
\caption{\textbf{StarImprove}$(G,c)$}
\label{alg:grd}
\begin{small}
$M \gets \emptyset$ \\
\Repeat{$\text{done}$}{
	$\text{done} \leftarrow \textsc{True}$ \\
	\For{$v \in V$}{
		\If{there exists an improvement $S_v$}{
			$M \leftarrow M \setminus \inc(S_v) \cup S_v$	\\
			$\text{done} \leftarrow \textsc{False}$ \\
		}
	}
}
\end{small}
\end{algorithm}

\begin{figure}
\centering
\add{fig-improvement}
\caption[]{An improvement example.}
\label{fig:improvement}
\end{figure}

We proceed to bound the approximation ratio of the algorithm, assuming
termination.

For a vertex $v$ and a set $S$ of edges entering $v$, let $\nin[S](v)
= \set{u : (u,v) \in S}$ be the set in-neighbors corresponding to $S$.

\begin{lemma}
\label{lm:no improve}
Let $M$ be a matching computed by \textbf{StarImprove}.  Let $v$ be a
vertex with no improvement, and let $S \subseteq \nin(v)$, such that
$\abs{S} \leq c(v)$, then $w(S) \leq w_M(v) + w_M(\nin[S](v))$.
\end{lemma}
\begin{proof}
If no improvement exists, then we have that \\
\(
w(S) - w_M(\nin[S](v))
=    \sum_{(u,v) \in S} (w(u,v) - w_M(u))
=    \delta_M(S) 
\leq w_M(v)
\)
\end{proof}

To bound the approximation ratio of the algorithm, we use a charging
scheme argument.

\begin{lemma}
If \textbf{StarImprove} terminates, then the computed solution is
$\half$-approximate.
\end{lemma}
\begin{proof}
Let $M$ be the matching produced by the algorithm, and let $M^*$ be an
optimal matching.  We load every vertex $v$ with an amount of money
equal to $w_M(v)$, and then we show that this is enough to pay for
every arc in the optimal matching.  Due to
Observation~\ref{lm:val-twice} the total amount of money that we use
is exactly twice the weight of $M$.

Consider a driver $v \in D_{M^*}$, and let $S = (V \times \{v\}) \cap
M^*$.  By lemma~\ref{lm:no improve} we know that $w(S) \leq w_M(v) +
w_M(\nin[S](v))$, thus we can pay for $S$, using the money on $v$ and
on $N^-_S(v)$.  Clearly, these vertices will not be charged again.
\end{proof}

We show that our analysis is tight using in Figure~\ref{fig:grd
worst}.

\begin{figure}
\centering
\add{fig-2-tight}
\caption[]{%
Consider a path with $2n + 1$ arcs, and alternating arc weights (2 and
1), if \textbf{StarImprove} selects all arcs of weight 1, then no
further improvement can be done and the value of the matching is $n +
1$, while the optimal matching has value of $2n$.}
\label{fig:grd worst}
\end{figure}

It remains to consider the running time of the algorithm.

\begin{theorem}
\label{thm:improve-bounded}
Algorithm~\textbf{StarImprove} is a $\half$-approximation algorithm
for \carpool, if edge weights are integral and polynomially bounded.
\end{theorem}
\begin{proof}
First, observe that determining if a vertex $v$ can be improved can be
done efficiently by considering the incoming arcs to $v$ in a
non-increasing order of their $\delta_M$s, and only ones with positive
values.  A vertex $v$ can be improved, then, if the $\delta$s of the
first $c(v)$ (or less) arcs sum up to more than $w_M(v)$.
%
It follows that the running time of an iteration of the for-loop is
polynomial.  Since the edge weights are integral and polynomially
bounded, the weight of an optimal carpool matching is polynomially
bounded.  The algorithm runs in polynomial time, because in each
iteration the algorithm improves the weight of the matching by at
least one or otherwise it terminates.
\end{proof}

It remains to consider the case, where weight are not polynomially
bounded. 
We prove that one can use
standard scaling and rounding to ensure a polynomial running time in
the cost of a $(1+\eps)$ factor in the approximation ratio.

\begin{theorem}
\label{thm:improve}
There exists a $(\half-\eps)$-approximation algorithm for \carpool,
for every $\eps \in (0,\half)$.
\end{theorem}

\begin{proof}
Define the weight function $w'(e)
= \floor{\frac{w(e)}{w_{\max}} \cdot \frac{m}{2\eps}}$, where
$$w_{\max} \eqdf \max_{e \in A} w(v)$$
%
Let $M^*$ and $M'$ be an optimal carpool matchings with respect to $w$
and $w'$, resp.  We have that
\begin{align*}
w'(M^*)
& =    \sum_{e \in M^*} \floor{\frac{w(e)}{w_{\max}} \cdot \frac{m}{2\eps}}
\\ & >    \sum_{e \in M^*} \paren{ \frac{w(e)}{w_{\max}} \cdot \frac{m}{2\eps} } - m
\\ & =    \frac{m}{2\eps w_{\max}} \cdot w(M^*) - m
\end{align*}

Since $w(M^*) \geq w_{\max}$, we have that 
$$
\textstyle
w'(M^*)
>    \frac{m}{2\eps w_{\max}} \cdot (w(M^*) - 2\eps w_{\max})
\geq \frac{m}{2\eps w_{\max}} \cdot (1 - 2\eps) w(M^*)
$$

By Theorem~\ref{thm:improve-bounded}, our local improvement algorithm computes a
$\half$-approx\-imate carpool matching $M$ on $(G, c, w')$, and this
matching satisfies $w'(M) \geq w'(M')/2$.  Furthermore, since
$w'(M') \geq w'(M^*)$, it follows that $w'(M) \geq w'(M^*)/2$.
Therefore
\[
\textstyle
w(M)
\geq \frac{2\eps w_{\max}}{m} w'(M) 
\geq \frac{1}{2} \frac{2\eps w_{\max}}{m} w'(M^*)
>    \frac{1-2\eps}{2} w(M^*)
\]
as required.
\end{proof}
