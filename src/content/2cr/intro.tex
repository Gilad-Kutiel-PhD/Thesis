Let $G = (V, E)$ be a graph and let $\chi : V \rightarrow C$ be a
coloring function
\footnote{W.l.o.g. we assume that $\text{img}(\chi) = C$.}
, assigning each vertex in $V$ a color in $C$.  We
say that $\chi$ is a \emph{convex coloring} of $G$ if, for every color
$c \in C$, the vertices with color $c$ induce a connected sub-graph of
$G$. Figure~\ref{fig:convex} shows an example of convex and non-convex colorings.
%
In the \textsc{Convex Recoloring} problem (abbreviated \CRP{}), 
we are given a colored graph $G_\chi$,
and we wish to find a recoloring of a minimum number of vertices of $G$,
such that the resulting coloring is convex. 
That is,
the goal is to find a convex coloring $\chi'$,
that minimizes the size of the set $\{v : \chi(v) \neq \chi'(v)\}$.
%
The \textsc{$t$-Convex Recoloring} problem ($t$-\CRP{}) is the special
case, in which the given coloring assigns the same color to at most
$t$ vertices in $G$.
Figure~\ref{fig:input-output} depicts an input and a possible output for the 
\textsc{$2$-Convex Recoloring} problem

\begin{figure}
\centering
\add{fig-convex}
\caption[Convex Recoloring - Example of convex and non-convex colorings of a graph]{
\label{fig:convex}
Example of convex and non-convex colorings of a graph.}
\end{figure}

\begin{figure}
\centering
\add{fig-input-output}
\caption[Convex Recoloring - The $2$-Convex Recoloring problem]{
\label{fig:input-output}
The $2$-Convex Recoloring problem.
}
\end{figure}

