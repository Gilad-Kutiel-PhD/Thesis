Let $G = (V, E)$ be a graph and let $\chi : V \rightarrow C$ be a
coloring function
\footnote{W.l.o.g. we assume that $\text{img}(\chi) = C$.}
, assigning each vertex in $V$ a color in $C$.  We
say that $\chi$ is a \emph{convex coloring} of $G$ if, for every color
$c \in C$, the vertices with color $c$ induce a connected sub-graph of
$G$. Figure~\ref{fig:convex} shows an example of convex and non-convex colorings.
%
In the \textsc{Convex Recoloring} problem (abbreviated \CRP{}), 
we are given a colored graph $G_\chi$,
and we wish to find a recoloring of a minimum number of vertices of $G$,
such that the resulting coloring is convex. 
That is,
the goal is to find a convex coloring $\chi'$,
that minimizes the size of the set $\{v : \chi(v) \neq \chi'(v)\}$.
%
The \textsc{$t$-Convex Recoloring} problem ($t$-\CRP{}) is the special
case, in which the given coloring assigns the same color to at most
$t$ vertices in $G$.
Figure~\ref{fig:input-output} depicts an input and a possible output for the 
\textsc{$2$-Convex Recoloring} problem

\begin{figure}
\centering
\add{fig-convex}
\caption{
\label{fig:convex}
Example of convex and non-convex colorings of a graph.}
\end{figure}

\begin{figure}
\centering
\add{fig-input-output}
\caption{
\label{fig:input-output}
The $2$-Convex Recoloring problem.
}
\end{figure}

%%%%%

\subsection*{Related work.}
%
The \textsc{Convex Recoloring} problem (\CRP{}) in trees was
introduced by Moran and Snir~\cite{MoranSnir08} and was motivated by
its relation with the concept of \emph{perfect phylogeny}.  They
proved that the problem is NP-hard~\cite{kanj2009convex}, even when
the given graph is a path.  Later, Kanj et
al.~\cite{kanj2009convex} showed that \TWOCR{} is also NP-hard on
paths.
%
Applications of \CRP{} in general graphs, 
such as multicast communication, 
were described by Kammer and Tholey~\cite{kammer2012complexity}.
%
Many variants of the problem have been intensively investigated.
%
The differences between one variant to another can be related to
%
\begin{itemize}
\item The structure of the given graph $G$.
%
The given graph can be a path, 
a tree,
a bounded treewidth graph, 
a general graph, 
and others.
  
\item Constraints on the coloring function $\chi$.
%
In a $t$-coloring at most $t$ colors are used to color a graph,
while in a $t$-\CRP{} instance at most $t$ vertices are colored using the same color.

\item Type of weight function.
%
In the weighted case, 
each vertex is associated with a weight,
and the weight of a recoloring is the total weight of recolored vertices.
%
In the unweighted case the weight of the solution is the number of recolored vertices.
%
In a third variant, 
referred to as \emph{block recoloring}~\cite{kammer2012complexity}, 
a cost is incurred for a color $c$ if at least one vertex of color $c$ was recolored. 
\end{itemize}
%
Since \CRP{} was shown to be NP-hard, it was natural to try to design both
approximation algorithms and parameterized algorithms.

%%% approximation

Moran and Snir~\cite{moran2007efficient} presented a $2$-approximation
algorithm for \CRP{} in paths and a $3$-approximation algorithm for \CRP{}
in trees.  
%
Both algorithms work for the problem with weights.
%
Bar-Yehuda, Feldman and Rawitz~\cite{BFR08} improved the
latter by providing a $(2+\varepsilon)$-approximation algorithm for
\CRP{} in trees.  
%
This result was later extended to bounded treewidth graphs by 
Kammer and Tholey~\cite{kammer2012complexity}.
%
Recently, Lima and Wakabayashi~\cite{lima2014convex} gave a 
$\frac{3}{2}$-approximation algorithm for unweighted \TWOCR{} in paths.

On the negative side, 
Kammer and Tholey~\cite{kammer2012complexity}
proved that, if vertex weights are either $0$ or $1$, 
then \TWOCR{} has no polynomial time approximation algorithm with a ratio of size 
$(1 - o(1))\ln\ln n$ 
unless $\text{NP} \subseteq \text{DTIME}(n^{O(\log\log n)})$.
In Section~\ref{sec:hard} we show 
that this variant of the problem can not be approximated at all. 
%
Camp\^{e}lo et al.~\cite{campelo2013complexity} showed that,
for $t \geq 2$, 
\CRP{} is NP-hard on $t$-colored grids. 
%
They also proved that there is no polynomial time approximation algorithm 
within a factor of $c\ln n$ for some constant $c > 0$, 
unless $\text{P} = \text{NP}$, 
for unweighted \CRP{} in bipartite graphs with $2$-colorings.

%%% FPT

Moran and Snir~\cite{MoranSnir08} presented an algorithm for \CRP{}
whose running time is $O(n^4 \cdot k (\frac{k}{\log k})^k)$,
where $k$ is the number of recolorings in an optimal solution.
%
Razgon~\cite{Razgon07} gave a $2^{O(k)} \text{poly}(n)$ time algorithm for \CRP{} in trees.
%
Ponta et al.~\cite{ponta2008speeding} 
designed several algorithms for different variants of \CRP{} in trees. 
%
For the unweighted case, 
they gave a $O(3^b \cdot b \cdot n)$ time algorithm, 
where $b$ is the number of colors that do not induce a connected subtree,
and it is bounded from above by $2k$.
%  
Bar-Yehuda, Feldman and Rawitz~\cite{BFR08} 
provided an algorithm with an upper bound of $O(n^2 + n \cdot k 2^k)$ on the running time, 
but it is not hard to verify that this bound can be improved to $O(n \cdot k 2^k)$.
%
Bodlaender et al.~\cite{BFLRRW11} showed that \CRP{}  admits a kernel of size $O(k^2)$
in trees.
%
Bachoore and Bodlaender~\cite{bachoore2006convex} 
presented an algorithm for leaf-colored trees with a running time of $O(4^k \cdot n)$. 
%
Camp\^{e}lo et al.~\cite{campelo2013complexity} proved that, 
for $t \geq 2$, 
\CRP{} is $W[2]$-hard in $t$-colored graphs.

%%%%%

\paragraph{Our results.}
%
Section~\ref{sec:hard} contains our hardness result for the weighted
version of \TWOCR{}.  We show that even the \TWOCR{} feasibility question is 
NP-hard in the case of binary weights.
%
In Section~\ref{sec:path_recoloring} we provide
an alternative definition of \TWOCR{} in terms of maximal independent
set of paths.
%
We first show that we can focus on a specific type of recoloring,
called a \emph{path recoloring},
in which each color induces a path.
%
Then we show that finding a path recoloring can be translated 
into finding an independent set of paths.
%
In Section~\ref{sec:greedy} we present a greedy algorithm for
(unweighted) \TWOCR{} in general graphs that is based on iteratively
adding a shortest path to the current independent set of paths.
%
We provide a tight analysis for the algorithm and show
that its approximation ratio is $\frac{3}{2}$.
%
This is the first time a constant-ratio approximation algorithm is
given for a variant of
\CRP{} in general graphs.
%
We also show that when $G$ is a path, 
the same algorithm yields a $\frac{5}{4}$-approximation, 
improving the previous best known approximation ratio by 
Lima and Wakabayashi~\cite{lima2014convex}.
%
In Section~\ref{sec:fpt} we use the above mentioned characterization
of \TWOCR{} to show that a problem kernel of size $4k$ can be obtained
in linear time, 
where $k$ is the cost of an optimal solution.
%
This leads to a $O(|E|) + 2^{O(k \log k)}$ time algorithm for \TWOCR{}. 
