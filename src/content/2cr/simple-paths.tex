We show that for the special case where $G$ is a path, the greedy
algorithm achieves an even better approximation ratio.  To do that we
start by observing that, in the case of a path, we can replace
Lemma~\ref{lm:num_in_conflict} with the following lemma:

\begin{lemma}
$d_p \leq \ell_p - 2$.
\end{lemma}
\begin{proof}
Given a path $p \in I$, we show that the number of paths in $I^*$ that
are in conflict with $p$ is at most $\ell_p - 2$.  This is true,
because if $G$ is a  path, then for every path that is in
conflict with $p$, we can now associate a vertex that is not one of
the endpoints of $p$.  If the conflict is direct, then the two paths
must overlap, thus associate the other path's endpoint.  If the
conflict is indirect, then this must be due to some color other than
the one on the endpoints of $p$, or else this is a direct conflict.
{}\end{proof}

Using the above lemma we can strengthen
Lemma~\ref{lemma:alpha-squared}.

\begin{lemma}
\label{lemma:strong-alpha-squared}
$|D| \geq (\alpha^2+\alpha)|I|$.
\end{lemma}
\begin{proof}
The proof is similar to the proof of Lemma~\ref{lemma:alpha-squared},
where the main difference is that
\[
\sum_{p' \in I^*} \ell_{p'}
\geq \sum_{p \in I} d_p \cdot \ell_p
\geq \sum_{p \in I} (d_p^2 + 2d_p)
=    \sum_{p \in I} d_p^2 + 2|I^*|
~,
\]
which means that 
\[
|D| 
\geq \sum_{p' \in I^*} \ell_{p'} - |I^*|
\geq \sum_{p \in I} d_p^2 + |I^*|
=    \sum_{p \in I} d_p^2 + \alpha |I|
\geq (\alpha^2 + \alpha) |I|
~,
\]
and the lemma follows.
{}\end{proof}

In this case only one upper bound suffices.

\begin{theorem}
Greedy is a $\frac{5}{4}$-approximation algorithm for
\TWOCR{} on paths.
\end{theorem}
\begin{proof}
Using Lemma~\ref{lemma:strong-alpha-squared} we get that
\[
r	=    
\frac{|D| - |I|}{|D| - \alpha \cdot |I|}
\leq 
\frac{
	(\alpha^2 + \alpha)|I| - |I|
}{
	(\alpha^2 + \alpha)|I| - \alpha |I|
}
=    \frac{\alpha^2 + \alpha - 1}{\alpha^2}
\leq \frac{5}{4}
,
\]
as required.
{}\end{proof}

To see that our analysis is tight, consider the instance depicted in
Fig.~\ref{fig:tight_path}. On this colored path, the greedy algorithm might
recolor $\frac{5}{4}$ times more vertices than the optimal recoloring.


\begin{figure}[t]
\centering
\begin{tikzpicture}
\node(1) at (0bp,0) [black node] {1};
\node(2) at (1,0) [red node]{2}; 
\node (3) at (2,0) [blue node] {3}; 
\node (4) at (3,0) [green node] {4}; 
\node (5) at (4,0) [red node] {5}; 
\node (6) at (5,0) [orange node] {6}; 
\node (7) at (6,0) [green node] {7}; 
\node (8) at (7,0) [violet node] {8}; 
\node (9) at (8,0) [blue node] {9}; 
\node (10) at (9,0) [black node] {10}; 
\node (11) at (10,0) [violet node] {11}; 
\node (12) at (11,0) [orange node] {12}; 
\draw 
(1) -- (2) -- (3) -- (4) -- (5) -- (6) -- 
(7) -- (8) -- (9) -- (10) -- (11) -- (12);
\end{tikzpicture}

\caption{
The greedy algorithm might choose to color the path (2, 3, 4, 5), 
then it must recolor one of the vertices \{1, 10\}, 
one of \{6, 12\}, 
and one of \{8, 11\}, 
a total of five recolored vertices,
while an optimal recoloring can recolor the paths (4, 5, 6, 7) and
8, 9, 10, 11), a total of four recolored vertices.}
\label{fig:tight_path}
\end{figure}
