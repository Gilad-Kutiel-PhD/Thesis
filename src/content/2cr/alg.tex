\label{sec:greedy}

In this section we describe a natural greedy algorithm to construct a maximal
independent set of colorable paths, we discuss how this algorithm can be implemented,
and define the corresponding path-recoloring it computes.
%
We show that the approximation ratio of the greedy algorithm is $\frac{3}{2}$.
%
For the special case where the input graph is a path, 
we show that the greedy algorithm achieves a $\frac{5}{4}$ approximation ratio.

%%%%%

\subsection{The Algorithm}

While we are not attempting to achieve an approximation to the size
of the maximum independent set of colorable paths,
the alternative definition given at the end of the previous section
leads us to a natural greedy algorithm: 
choose the shortest colorable path that 
is not in conflict with colorable paths already been chosen and add it
to an independent set of colorable paths.
%
A formal description of this algorithm is given in
Algorithm~\ref{alg:conv:greedy}.

\begin{algorithm}
\begin{algorithmic}

\State $I \gets \emptyset$

\While{there is a colorable path, not in conflict with $I$}
\State add to $I$ a shortest colorable path, not in conflict with $I$ 
\EndWhile
\\
\Return the path-recoloring corresponding to $I$


\end{algorithmic}
\caption{Greedy algorithm for 2-CR.}
\label{alg:conv:greedy}
\end{algorithm}

We now describe how a shortest colorable path can be found.
%
To do that, 
at the initialization of the algorithm, 
all singletons and connected pairs should be removed from the graph.  
%
Every colorable path that is added to $I$ should be also removed from the graph.  
%
In addition, 
after each path removal, 
one should also remove all vertices with a unique color in the remaining graph.  
%
On the remaining graph, 
a shortest path, 
among those vertices of the same color, 
is guaranteed to be colorable and independent of $I$.
%
In particular,
it cannot contain two vertices that are colored by the same color, 
since such a path is not a shortest path.

%%%%%

\subsection{Analysis for General Graphs}

We now analyze the greedy algorithm and prove that it recolors at most 
$\frac{3}{2} \cdot k$ vertices, 
where $k$ is the minimum number of vertices that must be recolored to achieve a convex
coloring.
%
We also show that the analysis is tight even in trees.

Recall that $D$ is the set of disconnected pairs in $G$, 
$I \subseteq D$ is the set of independent paths colored by the greedy algorithm, 
and let $I^* \subseteq D$ be the set of independent paths colored by an arbitrary, 
but fixed,
optimal path-recoloring.  
%
Define $\alpha := \frac{|I^*|}{|I|}$,
then by Lemma~\ref{lm:cost} the ratio between the number of 
vertices recolored by the greedy algorithm and the number of vertices
recolored by an optimal path-recoloring is:
\[
r := \frac{|D| - |I|}{|D| - |I^*|}
= \frac{|D| - |I|}{|D| - \alpha |I|}
\].
Recall that due to Lemma~\ref{lm:does_not_recolor_connected_pair} we
may assume that an optimal solution does not recolor singletons or
disconnected pairs.
%
We now analyze the relationship between $|I|$, $|D|$ and $\alpha$.

Let $\ell_p$ be the number of vertices on a path $p$ (including its endpoints), 
and observe that:

\begin{lemma}
\label{lemma:assign}
If $p' \in I^* \setminus I$, then there is a path $p \in I$ that is in
conflict with $p'$, and $\ell_p \leq \ell_{p'}$.
\end{lemma}
\begin{proof}
Consider the running of the greedy algorithm at the most recent point,
where $I$ contains no path with a distance greater than $\ell_{p'}$.  If,
at this point, there is no path in $I$ that is in conflict with $p'$
then nothing prevents the greedy algorithm from adding $p'$ to $I$.
{}\end{proof}

Next, we assign each optimal path to a greedy path.
%
Given a path $p' \in I^*$, the \emph{conflict source} of $p'$ is $p'$
itself if $p' \in I$, otherwise it is an arbitrary shortest path in
$I$ that is in conflict with $p'$.
%
Notice that this is a many-to-one assignment, that is, several optimal
paths can be assigned to a single greedy path.
%
For a path $p \in I$, the set of paths in $I^*$ such that $p$ is their
conflict source is denoted as $N(p)$.
The members of $N(p)$ are called the \emph{neighbours} of $p$.
Figure~\ref{fig:greedy-vs-opt} depicts the mapping between paths in 
$I^*$ and their conflict source in $I$.

\begin{figure}
\centering
\add{fig-greedy-vs-opt}
\caption{
\label{fig:greedy-vs-opt}
A comparison between $I$ (the greedy solution) and $I^*$ (an optimal solution).
Each path in $I^*$ is mapped to its conflict source in $I$.
}
\end{figure}

Due to Lemma~\ref{lemma:assign} we have that:

\begin{observation}
\label{co:dpLeqDp'}
For every path $p \in I$, if $p' \in N(p)$, then $\ell_p \leq \ell_{p'}$.
\end{observation}

Denote $d_p := |N(p)|$ and refer to $d_p$ as the \emph{degree} of $p$.
Our next goal is to find an upper bound to $d_p$, for $p \in I$.

\begin{lemma}
\label{lm:num_in_conflict}
For every path $p \in I$, $d_p \leq \ell_p - 1$.
\end{lemma}
\begin{proof}
We show that the number of paths in $I^*$ that
are in conflict with $p$ is at most $\ell_p - 1$.
%
Associate a vertex in $p$ to every path that is in conflict with $p$.
For a path that is in direct conflict with $p$ associate a common
vertex of these paths, for a path that is in indirect conflict due to
color $c$, associate the vertex that has color $c$ in $p$.  Recall
that no two paths in $I^*$ are in conflict, and observe that
associating the same vertex to more than one path implies the
existence of such conflict. 
Finally, since the end points of $p$ have the same color, 
at most one of them can be associated with a path in $I^*$, 
or otherwise, $I^*$ is not independent.
{}\end{proof}

In what follows we obtain two lower bounds on $|D|$, which translate
into two upper bounds on the approximation ratio $r$.

\begin{lemma}
\label{lemma:kernel}
$|D| \geq 2|I^*|$.
\end{lemma}
\begin{proof}
Every path $p \in I^*$ has $l_p \geq 3$, 
since its end points form a disconnected pair. 
Also, the internal vertices must be part of another disconnected pair.
Thus, we can observe the existence of at least two disconnected pairs
for every path in $I^*$.  We do not count the same disconnected pair
twice, or otherwise, $I^*$ is not independent.
{}\end{proof}


\begin{observation}
\label{obs:sum}
$\sum_{p \in I}{d_p} = |I^*| = \alpha \cdot |I|$
\end{observation}
\begin{proof}
By definition every path $p' \in I^*$ has one, and only one, conflict
source in $I$, thus, 
there exists exactly one path $p \in I$ such that $p' \in N(p)$.
{}\end{proof}


\begin{lemma}
\label{lm:avg_ineq}
$\frac{\sum_{p \in I}{d_p^2}}{|I|} \geq \alpha^2$.
\end{lemma}
\begin{proof}
Due to Observation~\ref{obs:sum} we have that the average degree of
paths in $I$ is $\alpha$, i.e. $\frac{\sum_{p \in I}{d_p}}{|I|}
= \alpha$.  The lemma follows from the Jensen's inequality.
{}\end{proof}


\begin{lemma}
\label{lemma:alpha-squared}
$|D| \geq \alpha^2|I|$.
\end{lemma}
\begin{proof}
Let $v$ be a vertex in a path $p' \in I^*$.
Then $v$ must be a part of
a disconnected pair, otherwise $v$ is a singleton or part of a
connected pair and thus the considered recoloring is not a
path-recoloring.  Observe, also, that aside from the two endpoints of
each path, no two vertices of the same disconnected pair can belongs to paths
in $I^*$ (or otherwise $I^*$ is not independent).  
Thus, we can count $\ell_{p'} - 1$ disconnected pairs for every path $p'$ in
$I^*$.  
Hence
\[
|D| \geq \sum_{p' \in I^*} (\ell_{p'} - 1) = \sum_{p' \in I^*}{\ell_{p'}} -
|I^*|
\].

Consider a path $p \in I$. 
Recall from Observation~\ref{co:dpLeqDp'} that if $p' \in N(p)$ 
then $\ell_{p'} \geq \ell_p$.
%
Hence, 
$\sum_{p' \in N(p)}{\ell_{p'}} \geq d_p \cdot \ell_p$, 
and it follows that
\[
\sum_{p' \in I^*} \ell_{p'}
\geq \sum_{p \in I} d_p \cdot \ell_p
\geq \sum_{p \in I} d_p(d_p+1)
=    \sum_{p \in I} d_p^2 + |I^*|
\]
,where the second inequality is due to Lemma~\ref{lm:num_in_conflict}
and the equality is due to Observation~\ref{obs:sum}.
Hence, by Lemma~\ref{lm:avg_ineq} we have that
\(
|D| \geq \sum_{p \in I} d_p^2 \geq \alpha^2 |I|
\).
{}\end{proof}

We can now obtain two upper bounds on the approximation ratio $r$ as a
function of $\alpha$. 
%Figure~\ref{fig:upper_bound} depicts these two bounds.

\begin{theorem}
The greedy algorithm is a $\frac{3}{2}$-approximation algorithm for
\TWOCR{}.
\end{theorem}
\begin{proof}
Using Lemma~\ref{lemma:alpha-squared} and the fact that $\alpha \geq 1$ we get
that
$$
r
=    \frac{|D| - |I|}{|D| - \alpha \cdot |I|}
\leq \frac{\alpha ^ 2 \cdot |I| - |I|}{\alpha ^ 2 \cdot |I| - \alpha \cdot |I|}
=    \frac{\alpha ^ 2 - 1}{\alpha ^ 2 - \alpha}
=    \frac{\alpha+1}{\alpha}
$$
,and from Lemma~\ref{lemma:kernel} we get that
$$
r
=    \frac{|D| - |I|}{|D| - \alpha \cdot |I|}
\leq \frac{2 \alpha \cdot |I| - |I|}{2 \alpha \cdot |I| - \alpha \cdot |I|}
=    \frac{2 \alpha - 1}{\alpha}
$$,
Figure~\ref{fig:upper_bound} depicts the two upper bounds on $r$.
Putting the two bounds together, it follows that
\[
r
\leq \min \left\{\frac{\alpha+1}{\alpha} , \frac{2\alpha-1}{\alpha} \right\}
\leq    \frac{3}{2}
\]
,as required.
\end{proof}



\begin{figure}
\centering
\add{fig-2bounds}
\caption{
\label{fig:upper_bound}
An upper bound on the approximation ratio as a function of $\alpha$.
The dashed (blue) and the dotted (orange) plots are the two bounds for the general case.
The solid (green) plot is the bound for 2-cr on paths.
}
\end{figure}

\begin{figure}
\centering
\add{fig-greedy-tight}
\caption{
Greedy might choose to color the path (1, 3, 2), 
then it must recolor one of the vertices \{5, 6\} 
and one of the vertices \{7, 8\}, 
a total of three recolored vertices, 
while an optimal recoloring can color two paths: (5, 1, 6) and (7, 3, 8), 
a total of two recolored vertices.}
\label{fig:tight}
\end{figure}

We show that our analysis is tight even for colored trees, using the
instance depicted in Figure~\ref{fig:tight}, 
which consists of a colored graph where the greedy algorithm might recolor
$\frac{3}{2}$ times more vertices than the optimal convex recoloring.
%
We note that one can simply duplicate the instance using new colors for
each copy in order to construct an arbitrary large graph with the same
approximation ratio.

%%%%%

\subsection{Analysis for Paths}

For the special case when the input graph is a path, 
we show that the greedy algorithm achieves a $\frac{5}{4}$ approximation ratio.
We show that the analysis is tight for this special case as well.

We start our analysis  by observing that, in the case of a path, we can replace
Lemma~\ref{lm:num_in_conflict} with the following lemma:

\begin{lemma}
For every path $p \in I$, $d_p \leq \ell_p - 2$.
\end{lemma}
\begin{proof}
We show that the number of paths in $I^*$ that
are in conflict with $p$ is at most $\ell_p - 2$.  
This is true, because if $G$ is a  path, 
then for every path that is in conflict with $p$, 
we can now associate a (unique) vertex that is not one of the endpoints of $p$.  
If the conflict is direct, then the two paths
must overlap, thus associate the other path's endpoint.  If the
conflict is indirect, then this must be due to some color other than
the one on the endpoints of $p$, or else this is a direct conflict.
{}\end{proof}

Using the above lemma we can strengthen
Lemma~\ref{lemma:alpha-squared}.

\begin{lemma}
\label{lemma:strong-alpha-squared}
$|D| \geq (\alpha^2+\alpha)|I|$.
\end{lemma}
\begin{proof}
The proof is similar to the proof of Lemma~\ref{lemma:alpha-squared},
where the main difference is that
\[
\sum_{p' \in I^*} \ell_{p'}
\geq \sum_{p \in I} d_p \cdot \ell_p
\geq \sum_{p \in I} (d_p^2 + 2d_p)
=    \sum_{p \in I} d_p^2 + 2|I^*|
\]
,which means that 
\[
|D| 
\geq \sum_{p' \in I^*} \ell_{p'} - |I^*|
\geq \sum_{p \in I} d_p^2 + |I^*|
=    \sum_{p \in I} d_p^2 + \alpha |I|
\geq (\alpha^2 + \alpha) |I|
\]
,and the lemma follows.
{}\end{proof}

In this case only one upper bound suffices.

\begin{theorem}
Greedy is a $\frac{5}{4}$-approximation algorithm for
\TWOCR{} on paths.
\end{theorem}
\begin{proof}
Using Lemma~\ref{lemma:strong-alpha-squared} we get that
\begin{align*}
r	& =    
\frac{|D| - |I|}{|D| - \alpha \cdot |I|}
\\&
\leq 
\frac{
	(\alpha^2 + \alpha)|I| - |I|
}{
	(\alpha^2 + \alpha)|I| - \alpha |I|
}
\\&
=    \frac{\alpha^2 + \alpha - 1}{\alpha^2}
\\&
=    \frac{5}{4} - \frac{\alpha^2/4 - \alpha + 1}{\alpha^2}
\\&
=    \frac{5}{4} - \frac{(\alpha/2-1)^2}{\alpha^2}
\leq \frac{5}{4}
\end{align*}
,as required.
{}\end{proof}

Figure~\ref{fig:upper_bound} compares this last bound with the bounds of the general case. 
To see that our analysis is tight, consider the instance depicted in
Figure~\ref{fig:tight_path}. On this colored path, the greedy algorithm might
recolor $\frac{5}{4}$ times more vertices than the optimal recoloring.


\begin{figure}[t]
\centering
\add{fig-simple-paths}
\caption{
The greedy algorithm might choose to color the path $2 \ldots 5$, 
then it must recolor one of the vertices \{1, 10\} 
(the path $1 \ldots 10$ contains the path $2 \ldots 5$), 
one of \{6, 12\}
(the paths $2 \ldots 5$ and $6 \ldots 12$ are in indirect conflict), 
and one of \{8, 11\}
(the paths $2 \ldots 5$ and $8 \ldots 11$ are in indirect conflict),
a total of five recolored vertices,
while an optimal recoloring can recolor the paths $4 \ldots 7$ and
$8 \ldots 11$, a total of four recolored vertices.}
\label{fig:tight_path}
\end{figure}
