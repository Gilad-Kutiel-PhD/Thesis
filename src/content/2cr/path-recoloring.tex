As a first step we show that it may be assumed that all 
colors retain at least one representative.
%
In other words,
we show that there is always an optimal convex recoloring $\chi'$ 
that does not recolor singletons and recolors at most one vertex of every pair. 

Given a colored graph $G_\chi$ and a recoloring $\chi'$, 
a vertex $v$ \emph{retains} its color if $\chi(v)=\chi'(v)$. 
%
We say that $\chi'$ \emph{retains} a pair $p$, 
if both vertices of $p$ retain their color.  
%
The recoloring $\chi'$ \emph{retains} a color $c \in C$,
if there exists a vertex $v \in G$ such that ${\chi'(v) = \chi(v) = c}$
\footnote{
Note that this definition is different from the one given
in~\cite{kanj2009convex}.
}.
%
If a recoloring retains all the colors of a graph, 
we refer to it as a \emph{retains-all} recoloring (see Figure~\ref{fig:retains-all}).
Observe that a retains-all recoloring does not recolor singletons.

\begin{figure}
\centering
\add{fig-retains-all}
\caption{
\label{fig:retains-all}
Example of retains-all and non-retains-all (optimal) convex recolorings.
}
\end{figure}
%
We show that there exists a retains-all optimal recoloring.

\begin{lemma}
\label{lm:retains-all}
For every colored graph $G_\chi$, 
there exists a retains-all optimal convex recoloring.
\end{lemma}

\begin{proof}
Consider an optimal convex recoloring $\chi'$ that retains a maximum
number of colors over all optimal, 
convex recolorings of $G$.
%
Assume for contradiction that $\chi'$ does not retain a color $c$, 
and let $v$ be a vertex such that $\chi(v) = c$ and $\chi'(v) = c' \neq c$.
%
Without loss of generality we assume that $c$ is not used by $\chi'$ 
(otherwise, we can recolor each vertex in $G_{\chi'}[c]$ using a new unique color.
% \footnote{
% Unique colors are used for simplicity. 
% The new colors can be replaced by original colors, 
% by iteratively recoloring a vertex with unique color 
% using the color of an adjacent vertex which is colored by an original color
% }).
%
We define a recoloring $\chi''$ by considering $G_{\chi'}[c']$.
%
First, 
define $\chi''(v) = c$.
Next, 
if $G_{\chi'}[c']$ contains a vertex $u$ such that $\chi(u) = c'$, 
then define $\chi''(u) = c'$.
%
Each vertex of the remaining vertices in $G_{\chi'}[c']$ is colored by $\chi''$ 
using a unique new color.
%
Observe that if there exists a second vertex 
$u' \neq u$ such that $\chi'(u') = \chi(u') = c'$, 
then $\chi''$ recolors it.
%
Finally, 
$\chi''(x) = \chi'(x)$ for any vertex $x$ not in $G_{\chi'}[c']$.

$\chi''$ is convex since every vertex in
$G_{\chi'}[c']$ is colored by a unique color. 
%
$\chi''$ recolors at most as many vertices as $\chi'$ 
since it may recolor $u$'s mate (if it exists), 
but it avoids the recoloring of $v$. 
%
Finally, 
$\chi''$ retains more colors than $\chi'$ 
since it retains $c$.
%
Thus, 
we obtained an optimal recoloring that retains more colors than $\chi'$.  
%
A contradiction.
{}\end{proof}

Next we show that we need not recolor connected pairs.

\begin{lemma}
\label{lm:does_not_recolor_connected_pair}
For every colored graph $G_\chi$ there exists a retains-all,
optimal,
convex recoloring that does not recolor any connected pair.
\end{lemma}

\begin{proof}
Consider an optimal retains-all convex recoloring $\chi'$ 
that retains the maximum number of connected pairs, 
over all optimal, 
retains-all recolorings of $G$.  
%
Assume for contradiction that $\chi'$ 
recolors one of the vertices of a connected pair $\{u, v\}$, 
that is, 
without loss of generality,
$\chi(u) = \chi(v) = c$ and $\chi'(v) = c' \neq c$.
%
We can use a similar argument to the one used in the proof of Lemma~\ref{lm:retains-all}.
%
The difference is in the definition of $\chi''$.  
%
We let $\chi''(v) = c$. 
Next, if $G_{\chi'}[c']$ contains a vertex $w \neq v$ such that $\chi(w) = c'$,
then define $\chi''(w) = c'$, 
also, if $G_{\chi'}[c']$ contains a connected pair $u_1, u_2$ 
such that $\chi(u_1) = \chi(u_2) = \chi'(u_1) = \chi'(u_2) = c'$, 
then we define $\chi''(u_1) = \chi''(u_2) = c'$.
Each vertex of the remaining vertices in $G_{\chi'}[c']$ is colored by $\chi''$ 
using a unique new color.
%
Observe that $\chi''$ is convex, since apart from the possible
connected pair $u_1$ and $u_2$ which are colored by $c'$, every vertex
in $G_{\chi'}[c']$ is colored by a unique color.
%
Observe also that $\chi''$ is a retains-all recoloring that recolors at most as
many vertices as $\chi'$, and retains more connected pairs than $\chi'$. 
%
A contradiction.
{}\end{proof}

We are now ready to define \emph{path-recolorings}.
%
Given a colored graph $G_\chi$ and a convex recoloring $\chi'$, 
we say that $\chi'$ \emph{path-recolors} $G$ with respect to $c \in C$ 
if there is a Hamiltonian path 
\footnote{
Note that we are not concerned with the decision problem whether a given recoloring 
is a path-recoloring or not.
Later in the paper we will introduce a simple algorithm that computes a path-recoloring. 
} 
in $G_{\chi'}[c]$: 
$u, \dots, v$ such that $\chi(u) = \chi(v) = c$.
%
A special case of this definition is when  $G_{\chi'}[c]$ 
is a single vertex $v$ and $\chi(v) = c$.
%
We say that $\chi'$ is a \emph{path-recoloring} if:
\begin{enumerate}
	\item $\chi'$ does not recolor any connected pair, and 
	\item $\chi'$ path-recolors $G$ with respect to every $c \in C$.
\end{enumerate}
Clearly, every path-recoloring also retains all colors.

\begin{lemma}
\label{lm:optimal_path_recoloring}
For every colored graph $G_\chi$ 
there exists an optimal recoloring that is a path recoloring.
\end{lemma}

\begin{proof}
Let $\chi'$ be an optimal, 
retains-all recoloring that does not recolor any connected pair 
(whose existence was shown in Lemma~\ref{lm:does_not_recolor_connected_pair}) 
that path-recolors $G$ with respect to a subset of $C$ of a maximum possible size. 
%
Assume for contradiction that $\chi'$ is not a path-recoloring.  
%
Hence there is a color $c$ such that $\chi'$ does not path-recolor $G$ with $c$.
%
Consider $G_{\chi'}[c]$, 
and assume for now that there are two vertices $u$
and $v$ in $G_{\chi'}[c]$ such that $\chi(u) = \chi(v) = c$.  
%
Fix a simple path from $u$ to $v$ in $G_{\chi'}[c]$, 
and let $\chi''$ be identical to $\chi'$ with the following modification: 
$\chi''$ assigns a unique color for every vertex in $G_{\chi'}[c]$ 
that is not on the simple path from $u$ to $v$.
%
Clearly, 
$\chi''$ is an optimal recoloring that path-recolors 
$G$ with respect to more colors than $\chi'$.  
%
A contradiction.
	
Now,
if there is at most one vertex $v$ in $G_{\chi'}[c]$ such that $\chi(v) = c$ 
then consider a recoloring $\chi''$ that is identical to $\chi'$ 
except it assigns a unique color to every vertex in $G_{\chi'}[c]$ that is not $v$.
%
This time, 
again, 
we found an optimal recoloring that path-recolors $G$ with more colors than $\chi'$.  
%
A contradiction.
{}\end{proof}

Henceforth, 
whenever we refer to a recoloring, 
we assume that it is a path-recoloring.
In particular, 
we assume that for every disconnected pair in $G$ with color $c$, 
a recoloring either:
\begin{inparaenum}[(i)]
	\item colors exactly a path between the disconnected pair with color $c$, or
	\item changes the color of exactly one of its vertices. 
\end{inparaenum}
