\subfloat[an example of input to the problem]{
\label{sub:in}
\begin{tikzpicture}[x=1.1cm, y=1.1cm]

\begin{scope}[every node/.style={red node}]
\node(1) at(2,2) 		{1};
\node(2) at(2,-1)		{2};
\end{scope}

\begin{scope}[every node/.style={blue node}]
\node(3) at(1,-2) 		{3};
\node(4) at(1,0)		{4};
\end{scope}

\begin{scope}[every node/.style={green node}]
\node(5) at(3,0) 		{5};
\node(6) at(2,1)		{6};
\end{scope}

\begin{scope}[every node/.style={orange node}]
\node(7) at(1,-1) 		{7};
\node(8) at(4,0)		{8};
\end{scope}

\begin{scope}[every node/.style={violet node}]
\node(9) at(0,0) 		{9};
\end{scope}

\draw (9) -- (4);
\draw (4) -- (6);
\draw (6) -- (5);
\draw (5) -- (2);
\draw (2) -- (7);
\draw (7) -- (4);
\draw (7) -- (3);
\draw (6) -- (1);
\draw (5) -- (8);

\end{tikzpicture}}
\hfill
\subfloat[a possible output with cost of 2]{
\label{sub:out}
\begin{tikzpicture}[x=1.1cm, y=1.1cm]

\begin{scope}[every node/.style={red node}]
\node(2) at(2,-1)		{2};
\end{scope}

\begin{scope}[every node/.style={blue node}]
\node(3) at(1,-2) 		{3};
\node(4) at(1,0)		{4};
\node[dashed](7) at(1,-1) 		{7};
\end{scope}

\begin{scope}[every node/.style={green node}]
\node[dashed](1) at(2,2) 		{1};
\node(5) at(3,0) 		{5};
\node(6) at(2,1)		{6};
\end{scope}

\begin{scope}[every node/.style={orange node}]
\node(8) at(4,0)		{8};
\end{scope}

\begin{scope}[every node/.style={violet node}]
\node(9) at(0,0) 		{9};
\end{scope}

\draw (9) -- (4);
\draw (4) -- (6);
\draw (6) -- (5);
\draw (5) -- (2);
\draw (2) -- (7);
\draw (7) -- (4);
\draw (7) -- (3);
\draw (6) -- (1);
\draw (5) -- (8);

\end{tikzpicture}}
