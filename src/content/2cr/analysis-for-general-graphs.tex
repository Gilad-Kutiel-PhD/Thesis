

We now analyze the greedy algorithm and prove that it recolors at most 
$\frac{3}{2} \cdot k$ vertices, 
where $k$ is the minimum number of vertices that must be recolored to achieve a convex
coloring.
%
We also show that the analysis is tight even in trees.

Recall that $D$ is the set of disconnected pairs in $G$, 
$I \subseteq D$ is the set of independent paths colored by the greedy algorithm, 
and let $I^* \subseteq D$ be the set of independent paths colored by an arbitrary, 
but fixed,
optimal path-recoloring.  
%
Define $\alpha := \frac{|I^*|}{|I|}$,
then by Lemma~\ref{lm:cost} the ratio between the number of 
vertices recolored by the greedy algorithm and the number of vertices
recolored by an optimal path-recoloring is:
\[
r := \frac{|D| - |I|}{|D| - |I^*|}
= \frac{|D| - |I|}{|D| - \alpha |I|}
\].
Recall that due to Lemma~\ref{lm:does_not_recolor_connected_pair} we
may assume that an optimal solution does not recolor singletons or
disconnected pairs.
%
We now analyze the relationship between $|I|$, $|D|$ and $\alpha$.

Let $\ell_p$ be the number of vertices on a path $p$ (including its endpoints), 
and observe that:

\begin{lemma}
\label{lemma:assign}
If $p' \in I^* \setminus I$, then there is a path $p \in I$ that is in
conflict with $p'$, and $\ell_p \leq \ell_{p'}$.
\end{lemma}
\begin{proof}
Consider the running of the greedy algorithm at the most recent point,
where $I$ contains no path with a distance greater than $\ell_{p'}$.  If,
at this point, there is no path in $I$ that is in conflict with $p'$
then nothing prevents the greedy algorithm from adding $p'$ to $I$.
{}\end{proof}

Next, we assign each optimal path to a greedy path.
%
Given a path $p' \in I^*$, the \emph{conflict source} of $p'$ is $p'$
itself if $p' \in I$, otherwise it is an arbitrary shortest path in
$I$ that is in conflict with $p'$.
%
Notice that this is a many-to-one assignment, that is, several optimal
paths can be assigned to a single greedy path.
%
For a path $p \in I$, the set of paths in $I^*$ such that $p$ is their
conflict source is denoted as $N(p)$.
The members of $N(p)$ are called the \emph{neighbours} of $p$.
Figure~\ref{fig:greedy-vs-opt} depicts the mapping between paths in 
$I^*$ and their conflict source in $I$.

\begin{figure}
\centering
\add{fig-greedy-vs-opt}
\caption[Convex Recoloring - greedy versus optimal solutions]{
\label{fig:greedy-vs-opt}
A comparison between $I$ (the greedy solution) and $I^*$ (an optimal solution).
Each path in $I^*$ is mapped to its conflict source in $I$.
}
\end{figure}

Due to Lemma~\ref{lemma:assign} we have that:

\begin{observation}
\label{co:dpLeqDp'}
For every path $p \in I$, if $p' \in N(p)$, then $\ell_p \leq \ell_{p'}$.
\end{observation}

Denote $d_p := |N(p)|$ and refer to $d_p$ as the \emph{degree} of $p$.
Our next goal is to find an upper bound to $d_p$, for $p \in I$.

\begin{lemma}
\label{lm:num_in_conflict}
For every path $p \in I$, $d_p \leq \ell_p - 1$.
\end{lemma}
\begin{proof}
We show that the number of paths in $I^*$ that
are in conflict with $p$ is at most $\ell_p - 1$.
%
Associate a vertex in $p$ to every path that is in conflict with $p$.
For a path that is in direct conflict with $p$ associate a common
vertex of these paths, for a path that is in indirect conflict due to
color $c$, associate the vertex that has color $c$ in $p$.  Recall
that no two paths in $I^*$ are in conflict, and observe that
associating the same vertex to more than one path implies the
existence of such conflict. 
Finally, since the end points of $p$ have the same color, 
at most one of them can be associated with a path in $I^*$, 
or otherwise, $I^*$ is not independent.
{}\end{proof}

In what follows we obtain two lower bounds on $|D|$, which translate
into two upper bounds on the approximation ratio $r$.

\begin{lemma}
\label{lemma:kernel}
$|D| \geq 2|I^*|$.
\end{lemma}
\begin{proof}
Every path $p \in I^*$ has $l_p \geq 3$, 
since its end points form a disconnected pair. 
Also, the internal vertices must be part of another disconnected pair.
Thus, we can observe the existence of at least two disconnected pairs
for every path in $I^*$.  We do not count the same disconnected pair
twice, or otherwise, $I^*$ is not independent.
{}\end{proof}


\begin{observation}
\label{obs:sum}
$\sum_{p \in I}{d_p} = |I^*| = \alpha \cdot |I|$
\end{observation}
\begin{proof}
By definition every path $p' \in I^*$ has one, and only one, conflict
source in $I$, thus, 
there exists exactly one path $p \in I$ such that $p' \in N(p)$.
{}\end{proof}


\begin{lemma}
\label{lm:avg_ineq}
$\frac{\sum_{p \in I}{d_p^2}}{|I|} \geq \alpha^2$.
\end{lemma}
\begin{proof}
Due to Observation~\ref{obs:sum} we have that the average degree of
paths in $I$ is $\alpha$, i.e. $\frac{\sum_{p \in I}{d_p}}{|I|}
= \alpha$.  The lemma follows from the Jensen's inequality.
{}\end{proof}


\begin{lemma}
\label{lemma:alpha-squared}
$|D| \geq \alpha^2|I|$.
\end{lemma}
\begin{proof}
Let $v$ be a vertex in a path $p' \in I^*$.
Then $v$ must be a part of
a disconnected pair, otherwise $v$ is a singleton or part of a
connected pair and thus the considered recoloring is not a
path-recoloring.  Observe, also, that aside from the two endpoints of
each path, no two vertices of the same disconnected pair can belongs to paths
in $I^*$ (or otherwise $I^*$ is not independent).  
Thus, we can count $\ell_{p'} - 1$ disconnected pairs for every path $p'$ in
$I^*$.  
Hence
\[
|D| \geq \sum_{p' \in I^*} (\ell_{p'} - 1) = \sum_{p' \in I^*}{\ell_{p'}} -
|I^*|
\].

Consider a path $p \in I$. 
Recall from Observation~\ref{co:dpLeqDp'} that if $p' \in N(p)$ 
then $\ell_{p'} \geq \ell_p$.
%
Hence, 
$\sum_{p' \in N(p)}{\ell_{p'}} \geq d_p \cdot \ell_p$, 
and it follows that
\[
\sum_{p' \in I^*} \ell_{p'}
\geq \sum_{p \in I} d_p \cdot \ell_p
\geq \sum_{p \in I} d_p(d_p+1)
=    \sum_{p \in I} d_p^2 + |I^*|
\]
,where the second inequality is due to Lemma~\ref{lm:num_in_conflict}
and the equality is due to Observation~\ref{obs:sum}.
Hence, by Lemma~\ref{lm:avg_ineq} we have that
\(
|D| \geq \sum_{p \in I} d_p^2 \geq \alpha^2 |I|
\).
{}\end{proof}

We can now obtain two upper bounds on the approximation ratio $r$ as a
function of $\alpha$. 
%Figure~\ref{fig:upper_bound} depicts these two bounds.

\begin{theorem}
The greedy algorithm is a $\frac{3}{2}$-approximation algorithm for
\TWOCR{}.
\end{theorem}
\begin{proof}
Using Lemma~\ref{lemma:alpha-squared} and the fact that $\alpha \geq 1$ we get
that
$$
r
=    \frac{|D| - |I|}{|D| - \alpha \cdot |I|}
\leq \frac{\alpha ^ 2 \cdot |I| - |I|}{\alpha ^ 2 \cdot |I| - \alpha \cdot |I|}
=    \frac{\alpha ^ 2 - 1}{\alpha ^ 2 - \alpha}
=    \frac{\alpha+1}{\alpha}
$$
,and from Lemma~\ref{lemma:kernel} we get that
$$
r
=    \frac{|D| - |I|}{|D| - \alpha \cdot |I|}
\leq \frac{2 \alpha \cdot |I| - |I|}{2 \alpha \cdot |I| - \alpha \cdot |I|}
=    \frac{2 \alpha - 1}{\alpha}
$$,
Figure~\ref{fig:upper_bound} depicts the two upper bounds on $r$.
Putting the two bounds together, it follows that
\[
r
\leq \min \left\{\frac{\alpha+1}{\alpha} , \frac{2\alpha-1}{\alpha} \right\}
\leq    \frac{3}{2}
\]
,as required.
\end{proof}



\begin{figure}
\centering
\add{fig-2bounds}
\caption[Convex Recoloring - approximation ratio as function of $\alpha$]{
\label{fig:upper_bound}
An upper bound on the approximation ratio as a function of $\alpha$.
The dashed (blue) and the dotted (orange) plots are the two bounds for the general case.
The solid (green) plot is the bound for 2-cr on paths.
}
\end{figure}

\begin{figure}
\centering
\add{fig-greedy-tight}
\caption[Convex Recoloring - tight analysis]{
Greedy might choose to color the path (1, 3, 2), 
then it must recolor one of the vertices \{5, 6\} 
and one of the vertices \{7, 8\}, 
a total of three recolored vertices, 
while an optimal recoloring can color two paths: (5, 1, 6) and (7, 3, 8), 
a total of two recolored vertices.}
\label{fig:tight}
\end{figure}

We show that our analysis is tight even for colored trees, using the
instance depicted in Figure~\ref{fig:tight}, 
which consists of a colored graph where the greedy algorithm might recolor
$\frac{3}{2}$ times more vertices than the optimal convex recoloring.
%
We note that one can simply duplicate the instance using new colors for
each copy in order to construct an arbitrary large graph with the same
approximation ratio.

%%%%%
