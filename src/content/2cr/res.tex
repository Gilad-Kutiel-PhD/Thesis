Section~\ref{sec:hard} contains our hardness result for the weighted
version of \TWOCR{}.  In Section~\ref{sec:path_recoloring} we provide
an alternative definition of \TWOCR{} in terms of maximal independent
set of paths.
%
We first show that we can focus on a specific type of recoloring,
called a \emph{path recoloring},
in which each color induces a path.
%
Then we show that finding a path recoloring can be translated 
into finding an independent set of paths.
%
In Section~\ref{sec:greedy} we present a greedy algorithm for
(unweighted) \TWOCR{} in general graphs that is based on iteratively
adding a shortest path to the current independent set of paths.
%
We provide a tight analysis for the algorithm and show
that its approximation ratio is $\frac{3}{2}$.
%
This is the first time a constant-ratio approximation algorithm is
given for a variant of
\CRP{} in general graphs.
%
We also show that when $G$ is a simple path, 
the same algorithm yields a $\frac{5}{4}$-approximation, 
improving the previous best known approximation ratio by 
Lima and Wakabayashi~\cite{lima2014convex}.
%
In Section~\ref{sec:fpt} we use the above mentioned characterization
of \TWOCR{} to show that a problem kernel of size $4k$ can be obtained
in linear time.
%
This leads to a $O(|E|) + 2^{O(k \log k)}$ time algorithm for \TWOCR{} 
parameterized by the number of color changes $k$.