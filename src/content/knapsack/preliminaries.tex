First, we start with some notations that will enable us to simplify the presentation of the algorithms and their proofs.
We use the notation of $f(B|A)$ to denote the marginal value of $B$ with respect to $A$, {\em i.e.}, $f(A\cup B)-f(A)$.
When it is clear from the context we use $x$, where $x\in U$, to denote the set $\{ x\}$.
For any subset $A\subseteq U$ we denote by $c(A)\triangleq \sum _{x\in A}c(x)$.
Moreover, for an ordered set $A = \{a_1, \dots, a_k\}$ we use $A_i$ to denote its prefix of size $i$, {\em i.e.}, $\{a_1, \dots, a_i\}$ ($A_0$ is set to the empty set).
Since $f$ is monotone and non-negative, we can assume without loss of generality that $f(\emptyset)=0$.

Second, let us present known fast algorithms for \SK.
The greedy algorithm for \SK which uses the ``bang per buck'' greedy rule appears in Algorithm \ref{alg:greedy}.
It receives as parameters the ground set $U$, the submodular objective $f$ (via its value oracle), the cost function $c$, and the budget $\beta$.

\begin{algorithm}[H]
\caption{Greedy$(U, f, c, \beta)$}
\label{alg:greedy}
$S \leftarrow \emptyset$
\\
\While{$U \neq \emptyset$}{
	$e' \leftarrow \displaystyle{\argmax}\left\{\nicefrac[]{f(e|S)}{c(e)}:e\in U\right\}$
	\\
	$U \leftarrow U \setminus \{e'\}$
	\\
	\If{$c(S \cup \{e'\})\leq \beta$}{
		\label{line:dropped}
		$S \leftarrow S \cup \{e'\}$
	}
}
\Return{S}.
\end{algorithm}

If the condition in line \ref{line:dropped} is evaluated to false we say that
the algorithm \emph{dropped} the element $e'$.
It is well known that in general the approximation ratio of Algorithm \ref{alg:greedy}
can be arbitrarily bad.

Khuller {\em et. al.} \cite{khuller1999budgeted} presented the Modified Greedy algorithm, which returns the best between the greedy algorithm (Algorithm \ref{alg:greedy}) and the single best element.
This appears in Algorithm \ref{alg:mgreedy} (its input is identical to that of Algorithm \ref{alg:greedy}).

\begin{algorithm}[H]
\caption{Modified Greedy$(U, f, c, \beta)$}
\label{alg:mgreedy}

$S \leftarrow \text{Greedy}(U, f, c, \beta)$
\\
$T \leftarrow \argmax\left\{S, \argmax \left\{f(e):e\in U, c(e)\leq \beta \right\} \right\}$
\\
\Return{$T$}
\end{algorithm}

Third and last, we present a lemma (can be derived from, {\em e.g.}, \cite{khuller1999budgeted}) that enables one to analyze the performance of the greedy algorithm (Algorithm \ref{alg:greedy}).
\begin{lemma}
\label{lemma:sub-main}
Let $A = \{a_1, \dots, a_k\}$ and $B$ be two subsets such that for all $1 \leq i \leq k$
and for all $e \in B$ it holds that
$\frac{f(a_i|A_{i-1})}{c(a_i)} \geq \frac{f(e|A_{i-1})}{c(e)}$.
Then, $f(A) \geq (1 - e^{-\frac{c(A)}{c(B)}})f(B)$.
\end{lemma}

The proof for the lemma is given in Appendix  \ref{appendix:omitted}. We note that 
the proof only uses standard techniques from previous works.

A particular corollary of the above lemma is that if $A$ is the set of elements chosen by the greedy algorithm in
the $k$ first iterations and $B$ is any subset such that no element from $B$ was dropped by the greedy algorithm in the first $k$ iterations,
then all the condition of Lemma~\ref{lemma:sub-main} are met and it can be applied to lower bound the value of $A$.


%%We consider the Submodular Function Under
%%Knapsack Constraint Maximization problem.
%For a given instance we denote by $O$ an optimal solution and for the rest of the
%paper we assume that the cost function is in the range $[0, b]$.
%
%If it is clear from the context we use $x$ to denote the set $\{x\}$.
%We use the notation $f(B|A)$ to denote $f(A \cup B) - f(A)$, that is the marginal value of $B$
%given $A$.
%For a set of elements, $A$, we use $c(A)$ to denote $\sum_{e \in A}c(e)$.
%For an ordered set $A = \{a_1, \dots, a_k\}$ we use $A_i$ to denote the subset
%$\{a_1, \dots, a_i\}$, $A_0$ denotes the empty set.
%We refer to the cost function in terms of value (e.g. cheap, expensive, most valuable, etc...) and
%to the weight function in terms of size (e.g. small, medium, larger, etc...)

%The greedy algorithm from \cite{khuller1999budgeted} and \cite{krause2005note}
%is described in Algorithm \ref{alg:greedy}.
%
%\begin{algorithm}[H]
%\caption{Greedy$(U, f, c, \beta)$}
%\label{alg:greedy}
%
%initialization: $S \leftarrow \emptyset$
%\\
%\While{$U \neq \emptyset$}{
%	$e' \leftarrow \displaystyle{\argmax_{e \in U}}\frac{f(e|S)}{c(e)}$
%	\\
%	$U \leftarrow U \setminus \{e'\}$
%	\\
%	\If{$c(S \cup \{e'\})\leq \beta$}{
%	\label{line:dropped}
%		$S \leftarrow S \cup \{e'\}$
%	}
%}
%\Return{S}
%\end{algorithm}

%If the condition on line \ref{line:dropped} is evaluated to false we say that
%the algorithm \emph{dropped} the element $e'$.
%It can be shown that in general the approximation ratio of the greedy algorithm
%can be arbitrarily bad.
%
%Let $S$ be the output of the greedy algorithm, the \emph{Modified Greedy} algorithm
%from \cite{khuller1999budgeted} and \cite{krause2005note}
%returns the best among the output of the greedy algorithm and the best singleton,
%that is $\argmax\{S, \displaystyle{\argmax_{e \in U}}f(\{e\})\}$.
%This algorithm is described in Algorithm~\ref{alg:mgreedy}.
%
%\begin{algorithm}[H]
%\caption{Modified Greedy$(U, f, c)$}
%\label{alg:mgreedy}
%
%$S \leftarrow \text{Greedy}(U, f, c)$
%\\
%$T \leftarrow \argmax\{S, \argmax_{e \in U}f(\{e\})\}$
%\\
%\Return{$T$}
%\end{algorithm}

%The following lemma plays an important roll in the rest of this paper.

%\begin{lemma}
%\label{lemma:sub-main}
%Let $X = \{a_1, \dots, a_k\}$ and $Y$ be two subsets such that for all $1 \leq i \leq k$
%and for all $e \in Y$ it holds that
%$\frac{f(a_i|X_{i-1})}{c(a_i)} \geq \frac{f(e|X_{i-1})}{c(e)}$
%then $f(X) \geq (1 - e^{-\frac{c(X)}{c(Y)}})f(Y)$.
%\end{lemma}
%
%\begin{proof}
%Consider $a_i$ and observe that:
%\begin{align}
%	\frac{f(a_i|X_{i-1})}{c(a_i)}c(Y)
%	& = \sum_{e \in Y} \frac{f(a_i|X_{i-1})}{c(a_i)}c(e)
%	\nonumber
%	\\ 	& \geq \sum_{e \in Y} \frac{f(e|X_{i-1})}{c(e)}c(e)
%	\label{ineq:main:cond}
%	\\	& \geq f(Y|X_{i-1})
%	\label{ineq:main:sub}
%	\\ 	& \geq f(Y) - f(X_{i-1})
%	\label{ineq:main:mon}
%\end{align}
%Where inequality \ref{ineq:main:cond} is due to the condition in the lemma, inequality \ref{ineq:main:sub} is due to submodularity of $f$ and inequality \ref{ineq:main:mon} is due to monotonicity of $f$.
%Thus, from the above we get that:
%
%$$
%f(Y) - f(X_i)  \leq (f(Y) - f(X_{i - 1}))
%\left(1 - \frac{c(a_i)}{c(Y)}\right).
%$$
%Hence,
%$$
%f(Y) - f(X_i)  \leq f(Y) \prod_{j = 1}^{i}
%\left(1 - \frac{c(a_j)}{c(Y)}\right).
%$$
%Applying inequality $1 - x \leq e^{-x}$ we get that:
%$$
%f(Y) - f(X_i)  \leq f(Y)
%\left(e^{-\sum_{j = 1}^{i}\frac{c(e_j)}{c(Y)}}\right).
%$$
%Rearranging the terms and setting $i = k$ completes the proof.
%\end{proof}
%
%In particular, if $X$ is the set of elements chosen by the greedy algorithm in
%the $k$ first iterations and $Y$ is any subset such that no element from $Y$ was dropped
%by the greedy algorithm,
%then Lemma~\ref{lemma:sub-main} can be applied on $X$ and $Y$.




