First, we start with some notations that will enable us to simplify the presentation of the algorithms and their proofs.
We use the notation of $f(B|A)$ to denote the marginal value of $B$ with respect to $A$, {\em i.e.}, $f(A\cup B)-f(A)$.
When it is clear from the context we use $x$, where $x\in U$, to denote the set $\{ x\}$.
For any subset $A\subseteq U$ we denote by $c(A)\triangleq \sum _{x\in A}c(x)$.
Moreover, for an ordered set $A = \{a_1, \dots, a_k\}$ we use $A_i$ to denote its prefix of size $i$, {\em i.e.}, $\{a_1, \dots, a_i\}$ ($A_0$ is set to the empty set).
Since $f$ is monotone and non-negative, we can assume without loss of generality that $f(\emptyset)=0$.

Second, let us present known fast algorithms for \SK.
The greedy algorithm for \SK which uses the ``bang per buck'' greedy rule appears in Algorithm \ref{alg:greedy}.
It receives as parameters the ground set $U$, the submodular objective $f$ (via its value oracle), the cost function $c$, and the budget $\beta$.

\begin{algorithm}[H]
\caption{Greedy$(U, f, c, \beta)$}
\label{alg:greedy}
$S \leftarrow \emptyset$
\\
\While{$U \neq \emptyset$}{
	$e' \leftarrow \displaystyle{\argmax}\left\{\nicefrac[]{f(e|S)}{c(e)}:e\in U\right\}$
	\\
	$U \leftarrow U \setminus \{e'\}$
	\\
	\If{$c(S \cup \{e'\})\leq \beta$}{
		\label{line:dropped}
		$S \leftarrow S \cup \{e'\}$
	}
}
\Return{S}.
\end{algorithm}

If the condition in line \ref{line:dropped} is evaluated to false we say that
the algorithm \emph{dropped} the element $e'$.
It is well known that in general the approximation ratio of Algorithm \ref{alg:greedy}
can be arbitrarily bad.

Khuller {\em et. al.} \cite{khuller1999budgeted} presented the Modified Greedy algorithm, which returns the best between the greedy algorithm (Algorithm \ref{alg:greedy}) and the single best element.
This appears in Algorithm \ref{alg:mgreedy} (its input is identical to that of Algorithm \ref{alg:greedy}).

\begin{algorithm}[H]
\caption{Modified Greedy$(U, f, c, \beta)$}
\label{alg:mgreedy}

$S \leftarrow \text{Greedy}(U, f, c, \beta)$
\\
$T \leftarrow \argmax\left\{S, \argmax \left\{f(e):e\in U, c(e)\leq \beta \right\} \right\}$
\\
\Return{$T$}
\end{algorithm}

Third and last, we present a lemma (can be derived from, {\em e.g.}, \cite{khuller1999budgeted}) that enables one to analyze the performance of the greedy algorithm (Algorithm \ref{alg:greedy}).
\begin{lemma}
\label{lemma:sub-main}
Let $A = \{a_1, \dots, a_k\}$ and $B$ be two subsets such that for all $1 \leq i \leq k$
and for all $e \in B$ it holds that
$\frac{f(a_i|A_{i-1})}{c(a_i)} \geq \frac{f(e|A_{i-1})}{c(e)}$.
Then, $f(A) \geq (1 - e^{-\frac{c(A)}{c(B)}})f(B)$.
\end{lemma}

\begin{proof}[Proof of Lemma \ref{lemma:sub-main}]
	Consider $a_i$ and observe that:
	\begin{align}
	\frac{f(a_i|A_{i-1})}{c(a_i)}c(B)
	& = \sum_{e \in B} \frac{f(a_i|A_{i-1})}{c(a_i)}c(e)
	\nonumber
	\\ 	& \geq \sum_{e \in B} \frac{f(e|A_{i-1})}{c(e)}c(e)
	\label{ineq:main:cond}
	\\	& \geq f(B|A_{i-1})
	\label{ineq:main:sub}
	\\ 	& \geq f(B) - f(A_{i-1}).
	\label{ineq:main:mon}
	\end{align}
	Inequality \eqref{ineq:main:cond} follows from the condition in the lemma, inequality \eqref{ineq:main:sub} follows from the submodularity of $f$, and inequality \eqref{ineq:main:mon} follows from the monotonicity of $f$.
	Thus, from the above we can conclude that:
	$$
	f(B) - f(A_i)  \leq (f(B) - f(A_{i - 1}))
	\left(1 - \frac{c(a_i)}{c(B)}\right).
	$$
	Hence,
	$$
	f(B) - f(A_i)  \leq f(B) \prod_{j = 1}^{i}
	\left(1 - \frac{c(a_j)}{c(B)}\right).
	$$
	Applying the inequality $1 - x \leq e^{-x}$ we get that:
	$$
	f(B) - f(A_i)  \leq f(B)\cdot
	e^{-\frac{c(A_i)}{c(B)}}.
	$$
	Rearranging the terms and setting $i = k$ completes the proof.
\end{proof}


A particular corollary of the above lemma is that if $A$ is the set of elements chosen by the greedy algorithm in
the $k$ first iterations and $B$ is any subset such that no element from $B$ was dropped by the greedy algorithm in the first $k$ iterations,
then all the condition of Lemma~\ref{lemma:sub-main} are met and it can be applied to lower bound the value of $A$.


