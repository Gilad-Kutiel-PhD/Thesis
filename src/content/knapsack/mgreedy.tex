We now analyze the approximation ratio of the
Modified Greedy Algorithm~\ref{fig:mgreedy}.
Our analysis matches the approximation ratio that was stated in
\cite{khuller1999budgeted} and \cite{krause2005note} where, we believe, 
there is a logical flaw in the analysis.


Let $O$ be an optimal set, and recall that $T$ is the output of the Modified Greedy algorithm.

The following theorem proves that Algorithm~\ref{alg:mgreedy} is a $1-e^{-1/2}$-approximation.

\begin{theorem}
	$f(T) \geq (1 - e^{-1/2})f(O)$.
\end{theorem}

\begin{proof}
	We can assume that $f(\{e\}) \leq (1 - e^{-1/2})f(O)$ for every $e \in U$, otherwise the value of $e$, and hence of $T$, is large enough and the theorem holds.
	Recall that by the definition of Algorithm~\ref{alg:mgreedy} $S$ is the output of the greedy algorithm, and let $e^*$ be the most valuable element.
	Consider two cases depending on how many elements of $O$ are discarded by the greedy algorithm, i.e., $|O \setminus S$.
	If $|O \setminus S \leq 1$, then 
	$$
	f(O) \leq f(O \cup S) \leq f(O \setminus S) + f(S) \leq f(e^*) + f(S).
	$$
	The above implies that either $f(S) \geq \frac{1}{2}f(O)$ or $f(e) \geq \frac{1}{2}f(O)$ concluding the proof.
	Otherwise, $|O \setminus S| \geq 2$,
	Denote by $a$ and $b$ the first and second elements of $O$ the greedy algorithm drops.
	Denote by $A$ the set of elements chosen by the algorithm just before dropping $a$ and by
	$B$ the set of elements chosen right after dropping $a$ and before dropping $b$.
	If $f(A) \geq (1 - e^{-1/2})f(O)$ then the theorem holds.
	Otherwise, denote by $f(A) = (1 - e^{-(1/2 - \delta)})f(O)$, where $0 < \delta \leq \frac{1}{2}$.

	To prove the theorem we need to infer from the above bounds on both size anf value.
	First, we focus on bounds on the size:

	\begin{align}
		\label{mgreedy:ineq1}
		c(A) \leq (0.5 - \delta)c(O)
		\\
		\label{mgreedy:ineq2}
		c(a) > (0.5 + \delta)c(O)
		\\
		\label{mgreedy:ineq3}
		c(b) \leq c(O \setminus x) \leq (0.5 - \delta)c(O)
	\end{align}
	% 
	Where inequality \ref{mgreedy:ineq1} is due to Lemma~\ref{lemma:sub-main},
	inequality \ref{mgreedy:ineq2} is followed from the fact that $a$ was dropped by the greedy algorithm, inequality \ref{mgreedy:ineq4} is due to inequalities \ref{mgreedy:ineq3} and \ref{mgreedy:ineq2}.
	The above inequalities imply a lower bound  on the size of $B$:
	\begin{equation}
		\label{mgreedy:ineq4}
		c(B) > (2\delta)c(O)
	\end{equation}

	Second, we focus on bounds on the value.
	We need to lower bound of $O$ after $a$ is dropped with respect to $A$, i.e. $f(O \setminus x | A)$:
	\begin{equation}
		\label{ineq:T}
		f(O \setminus a | A) \geq f(O) - f(a) - f(A) \geq (e^{-1/2} - 1 + e^{-(1/2 - \delta)})f(O).
	\end{equation}
	Where the inequality follows from submodularity and monotonicity of $f$.
	Furthermore, we need to lower bound the value of the elements the greedy algorithm chooses after discarding $a$ and up to discarding $b$ with respect to $A$, i.e. 
	\begin{equation}
		\label{ineq:B}
		f(B|A) \geq (1 - e^{-\frac{2\delta}{1/2 - \delta}})f(O \setminus x | A).
	\end{equation}
	Where this inequality is a result of applying Lemma~\ref{lemma:sub-main} on the sets $B$ and $O \setminus x$.

	Finally, we lower bound the value of the modified greedy algorithm:
	\begin{equation}
		f(T) \geq f(A \cup B) = f(A) + f(B | A).
	\end{equation}

	Substituting $f(A) = (1 - e^{-(1/2 - \delta)})f(O)$
	and $f(B|A) \geq (1 - e^{-\frac{2\delta}{1/2 - \delta}})f(O \setminus x | A)$
	into the last inequality gives the desired result as can be seen in Figure~\ref{fig:mgreedy}
\end{proof}

In \cite{khuller1999budgeted} it was claimed that 0.44 is an upper bound on the approximation ratio of this algorithm but no explicit example to support this claim was given and we could not recover such an example.

\begin{figure}
	\caption{
		\label{fig:mgreedy}
		Modified Greedy Approximation Ratio
	}
	\begin{tikzpicture}
		\begin{axis}[
				width=\textwidth
				,domain=0:0.49
				,ymax=.7
				,xmax=.51
				,xlabel=$\delta$
				,xtick distance=0.1
				,ytick distance=0.1
				,axis lines=left
				,grid=both
				,grid style={
						draw=gray!20
					}
				,minor tick num=5
				,legend pos=south west
				,legend entries={
						$f(A)$
						,$f(B|A)$
						,$f(A) + f(B|A)$
						,$1 - e^{-\frac{1}{2}}$
					}
			]
			\addplot[line blue]{1-exp(-(0.5-x))};
			\addplot[line red]{(1 - exp(-(4*x/(1-2*x)))) * (exp(-1/2) - 1 + exp(-(1/2-x)))};
			\addplot[line brown]{1-exp(-(0.5-x)) + (1 - exp(-(4*x/(1-2*x)))) * (exp(-1/2) - 1 + exp(-(1/2-x)))};
			\addplot[line teal]{1-exp(-0.5)};
		\end{axis}
	\end{tikzpicture}
\end{figure}

% \textbf{Upper Bound}
% In \cite{khuller1999budgeted} it is claimed that the upper bound on the approximation ratio
% of the modified greedy algorithm is at most 0.44, but no explicit instance was given to 
% support this claim and we could not reproduce such an example.
% Here we give a simple example that bound the approximation ratio from above by $1-e^{-2/3}$, 
% leaving a gap between the lower and upper bounds.
% Consider an instance of the budgeted maximum coverage problem where the optimal cover consists
% of 3 disjoint subsets, $S_1, S_2, S_3$ each cover exactly third of the ground set. 
% In this example we assume a uniform weight function, and suppose the cost of each subsets of 
% the optimal solution is $1/3$. We now describe the subsets that the modified greedy algorithm
% may choose. Consider $k + 1$ disjoint subsets $T_1, \dots, T_{k + 1}$ where $c(T_i) = 2/3k$, 
% where subset $T_i$ covers $\frac{2}{3k}(1-\frac{2}{3k})^i$ fraction of the ground set and for
% every $i$ it holds that $|T_i \cap S_1| = |T_i \cap S_2| = |T_i \cap S_3|$.
% One can verify that the modified greedy algorithm might chose those sets leaving no budget for
% any of the optimal subsets. As $k$ approaches infinity the total value of those sets approaches
% $1-e^{-2/3}$. 
% 
% 
% 
%   