We analyze the Modified Greedy algorithm (Algorithm~\ref{alg:mgreedy}) of \cite{khuller1999budgeted}.
Our analysis matches the $(1 - e^{-1/2})$-approximation claimed in \cite[]{khuller1999budgeted}, correcting the logical flaw in the analysis.

Let $O$ be an optimal set, and recall that $T$ is the output of the Modified Greedy algorithm.
The following theorem proves that Algorithm~\ref{alg:mgreedy} is a $(1-e^{-1/2})$-approximation.

\begin{theorem}
	\label{theorem:mgreedy}
	The Modified Greedy (Algorithm~\ref{alg:mgreedy}) achieves an approximation of $1 - e^{-1/2}$.
\end{theorem}

\begin{proof}
	We can assume that $f(\{e\}) \leq (1 - e^{-1/2})f(O)$ for every $e \in U$, otherwise the value of $e$, and hence of $T$, is large enough and the theorem holds.
	Recall that by the definition of Algorithm~\ref{alg:mgreedy} $S$ is the output of the greedy algorithm, and let $e^*$ be the most valuable element.
	
	Consider two cases depending on how many elements of $O$ are dropped by the greedy algorithm, i.e., $|O \setminus S|$.
	If $|O \setminus S| \leq 1$, then 
	$$
	f(O) \leq f(O \cup S) \leq f(O \setminus S) + f(S) \leq f(e^*) + f(S).
	$$
	The above implies that either $f(S) \geq \frac{1}{2}f(O)$ or $f(e) \geq \frac{1}{2}f(O)$ concluding the proof for this case.
	Otherwise, $|O \setminus S| \geq 2$,
	Denote by $a$ and $b$ the first and second elements of $O$ the greedy algorithm drops, i.e. $a$ is the considered element at the first time the condition on line~\ref{line:dropped} in Algorithm~\ref{alg:greedy} is evaluated to false and $b$ is the considered element the second time this condition is evaluated to false.
	Denote by $A$ the set of elements chosen by the algorithm just before dropping $a$ and by
	$B$ the set of elements chosen right after dropping $a$ and before dropping $b$.
	If $f(A) \geq (1 - e^{-1/2})f(O)$ then the theorem holds.
	Otherwise, $f(A) = (1 - e^{-(1/2 - \delta)})f(O)$ where $0 < \delta \leq \frac{1}{2}$.

	To prove the theorem we need to infer from the above bounds on both size anf value.
	First, we focus on bounds on the size:

	\begin{align}
		\label{mgreedy:ineq1}
		c(A) \leq (0.5 - \delta)c(O)
		\\
		\label{mgreedy:ineq2}
		c(a) > (0.5 + \delta)c(O)
		\\
		\label{mgreedy:ineq3}
		c(b) \leq c(O \setminus a) \leq (0.5 - \delta)c(O)
	\end{align}
	% 
	Inequality \ref{mgreedy:ineq1} upper bounds the size of $A$ and follows from  Lemma~\ref{lemma:sub-main} since if it does not hold then $f(A) > (1 - e^{-1/2 - \delta})f(O)$.
	Inequality \ref{mgreedy:ineq2} lower bound the size of $a$ and follows from the fact that $a$ was dropped by the greedy algorithm, i.e., $c(A \cup \{a\}) > \beta \geq c(O)$. Inequality \ref{mgreedy:ineq3} is due to inequality \ref{mgreedy:ineq2}, since $c(O \setminus \{a\}) = c(O) - c(a)$ and $\{b\} \subseteq O \setminus \{a\}$.
	The above inequalities imply a lower bound  on the size of $B$:
	\begin{equation}
		\label{mgreedy:ineq4}
		c(B) > (2\delta)c(O)
	\end{equation}

	Second, we focus on bounds on the value.
	We need to lower bound of $O$ after $a$ is dropped with respect to $A$, i.e. $f(O \setminus a | A)$:
	\begin{equation}
		f(O \setminus a | A) \geq f(O) - f(a) - f(A) \geq (e^{-1/2} - 1 + e^{-(1/2 - \delta)})f(O).
	\end{equation}
	Where the inequality follows from submodularity and monotonicity of $f$.
	Furthermore, we need to lower bound the value of the elements the greedy algorithm chooses after discarding $a$ and up to discarding $b$ with respect to $A$, i.e. 
	
	\begin{align}
		\label{ineq:B}
		f(B|A)	& \geq (1 - e^{-\frac{2\delta}{1/2 - \delta}})f(O \setminus a | A)
		\\ 		& \geq (1 - e^{-\frac{2\delta}{1/2 - \delta}})(e^{-1/2} - 1 + e^{-(1/2 - \delta)})f(O).
		\nonumber
	\end{align}
	
	Inequality~\ref{ineq:B} follows from applying Lemma~\ref{lemma:sub-main} on the sets $B$ and $O \setminus a$, and considering the submodular function $f(X|A)$ for every $X \subseteq U$.
	
	Finally, we lower bound the value of the modified greedy algorithm:
	\begin{equation}
		f(T) \geq f(A \cup B) = f(A) + f(B | A).
	\end{equation}
	
	Substituting $f(A)$ with $(1 - e^{-(1/2 - \delta)})f(O)$ and lower bounding $f(B|A)$ using~\ref{ineq:B} results in the following:
	\begin{equation}
		\label{ineq:T}
		f(T) \geq (1 - e^{-(1/2 - \delta)})f(O) + (1 - e^{-\frac{2\delta}{1/2 - \delta}})(e^{-1/2} - 1 + e^{-(1/2 - \delta)})f(O)
	\end{equation}
	One can verify that the expression in~\ref{ineq:T} is, indeed, at least $(1-e^{-1/2})f(O)$ for $0 < \delta \leq \frac{1}{2}$.
\end{proof}

In \cite{khuller1999budgeted} it was claimed that 0.44 is an upper bound on the approximation ratio of this algorithm but no explicit example to support this claim was given and we could not recover such an example.

% \textbf{Upper Bound}
% In \cite{khuller1999budgeted} it is claimed that the upper bound on the approximation ratio
% of the modified greedy algorithm is at most 0.44, but no explicit instance was given to 
% support this claim and we could not reproduce such an example.
% Here we give a simple example that bound the approximation ratio from above by $1-e^{-2/3}$, 
% leaving a gap between the lower and upper bounds.
% Consider an instance of the budgeted maximum coverage problem where the optimal cover consists
% of 3 disjoint subsets, $S_1, S_2, S_3$ each cover exactly third of the ground set. 
% In this example we assume a uniform weight function, and suppose the cost of each subsets of 
% the optimal solution is $1/3$. We now describe the subsets that the modified greedy algorithm
% may choose. Consider $k + 1$ disjoint subsets $T_1, \dots, T_{k + 1}$ where $c(T_i) = 2/3k$, 
% where subset $T_i$ covers $\frac{2}{3k}(1-\frac{2}{3k})^i$ fraction of the ground set and for
% every $i$ it holds that $|T_i \cap S_1| = |T_i \cap S_2| = |T_i \cap S_3|$.
% One can verify that the modified greedy algorithm might chose those sets leaving no budget for
% any of the optimal subsets. As $k$ approaches infinity the total value of those sets approaches
% $1-e^{-2/3}$. 
% 
% 
% 
%   