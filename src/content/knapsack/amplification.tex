\def\pLarge{P_{\text{large}}}
\def\pVal{P_{\text{val}}}
\def\MGreedy{Modified$^2$Greedy}
\def\BOTAlg{BestOfThree}
\def\mA{\mathcal{A}}

In this section we show a simple algorithm, which given
an $r$-approximation $\mA$ for \SK, $r<1/2$ can 
be used to derive an $r'$-approximation, $r<r' <1/2$, for 
\SK using  a constant number of calls
for $\mA$ and $\frac{3n^2}{n}$ oracle queries. 

Algorithm \ref{algorithm:amplify}, {\sc Amplify}
({\bf solve the font problem}),  receives 
an approximation algorithm $\mA$ for $\SK$, 
a universe $U$, a monotone, non-negative
submodular function $f:2^U \rightarrow \mathbb{R}$,
a non-negative cost function $c:U \rightarrow \mathbb{R}$ 
for the elements, and the budget $\beta$. 
The procedure also gets two scalar parameters $\epsilon$ 
and $\rho$.  These two parameters control the approximation
ratio and complexity of the algorithm, and should be set according
to the claims in this section to obtain the required approximation 
ratio and complexity. 

Broadly speaking,  the algorithm finds the pair of  elements with maximal value $\pVal$,  and then divides all pairs of elements $\{a,b\} \subseteq$ for which 
$\frac{f(\{a,b\})}{f(\pVal)} \geq \rho$ into {\em buckets}, where the subsets 
in each bucket have the same profit to a factor of $(1+\epsilon)$. 
The algorithm then extends the set of smallest cost in  each bucket 
to a solution using the algorithm $\mA$.  The algorithm returns the 
best between the solutions found using the buckets and $\mA$, the pair of
elements with maximal profit, the single element with maximal profit, and the result of the greedy algorithm 
over the input instance.

\begin{algorithm}
	\caption{Amplify($\mA, U, f, c, \beta, \epsilon, \rho$)}
	\label{algorithm:amplify}
	% Initialization
	\tcp{Initialization}
	$\pVal \leftarrow \argmax_{ \{a,b\} \subseteq U, c(\{a,b\})\leq \beta } f(\{a,b\})$
	\\ $w_{\max}\leftarrow f(\pVal)$
	\\ $i_{\max} \leftarrow \floor{\log_{1 + \epsilon} \frac{1}{\rho}}$
	\\
	\tcp{Buckets}
	\For{$i \in \{0,\dots,i_{\max}\}$}{
		$B_i = \{ \{a,b\}\subseteq U| c(\{a,b\})\leq \beta, \rho(1+\epsilon)^i \leq \frac{f(\{a,b\})}{w_{\max}}  < \rho (1+\epsilon)^{i+1} \}$
		\\
		$P_i \leftarrow \argmax_{\{a,b\}\in B_i} c(\{a,b\})$
%	}
%	\tcp{Candidate Solutions}
%	\For{$i \in \{0,\dots,i_{\max}\}$}{
\\
		$S_i \leftarrow P_i \cup \mA(U , f_{P_i}, c, \beta - c(P_i))$
	}
	$G \leftarrow \text{Greedy}(U, f, c, \beta))$ \label{amplify:greedy}
	\\
	$T \leftarrow \argmax_{\{a\}\subseteq U| c(a)\leq \beta} f(\{a\})$ \label{amplify:singletons}
	\\
	\Return $\argmax_{S \in \{S_0, S_1, \ldots, S_{i_{\max}}, G, \pVal,T\}} f(S)$
	% 

\end{algorithm}

The following function, which already appeared on Theorem \ref{thrm:Amplification}, will be useful throughout the analysis of the algorithm:

$$ B(\alpha)=1-e^{-\frac{\alpha}{2\alpha - 1}}$$
$$ A(\alpha) = \frac{1}{1-e^{-\alpha}}-\frac{1}{B(\alpha)}$$
 $$D(\alpha)=(1-e^{-\alpha})/B(\alpha)$$
 
 \begin{lemma}
 	\label{lemma:amplification}
 	Given an $r$-approximation $\mA$ for $\SK$ where $r<\frac{1}{2}$.
 	For any $\frac{2}{3}\leq \alpha \leq  \ln 2$, input $U,f,c,\beta$ of $\SK$, 
 	and $\epsilon$ such that $0<\epsilon< \frac{1-2r}{r}$, then executing 
 	Algorithm \ref{algorithm:amplify}  with parameters 
 	$(\mA, U, f, c, \beta, \epsilon, A(\alpha))$ returns
 	$$min\left(1-e^{-\alpha}, D(\alpha) + (1-r) \left( \frac{2+\epsilon}{1+\epsilon} D(\alpha) -1 \right) \right)$$
 	approximation for the input instance. 
 \end{lemma}



\begin{proof}

	Fix an optimal solution $O$.
	 If $|O|\leq 2$ the algorithms returns an optimal solution and the Lemma hold.
	Therefore we can assume $|O| \geq 3$.  Denote by $\pLarge$ the set of  two largest elements in $O$.
	
	Let $L\beta$ be the size of the largest element in $O$.  If $L\leq 1-\alpha$ then
	by Lemma \ref{lemma:sub-main} ({\bf replace with better explanation when we have a better lemma}), $f(G) \geq (1-e^{-\alpha}) f(O)$  ($G$ is the result of 
	greedy  in line \ref{amplify:greedy} of the algorithm). Therefore, the algorithm
	returns a solution of value least $(1-e^{-\alpha})f(O)$ and the Lemma holds 
	in case that $L\leq 1-\alpha$. Therefore we can assume that $L> 1-\alpha$.
	
	If $O\setminus \pLarge \subseteq G$, then
	$f(G)+ f(\pVal)\geq f(O\setminus \pLarge) + f(\pLarge) \geq f(O)$.
	Therefore $f(G)\geq \frac{f(O)}{2}$ or $f(\pVal)\geq \frac{f(O)}{2}$ 
	and we get than the Lemma holds (note that $\alpha \leq ln 2$ and therefore
	$1-e^{-\alpha} < \frac{1}{2}$). Thus we can assume that 
	$O\setminus \pLarge \nsubseteq G$.
	
	The following corollary states that if $f(\pLarge)$ is fairly small
	with respect to $f(O)$ than the solution $G$ from the greedy algorithm provides
	the required approximation ratio. {\bf maybe we should change the corollary to match the description}
	
	\begin{corollary}
		\label{corollary:cor1}
	If $f(O|\pLarge) \geq D(\alpha) f(O)$ then $f(G)\geq (1-e^{-\alpha})f(O)$.
	\end{corollary}

\begin{proof}
	
	As the size of the larget element in $O$ is $\beta L$, the size of the third 
	largest 
	element in $O$ is at most $\beta M(L)$ when $M(L)= \min \{\frac{1-L}{2}, L\}$. Since 
	$O\setminus \pLarge \nsubseteq G$, there most be an element from $O \setminus \pLarge$ the greedy in line \ref{amplify:greedy} drops. Since all the elements 
	in $O\setminus \pLarge$ are of size at most $\beta M(L)$ the knapsack most 
	already have elements of capacity $\beta (1-M(L))$ at the first time an element from $O\setminus \pLarge$ is dropped. Also, $c(O\setminus \pLarge) \leq \beta(1-L -M(L))$, therefore by Lemma \ref{lemma:sub-main} we get that 
  	$$f(G)\geq \left(1-e^{- \frac{\beta (1-M(L)) }{c(O\setminus \pLarge)}}\right) f(O\setminus \pLarge)
  			\geq \left(1-e^{- \frac{1-M(L) }{1-L-M(L)}}\right) f(O\setminus \pLarge) $$
  			
  	The term  $1-e^{- \frac{1-M(L) }{1-L-M(L)}}$ is increasing as a function of $L$
  	({\bf this is not trivial, I simply drew the graph on Desmos for that}), therefore
  		as $L> (1-\alpha)$ we get 
  		$$f(G)\geq  \left(1-e^{- \frac{1-M(L) }{1-L-M(L)}}\right) f(O\setminus \pLarge) 
  		\geq \ \left(1-e^{- \frac{1-M(1-\alpha) }{1-(1-\alpha)-M(1-\alpha)}}\right) f(O\setminus \pLarge) $$
  		
  	As $\frac{2}{3} \leq \alpha \leq \ln 2$ we have $M(1-\alpha) = 1-\alpha$. Combining this with the previous inequality we get 
  	  		$$f(G)
  	\geq \ \left(1-e^{- \frac{1-M(1-\alpha) }{1-(1-\alpha)-M(1-\alpha)}}\right) f(O\setminus \pLarge) \geq \left( 1- e ^ {- \frac{\alpha}{ 2\alpha -1}}\right)f(O\setminus \pLarge) 
  	= B(\alpha)f(O\setminus \pLarge) $$
  	
  	Now, recall that $D(\alpha)= \frac{1-e^{-\alpha}}{B(\alpha)}$ 
	$f(O|\pLarge) \geq D(\alpha) f(O)$ by the condition of the 
	lemma and $f(O|\pLarge) \leq f(O\setminus \pLarge)$ as
	$f$ is non-negative. Using these observations and the last lower 
	bound on $f(G)$ we get
	\begin{equation*}
\begin{array}{rcl}
f(G) &\geq& B(\alpha)f(O\setminus \pLarge)  \geq
B(\alpha) f(O|\pLarge)\\ & \geq& 
B(\alpha ) D(\alpha ) f(O)
= B(\alpha)\frac{1-e^{-\alpha}}{B(\alpha) }f(O)= (1-e^{-\alpha})f(O)
\end{array}
	\end{equation*}

	{\bf solve the alignment}.
	\end{proof}

	\begin{corollary}
		\label{corollary:cor2}
		If $f(G)< (1-e^{-\alpha})f(O)$ and $f(\pVal) < (1-e^{-\alpha})f(O)$ then
		$f(\pLarge) \geq A(\alpha) f(\pVal)$ 
		\end{corollary}
	\begin{proof}
		As $f(G)< (1-e^{-\alpha})$, the condition of corollary \ref{corollary:cor1} does not hold. Therefore
		$f(O|\pLarge)< D(\alpha) f(O)$. Hence, 
		$$f(O)= f(\pLarge) + f(O|\pLarge) < f(\pLarge) + D(\alpha) f(0)$$ 
		By changing side ({\bf is this the right term?}) and using $f(\pVal) < (1-e^{-1})f(O)$ 
		we get 
		$$f(\pLarge) > (1-D(\alpha)) f(O) > \frac{1-D(\alpha)}{1-e^{-\alpha}} f(\pVal) 
		= \left(\frac{1}{1-e^{-\alpha}} - \frac{D(\alpha)}{1-e^{-\alpha}}\right) f(\pVal) = A(\alpha) f(\pVal)$$
		The second inequality used the observation that $D(\alpha) < 1$ for $\frac{2}{3}\leq \alpha \leq \ln 2$ ({\bf again, numerically verified}). The last equality follows from the definitions
		of $D(\alpha)$ and $A(\alpha)$. 
		
			
		\end{proof}

	\begin{corollary}
		\label{corollary:cor3}
			If $f(G)< (1-e^{-\alpha})f(O)$ and $f(\pVal) < (1-e^{-\alpha})f(O)$ then
			there is $i\in \{0, \ldots, i_{\max}\}$ such that 
		$$\frac{f(S_i)}{f(O)} \geq D(\alpha) + (1-r)\left( \frac{2+\epsilon}{1+\epsilon} D(\alpha) -1 \right)$$
	\end{corollary}
\begin{proof}
		As the conditions of corollary \ref{corollary:cor2} holds, we have
		$f(\pLarge) \geq A(\alpha) f(\pVal)$.
		 Let $i$ by the minimal integer $i$ such
		that $\frac{f(\pLarge)}{f(\pVal)}\geq A(\alpha)(1+\epsilon) ^i$, as $f(\pLarge) \geq A(\alpha) f(\pVal)$ we get $i\geq 0$. 
		Recall that $i_{\max}= \floor{\log_{1+\epsilon} \frac{1}{\rho} } = 
		\floor{\log_{1+\epsilon} \frac{1}{A(\alpha)}} $ (as the algorithm is used 
			with $\rho=A(\alpha)$). 
			Therefore
			$A(\alpha) (1+\epsilon) ^{i_{\max} +1} < A(\alpha) \frac{1}{A(Alpha) } = 1 \leq \frac{f(\pLarge)}{f(\pVal)}$, and thus $i\leq i_{\max}$. 
			Also $c(\pLarge) \leq c(O) \leq \beta$, and we can conclude that 
			$\pLarge \in B_i$ (note that $w_{\max} = f(\pVal)$).
			
			From the definition of $P_i$ we get $c(P_i)\leq c(pLarge)$ and 
			$f(P_i) \geq \frac{1}{1+\epsilon} f(\pLarge)$. 
			Let $Q_i  =  \mA(U, f_{P_i}, c, \beta - c(P_i))$. 
			As $c(P_i) \leq c(\pLarge)$ we get that $O\setminus \pLarge$ 
			is a feasible solution for the problem instance given to $\mA$. 
			Hence, as $\mA$ is a $r$-approximation we get
			$$f(Q_i | P_i) \geq r f(O\setminus \pLarge| P_i)$$. 
			Therefore,
			\begin{align*}
			f(S_{i}) 
			&
			\geq f(P_i) + f(Q_i|P_i)
			\\ & 
			\geq f(P_i) + r f(O \setminus \pLarge | P_i)
			\\ & 
			\geq f(P_i) + r(f(O) - f(\pLarge) - f(P_i))
			\\ & 
			= (1-r)(P_i) + r(f(O) - f(\pLarge))
			\\ & 
			\geq \frac{1-r}{1 + \epsilon}f(\pLarge) + r(f(O) - f(\pLarge))
			\\ &
			= \left(
			\frac{(2 + \epsilon)}{(1 + \epsilon)}(1-r) - 1
			\right)
			f(\pLarge)
			+ rf(O)
			\label{eq:sub:bucket}
			\end{align*}
			
			As $\epsilon\leq \frac{1-2r}{r}$ one can deduce that 
			$\left(
			\frac{(2 + \epsilon)}{(1 + \epsilon)}(1-r) - 1
			\right) \geq 0$.
			Also, since $f(G)< (1-e^{-\alpha})f(O)$ by corollary \ref{corollary:cor1} we 
			get $f(O|\pLarge)  <D(\alpha)f(O)$, by using $f(O|\pLarge)= f(O) -f(\pLarge)$
			and rearranging  the terms we get 
			$f(\pLarge)> (1-D(\alpha)) f(O)$.
			Therefore,
			$$f(S_i) \geq 
			\left(
			\frac{(2 + \epsilon)}{(1 + \epsilon)}(1-r) - 1
			\right)
			f(\pLarge)
			+ rf(O) \geq 
				\left(
			\frac{(2 + \epsilon)}{(1 + \epsilon)}(1-r) - 1
			\right) (1-D(\alpha ) )f(O) +rf(O)$$
			and the corollary immediately follows. 
			
		
	\end{proof}
The Lemma follows immediately from corollary \ref{corollary:cor3} as it states 
that either $G$, $\pVal$ or $S_i$ for some $i$ would provide the required approximation 
ratio. 
\end{proof}

\begin{lemma}
	\label{lemma:amp_runtime}
Algorithm \ref{algorithm:amplify} uses $\floor{\log_{1+\epsilon} \frac{1}{\rho}}+1$ 
	invocation to $\mA$ and up to $\frac{3}{2} n ^2+n$ additional oracle queries.
 \end{lemma}
\begin{proof}
	The number of invocation to $\mA$ immediately follows from the algorithm. 
	Beside the invocation to $\mA$ the algorithm runs the greedy procedure 
	which uses up to $n^2$ queries. The queries for the initialization and
	buckets phases only considers sets of size $2$, and therefore can be 
	implemented by up to $n(n-1)/2\leq n^2/2$ queries. The execution 
	of line \ref{amplify:singletons} would take another $n$ queries.
	\end{proof}



\begin{proof}[Proof of theorem \ref{thrm:Amplification}]
	
	We obtain the described approximation ratio when running the algorithm
	with $\epsilon = \epsilon^*$ and $\rho = A(\alpha^*)$. 
	By the conditions of the theorem
	$$\epsilon^* = 2 ^{-\frac{\log_2 (A(\alpha^*))}{k-1}} -1 \leq \
	2 ^{-\frac{ \log_2(A(\alpha^*))}{  \left(\frac{1}{3\log_2(\nicefrac[]{1}{r}-1)} \right) }} -1  = 2 ^{- 3 \log_2(A(\alpha^*))\log_2(\nicefrac[]{1}{r}-1) } -1$$
	Now, using the observation that $ -3 \log_2(A(\alpha^*)\leq 1$ we get
	({\bf again, numerically verified for every alpha from 2/3 to ln 2})
	$$\epsilon^* \leq 
	2^{\log_2(\nicefrac[]{1}{r}-1)} -1 = \nicefrac[]{1}{r}-2 = \frac{1-2r}{r} $$
	therefore the conditions of Lemma \ref{lemma:amplification} apply, 
	and the approximation ratio follows. 
	
	By lemma \ref{lemma:amp_runtime} the number of invocations for $\mA$ 
	is 
	$$\floor{\log_{1+\epsilon^*} \frac{1}{A(\alpha^*)}}+1 \leq \
	\log_{1+\epsilon^*} \frac{1}{A(\alpha^*)} +1 =
	\frac{\log \frac{1}{A(\alpha^*)}} {\log(1+\epsilon^*)} +1 
	= \frac{\log \frac{1}{A(\alpha^*)}} {\log(2 ^{-\frac{\log_2 (A(\alpha^*))}{k-1}} )} +1 
	= \frac{\log \frac{1}{A(\alpha^*)}} {-\frac{\log_2 (A(\alpha^*))}{k-1}} +1
	= k$$
	and the number of addition oracle queries is $\nicefrac[]{3n^2}{2} + n $ 
	as required.
	
\end{proof}

