The modified greedy algorithm returns the best solution between the greedy solution and the best singleton.
Since scanning all pairs of elements rather than just singletons does not affect the running time of the modified greedy algorithm a natural question to ask is whether considering pairs of elements result in strictly better algorithm.
Here we do not give a complete answer to this question but we are able to improve our analysis for this algorithm. 

The \emph{Modified$^2$ Greedy} algorithm~\ref{alg:mmgreedy} 
is similar to the  modified greedy algorithm, 
except it considers the best pair of element rather than the best singleton element.


\begin{algorithm}
	$S \leftarrow \text{greedy}(U, f, c, \beta)$
	\\
	$T \leftarrow \arg\max\{S, \arg\max_{e_1, e_2 \in U}f(\{e_1, e_2\})\}$
	\\
	\Return{T}
	% 
	\caption{Modified$^2$ Greedy$(U, f, c, \beta)$}
	\label{alg:mmgreedy}
\end{algorithm}

We now analyze its performance.

\begin{theorem}
	\label{theorem:mmgreedy}
	The Modified$^2$ Greedy (Algorithm~\ref{alg:mmgreedy}) achieves an approximation of $(1-e^{-(1/2 + \epsilon)})$ where $0 < e \leq 1/6$ is an absolute constant.
\end{theorem}

\def\eps{0.104}
\begin{proof}
Let $\epsilon > 0$ an absolute constant to be determined later.
If there is a pair of elements $e_1, e_2 \in U$ whose value is high enough, i.e., $f(\{e_1, e_2\}) \geq (1 - e^{-1/2 + \epsilon})f(O)$, the theorem follows.
Thus, let us assume that $f(\{e_1, e_2\}) < (1 - e^{-1/2 + \epsilon})f(O)$ for every $e_1, e_2 \in U$.
Similarly to the proof of Theorem~\ref{theorem:mgreedy}, we can assume that the greedy algorithm discards at least three elements from $O$, i.e., $|S \setminus O| \geq 3$.
Otherwise, we are guaranteed that $f(T) \geq 0.5f(O)$.
% 
Let $a$ be the first element in $O$ that was dropped by the algorithm, i.e. the first time the condition on line~\ref{line:dropped} was false,
and let $A$ be the set of elements chosen by the algorithm just before dropping $a$.
% 
If $f(A) \geq (1 - e^{-(1/2 + \epsilon)})f(O)$ the theorem holds.
Otherwise denote $f(A) = (1 - e^{-(1/2 + \epsilon - \delta)})f(O)$ where $0 < \delta \leq 1/2 + \epsilon$.
% 
Lemma~\ref{lemma:sub-main} implies that: 
\begin{equation}
	\label{mmgreedy:ineq1}
	c(A) \leq (0.5 + \epsilon - \delta)c(O),
\end{equation}
otherwise $f(A) > (1-e^{-(1/2 + \epsilon - \delta)})f(O)$.
Moreover, since $a$ was dropped we confirm that:
\begin{equation}
	\label{mmgreedy:ineq2}
	c(a) > (0.5 -\epsilon + \delta)c(O),
\end{equation}
since $c(A \cup \{a\}) > \beta \geq c(O)$.

We say that an element, $e$, is \emph{big} if $c(e) \geq (1/4 + \epsilon/2 - \delta/2)c(O)$, 
otherwise it is \emph{small}.
Note that $a$ is big since $\epsilon \leq 1/6$ and $\delta > 0$.

Note that $O\setminus\{a\}$ contains at most one big element (if that is not the case then the size of two big elements is at least $(1/2 + \epsilon - \delta)c(O)$), in contradiction to \ref{mmgreedy:ineq2}.
Thus, since the greedy algorithm dropped at least three elements from $O$, it must be the case that at least one of them is small.
Let $b$ be the first such element, and let $B$ be the set of elements chosen by the 
algorithm right after dropping $a$ and just before dropping $b$.
Also denote by $C$ the subset of small elements in $O$, 
i.e. $C = \{e \in O : c(e) < (1/4 + \epsilon/2 - \delta/2)c(O)\}$.
We note that:

\begin{equation}
\label{mmgreedy:ineq3}
c(C) \leq (0.5 + \epsilon - \delta)c(O),
\end{equation}
since $a$ is big.
% 
The above inequalities derive a lower bound on the size of $B$:
\begin{equation}
	\label{mmgreedy:lower-bound-cB}
	c(B) \geq (1/4 - 3\epsilon/2 + 3\delta/2)c(O)
\end{equation}

Now we bound value.
First, we wish to lower bound $f(C)$.
Since $O$ contains at most two big elements, and the fact that the value of any pair of elements cannot exeed $(1 - e^{-(1/2 + \epsilon)})f(O)$, we can conclude that $f(C) \geq e^{-(1/2 + \epsilon)}$ (follows from the submodularity of $f$).

To lower bound the marginal value of $B$ with respect to $A$ we apply Lemma~\ref{lemma:sub-main} on $B$ and $C$ and choose the submodular function to be $f(X|A)$ for every $X \subseteq U$.
This gives us the following:
\begin{equation}
	\label{mmgreedy:lower-bound-B-given-A}
	f(B|A) \geq 
	(1-e^{-\frac{1-6\epsilon+6\delta}{2+4\epsilon-4\delta}})
	\left[
	e^{-(1/2 + \epsilon)}
	- (1 - e^{-(1/2 + \epsilon - \delta)})
	\right]f(O)
\end{equation}

Finally, we would like to lower bound $f(T)$.
Note that if $f(A) \geq (1 - e^{-(1/2 + \epsilon)})f(O)$ we can lower bound $f(T)$ by $(1 - e^{-(1/2 + \epsilon)})f(O)$ and otherwise
\begin{multline}
	f(A \cup B)  
	= 
	f(A) + f(B|A) 
	\geq 
	\\
	(1-e^{-(1/2 + \epsilon - \delta)})f(O) 
	\\
	+ 
	(1-e^{-\frac{1-6\epsilon+6\delta}{2+4\epsilon-4\delta}})
	\left[
	e^{-(1/2 + \epsilon)}
	- (1 - e^{-(1/2 + \epsilon - \delta)})
	\right]f(O)
\end{multline} 

Thus, for a given fixed $\\min epsilon > 0$ we have
\begin{equation}
	f(T) \geq \min \begin{cases}
		(1 - e^{-(1/2 + \epsilon)})
		\\
			\displaystyle{\min_{\delta}}
			(1-e^{-(1/2 + \epsilon - \delta)})f(O) 
			+ 
			\\ 
			(1-e^{-\frac{1-6\epsilon+6\delta}{2+4\epsilon-4\delta}})
			\left[
			e^{-(1/2 + \epsilon)}
			- (1 - e^{-(1/2 + \epsilon - \delta)})
			\right]f(O)
	\end{cases}
\end{equation}
for $0 < \delta \leq 1/2 + \epsilon$.
Optimizing over $0 < \epsilon \leq 1/6$ we get that $\epsilon = \text{TODO}$ and $\delta = \text{TODo}$.
\end{proof}

\paragraph{Remark:} From the proof of Theorem~\ref{theorem:mmgreedy} we obtain an approximation of $(1 - e^{-1/2 + \eps})$.

\textbf{Upper Bound}
We show that the approximation ratio of the modified$^2$ greedy algorithm is at most $0.5$.
To see this consider the following instance of the budgeted maximum coverage problem:
Let the elements be $X = \{x_1, \dots, x_{2n}\}$, 
and a collection of subsets over $a$, $\mathcal{S} = \{S_1, \dots, S_{n + 1}\} \cup \{S\}$,
where for each $1 \leq i \leq n + 1$, $S_i = \{x_i\}$, and $c(S_i) = 1$. 
Also, set $S = \{x_{n + 2}, \dots, x_{2n}\}$, and $c(S) = n$.
Finally, for each $x \in X$ set $w(x) = 1$ and set the budget for this instance to be $2n$.
One can verify that the modified modified algorithm will return a solution of value $n + 1$
While taking $S$ along with any other $n$ subsets yields a solution of value $2n - 1$.  

Note that this upper bound is also applicable for a modified$^k$ greedy algorithm for any $k$, i.e. the algorithm that returns the best among the greedy solution and the best $k$-tuple of elements.



