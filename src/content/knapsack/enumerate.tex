In \cite{khuller1999budgeted} it was shown that by enumerating all the triplets over the 
ground set and greedily adding elements to those seed sets one obtains a 
$1-e^{-1}$-approximation algorithm.
This result was also generalized to a general submodular functions in \cite{sviridenko2004note}.
The running time of this algorithm is $O(n^5)$.
Faster, $1 - \frac{1}{e} - \epsilon$-approximation algorithms
are known \cite{Alina2017, badanidiyuru2014fast}. 
The running time of these algorithms depend on $\epsilon$, 
and, as the author mention, are impractical. 
For example, the running time of the faster of the above algorithms for $\epsilon = 2^{-2}$ is
$2^{2O(2^{8})}n\log^2n$ achieving approximation ratio of $\approx 0.38$.
Here we describe a $0.48$-approximation algorithm with the same 
asymptotic running time as the simple greedy algorithm and with reasonable constants.


Before describing the algorithm we further analyze the performance of the greedy algorithm.
Fix an optimal solution, $O$, and consider its two largest elements, $P = \{a, b\}$.
Denote $x = f(P)$ so $f(O \setminus P | P) = 1 - x$.
Now consider the third largest element in $O$, $c$, and denote $|c| = 1/3 - \delta$,
and let $A$ be the output of the greedy algorithm, then we have the following theorem

\begin{theorem}
If the largest element in $O$ is bigger than $1/3$ then 
\begin{align*}
f(A) 	& \geq (1-e^{-\frac{2/3 + \delta}{1 - 1/3 - (1/3 - \delta)}})(1 - x)
\\		& \geq (1 - e^{-1.5})(1 - x) 
\end{align*}
\end{theorem} 

\begin{proof}
The proof follows immediately from Lemma~\ref{lemma:sub-main}.
\end{proof}

As a corollary we have that:

\begin{corollary}
If $x \leq \frac{1 - e^-{2/3}}{2 - e^{-1.5} - e^{-2/3}} \approx 0.39$ 
then $f(A) \geq (1 - e^{-2/3})f(O)$.
\end{corollary}

We now present a $(1-e^{-2/3})$-approximation algorithm that runs in $O(n^2)$ time.
This algorithm carefully selects a constant number of pairs of elements and greedily 
add elements to them.

We assume that the size of the knapsack is 1.
Consider all pairs of elements, we describe now how to choose the relevant pairs.

Let $\omega$ be the value of the most valuable pair and $\epsilon$ be a constant to 
be determined later.
We divide the pairs into buckets where the $i$th bucket contains all the pairs with value
between $0.38 (1 + \epsilon)^{i - 1}$ and $0.38 (1 + \epsilon)^{i}$.
We only consider the smallest pair from each bucket.  

Given such a pair, $P$, we construct a new Knapsack problem of budget $1 - c(P)$ and with 
respect to the function $f_P$.
We use the modified modified algorithm to obtain a $1-e^{-(1/2 + \eps)}$-approximation
solution to this problem and combine it with the pair to obtain a candidate 
solution to the original problem.

The algorithm returns the best solution from the following candidates:
\begin{enumerate}
  \item Best solution from the above candidates
  \item The pair with  largest value among all pairs
  \item The solution of the simple greedy algorithm (starting from an empty set)
\end{enumerate}       

\begin{theorem}
The above algorithm is $(1 - e^{-2/3})-approximation$ algorithm.
\end{theorem}

\begin{proof}
Fix an optimal solution, $O$, and consider two cases:
\begin{enumerate}
  \item $O$ has at least one element larger than $1/3$
  \item All elements in $O$ are smaller or equal to $1/3$
\end{enumerate}
In the later case the proof follows immediately from Lemma~\ref{lemma:sub-main}.
Now consider the first case and denote by $P^*$ the two largest elements in $O$.
Let $P$ be the pair considered by the algorithm from the same bucket of $P^*$ then we know
that:
\begin{enumerate}
  \item $c(P) \leq c(P^*)$
  \item $w(P) \geq (1 + \epsilon)^{-2}w(P^*)$
\end{enumerate}
Set $\omega = f(P^*)$ then $f(O\setminus P^* | P^*) = 1 - \omega$.
Let $S_P$ the solution produced by the bucket algorithm from $P$, 
then using Lemma~\ref{lemma:sub-main} and using the properties of $P$ we know that
\begin{equation}
\label{eq:sub:bucket}
f(S_P) \geq (1 + \epsilon)^{-2}\omega + (1-e^{-(1/2 + \eps)})(1 - \omega)
\end{equation}
On the other hand, denote the size of the third largest element in $O$ by $1/3 - \delta$
(for $0 \leq \delta \leq 1/3$) and  let $S$ be the solution of the greedy algorithm, then 
\begin{align}
\label{eq:sub:greedy}
f(S) & \geq (1 - e^{-\frac{2/3 + \delta}{1 - 1/3 - (1/3 - \delta)}})(1 - 2w) 
\\
& \geq (1 - e^{-1.5})(1-w)
\end{align}
Setting $\epsilon$ to (say) $0.002$ and taking the maximum between \ref{eq:sub:bucket} and
\ref{eq:sub:greedy} completes the proof as can be seen in Figure~\ref{fig:sub:bucket}.
\end{proof}

\begin{figure}
\def\zeta{0.002}
\caption{
\label{fig:sub:bucket}
Bucket Algorithm - Approximation Ratio
}
\begin{tikzpicture}
\begin{axis}[
	width=\textwidth
	,domain=0:1
	,ymax=.55
	,ymin=0.3
% 	,xmax=.51
	,xlabel=$\omega$
	,xtick distance=0.1
	,ytick distance=0.1
	,axis lines=left
	,grid=both
	,grid style={
		draw=gray!20
	}
	,minor tick num=5
	,legend pos=south west
	,legend entries={
		Bucket Algorithm
		,Greedy Algorithm
		,$(1 - e^{-2/3})$
	}
]
  \addplot[blue, dashed]{(1 + \zeta)^(-2)*x + (1 - e^(-(1/2 + \eps))) * (1 - 2 * x)};
  \addplot[red, dotted]{(1 - exp(-1.5)) * (1 - x)};
  \addplot[green]{(1 - exp(-2/3))};
\end{axis}
\end{tikzpicture}
\end{figure}
