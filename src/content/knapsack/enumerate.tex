\def\pLarge{P_{\text{large}}}
\def\pValuable{P_{\text{valuable}}}

We now present a $(1-e^{-2/3})$-approximation algorithm that runs in $O(n^2)$ time.
The main procedure~\ref{procedure:bucket} of this algorithm  groups pairs of elements with similar value into (a constant number) buckets.
Then, from each bucket, it takes the smallest pair (i.e. the pair that minimizes $c$) and, using the Modified$^2$ Greedy algorithm, extend into a candidate solution.
Finally it returns the best solution among the above candidates.

\begin{procedure}
	\caption{Bucket($U, f, c, \beta, \epsilon, w_{\min}$)}
	\label{procedure:bucket}
	% Initialization
	\tcp{Initialization}
	$w_{\max} \leftarrow \max_{\{a, b\} \in U}f(\{a, b\})$
	\\
	$i_{\max} \leftarrow \ceil{\log_{1 + \epsilon}\frac{w_{\max}}{w_{\min}}\}}$
	\\
	\For{$i \in \{0,\dots,i_{\max}\}$}{
		$P_i \leftarrow \emptyset$
	}
	% Bucketing
	\tcp{Bucketing}
	\For{$\{a,b\} \in U$}{
		$P \leftarrow \{a,b\}$
		\\
		$i \leftarrow \floor{\log_{1 + \epsilon}\frac{f(P)}{w_{\min}}}$
		\\
		\eIf{$P_i = \emptyset$}{
			$P_i \leftarrow P$
		}{
			$P_i \leftarrow \displaystyle{\argmin_{Q \in \{P_i, P\}}c(Q)}$
		}
	}
	% Candidate Solutions
	\tcp{Candidate Solutions}
	\For{$i \in \{0,\dots,i_{\max}\}$}{
		$S_i \leftarrow P_i \cup \text{Modified}^2\text{Greedy}(U \setminus P_i, f_{P_i}, c, \beta - c(P_i))$
	}
	$B \leftarrow \arg\max_{S_i}f(S_i)$
	\\
	\Return{B}
\end{procedure}

It is easy to see that the running time of the procedure is $O(i_{\max}n^2)$,
later on we will show that we can upper bound $i_{\max}$ with a small, absolute constant.

Our algorithm~\ref{algorithm:bucket} returns the best among the solution of the Bucket procedure and the output of the Modified$^2$ Greedy algorithm.

\begin{algorithm}
	\caption{
		\label{algorithm:bucket}
		TODO$U, f, c, \beta, \epsilon, w_{\min}$
	}
	$B \leftarrow \text{Bucket}(TODO)$
	$M \leftarrow \text{Modified}^2(TODO)$
	\Return $\argmax_{S \in \{B, M\}}f(S)$
\end{algorithm}

We now analyze the performance of the TODO algorithm.
We start by showing that,  under certain conditions, the Modified$^2$ Greedy algorithm gives $1 - e^{-2/3}$-approximation ratio.
From now on fix an optimal solution, $O$, and denote by $\pLarge$ its two largest elements.
We also assume that an optimal solution contains an element larger than $(1/3)c(O)$, otherwise, by Lemma~\ref{lemma:sub-main}, we can obtain such approximation ratio by the greedy algorithm.
Now consider the third largest element in $O$, and denote its size by $(1/3 - \delta)c(O)$.
Finally, let $S$ be the output of the greedy algorithm, then we have the following lemma

\begin{lemma}
	\label{eq:sub:greedy}
	If the largest element in $O$ is bigger than $1/3$ then
	\begin{align*}
		f(S) & \geq (1-e^{-\frac{2/3 + \delta}{1 - 1/3 - (1/3 - \delta)}})f(O|\pLarge)
		\\		& \geq (1 - e^{-3/2})f(O|\pLarge)
	\end{align*}
\end{lemma}

\begin{proof}
	The proof follows immediately from Lemma~\ref{lemma:sub-main}.
\end{proof}

The following corollary state a condition under which the greedy algorithm is good enough:
\begin{corollary}
	\label{corollary:greedy-good}
	If $f(O|\pLarge) \geq \frac{1-e^{-2/3}}{1-e^{-3/2}}f(O)$ then $f(S) \geq (1 - e^{-2/3})f(O)$.
\end{corollary}

Now, let $\pValuable$ be the most valuable pair in the ground set.
We show that if the value of $\pValuable$ is large enough compared to the value of $\pLarge$ and if the greedy algorithm fails to find a good solution then $\pValuable$ by itself is a good solution, formally:

\begin{lemma}
	\label{lemma:sub:alpha}
	If $f(S) < (1 - e^{-2/3})f(O)$ then there is a constant $0 < \alpha < 1$ such that if $f(\pLarge) < \alpha f(\pValuable)$ then $f(\pValuable) \geq (1 - e^{-2/3})f(O)$.
\end{lemma}

\begin{proof}
	By the assumptions and by Corollary \ref{corollary:greedy-good} we know that 
	\begin{align}
		f(O)	& = f(\pLarge) + f(O|\pLarge) 
		\\ 		& < \alpha f(\pValuable) + \frac{1-e^{-2/3}}{1-e^{-3/2}}f(O)
	\end{align}
	So for $\alpha \leq \frac{e^{-2/3} - e^{-3/2}}{(1-e^{-2/3})(1-e^{-3/2})} \approx 0.77$ it holds that $f(\pValuable) \geq (1 - e^{-2/3})f(O)$.
\end{proof}

We show that either the 
\begin{lemma}
	\label{lemma:mmgreedy-or-bucket}
	If $f(M) < (1 - e^{-2/3})f(O)$ then $f(B) \geq (1 - e^{-2/3})f(O)$
\end{lemma}

\begin{proof}
	Let $P'$ be the pair considered by the algorithm from the bucket of $\pLarge$ then we know that:
	\begin{enumerate}
		\item $c(P') \leq c(\pLarge)$
		\item $f(P') \geq \frac{1}{(1 + \epsilon)}w(\pLarge)$
	\end{enumerate}
	Let $S_{i'}$ be the solution produced by the Bucket procedure from $P'$,
	then using Lemma~\ref{lemma:sub-main} and using the properties of $\pLarge$ we know that:
	\begin{align}
		f(S_{P'}) 
		& 
		\geq f(P') + (1-e^{-(1/2 + \eps)})f(O \setminus \pLarge \setminus P')
		\\ & 
		\geq f(P') + (1-e^{-(1/2 + \eps)})(1 - f(\pLarge) - f(P'))
		\\ & 
		\geq \frac{1}{(1 + \epsilon)}f(P) + (1-e^{-(1/2 + \eps)})(1 - 2f(P))
		\label{eq:sub:bucket}
	\end{align}
	Setting $\epsilon$ to (say) $0.002$ and taking the maximum between \ref{eq:sub:bucket} and
	\ref{eq:sub:greedy} completes the proof as can be seen in Figure~\ref{fig:sub:bucket}.
\end{proof}

From the discussions above we have the following theorem:

\begin{theorem}
	The TODO algorithm is $(1 - e^{-2/3})$-approximation.
\end{theorem}

\paragraph{Running Time:}
The Bucket procedure dominate the running time of the TODO algorithm, and, as mentioned above, its running time is $O(i_{\max}n^2)$.
The proofs of Lemma~\ref{lemma:sub:alpha} and from the proof of Lemma~\ref{lemma:mmgreedy-or-bucket} implies that when running the Bucket procedure it is enough to set $w_min = 0.77f(\pValuable)$ and to set $\epsilon = 0.002$ which upper bound $i_{\max}$ by 132.

\begin{figure}
	\def\zeta{0.002}
	\caption{
		\label{fig:sub:bucket}
		Bucket Algorithm - Approximation Ratio
	}
	\begin{tikzpicture}
		\begin{axis}[
				width=\textwidth
				,domain=0:1
				,ymax=.55
				,ymin=0.3
				% 	,xmax=.51
				,xlabel=$\omega$
				,xtick distance=0.1
				,ytick distance=0.1
				,axis lines=left
				,grid=both
				,grid style={
						draw=gray!20
					}
				,minor tick num=5
				,legend pos=south west
				,legend entries={
						Bucket Algorithm
						,Greedy Algorithm
						,$(1 - e^{-2/3})$
					}
			]
			\addplot[blue, dashed]{(1 + \zeta)^(-2)*x + (1 - e^(-(1/2 + \eps))) * (1 - 2 * x)};
			\addplot[red, dotted]{(1 - exp(-1.5)) * (1 - x)};
			\addplot[green]{(1 - exp(-2/3))};
		\end{axis}
	\end{tikzpicture}
\end{figure}
