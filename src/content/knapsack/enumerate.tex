Before describing the algorithm we further analyze the performance of the greedy algorithm.
Fix an optimal solution, $O$, and consider its two largest elements, $P = \{a, b\}$.
Denote $x = f(P)$ so $f(O \setminus P | P) = 1 - x$.
Now consider the third largest element in $O$, $e$, and denote $w(e) = 1/3 - \delta$,
and let $A$ be the output of the greedy algorithm, then we have the following theorem

\begin{lemma}
	\label{eq:sub:greedy}
	If the largest element in $O$ is bigger than $1/3$ then
	\begin{align*}
		f(A) & \geq (1-e^{-\frac{2/3 + \delta}{1 - 1/3 - (1/3 - \delta)}})(1 - x)
		\\		& \geq (1 - e^{-1.5})(1 - x)
	\end{align*}
\end{lemma}

\begin{proof}
	The proof follows immediately from Lemma~\ref{lemma:sub-main}.
\end{proof}

We now present a $(1-e^{-2/3})$-approximation algorithm that runs in $O(n^2)$ time.
This algorithm carefully selects a constant number of pairs of elements and greedily
add elements to them.
We assume that the size of the knapsack is 1.
Consider all pairs of elements, we describe now how to choose the relevant ones.

Let $\omega$ be the value of the most valuable pair and $\alpha$ and $\epsilon$ be constants to be determined later.
We divide the pairs into buckets where the $i$th bucket contains all the pairs with value
between $\alpha (1 + \epsilon)^{i - 1}$ and $\alpha (1 + \epsilon)^{i}$
We only consider the smallest pair from each bucket
(i.e. the pair that minimizes the weight function in each bucket).

Given such a pair, $P$, we construct a new Knapsack problem of budget $1 - c(P)$ and with respect to the function $f_P$.
We use the modified modified algorithm to obtain a $1-e^{-(1/2 + \eps)}$-approximation
solution to this problem and combine it with the initial pair to obtain a (candidate)
solution to the original problem.

The algorithm returns the best solution from the following candidates:
\begin{enumerate}
	\item Most valuable pair
	\item Best solution from the above candidates
	\item The solution of the simple greedy algorithm (starting from an empty set)
\end{enumerate}

\begin{theorem}
	The above algorithm is $(1 - e^{-2/3})-approximation$ algorithm.
\end{theorem}

\begin{proof}
	Fix an optimal solution, $O$, and consider two cases:
	\begin{enumerate}
		\item $O$ has at least one element larger than $1/3$
		\item All elements in $O$ are smaller or equal to $1/3$
	\end{enumerate}
	In the later case the proof follows immediately from Lemma~\ref{lemma:sub-main}.
	Now consider the first case and denote by $P^*$ the two largest elements in $O$.
	Let $P$ be the pair considered by the algorithm from the same bucket of $P^*$ (assume for now that indeed $P^*$ falls in one of the buckets)then we know
	that:
	\begin{enumerate}
		\item $c(P) \leq c(P^*)$
		\item $w(P) \geq (1 + \epsilon)^{-2}w(P^*)$
	\end{enumerate}
	Let $S_P$ be the solution produced by the algorithm from $P$, and let $\omega = \max\{f(P), f(P^*)\}$,
	then using Lemma~\ref{lemma:sub-main} and using the properties of $P$ we know that
	\begin{equation}
		\label{eq:sub:bucket}
		f(S_P) \geq \frac{1}{(1 + \epsilon)}\omega + (1-e^{-(1/2 + \eps)})(1 - 2\omega)
	\end{equation}
	Setting $\epsilon$ to (say) $0.002$ and taking the maximum between \ref{eq:sub:bucket} and
	\ref{eq:sub:greedy} completes the proof as can be seen in Figure~\ref{fig:sub:bucket}.

	We now consider the case where $P$ was not assigned to any of the buckets...
\end{proof}

\begin{figure}
	\def\zeta{0.002}
	\caption{
		\label{fig:sub:bucket}
		Bucket Algorithm - Approximation Ratio
	}
	\begin{tikzpicture}
		\begin{axis}[
				width=\textwidth
				,domain=0:1
				,ymax=.55
				,ymin=0.3
				% 	,xmax=.51
				,xlabel=$\omega$
				,xtick distance=0.1
				,ytick distance=0.1
				,axis lines=left
				,grid=both
				,grid style={
						draw=gray!20
					}
				,minor tick num=5
				,legend pos=south west
				,legend entries={
						Bucket Algorithm
						,Greedy Algorithm
						,$(1 - e^{-2/3})$
					}
			]
			\addplot[blue, dashed]{(1 + \zeta)^(-2)*x + (1 - e^(-(1/2 + \eps))) * (1 - 2 * x)};
			\addplot[red, dotted]{(1 - exp(-1.5)) * (1 - x)};
			\addplot[green]{(1 - exp(-2/3))};
		\end{axis}
	\end{tikzpicture}
\end{figure}
