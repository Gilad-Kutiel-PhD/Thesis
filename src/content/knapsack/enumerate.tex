\def\pLarge{P_{\text{large}}}
\def\pValuable{P_{\text{valuable}}}

We now present a $(1-e^{-2/3})$-approximation algorithm that runs in $O(n^2)$ time.
The core of this algorithm is the \ref{procedure:bucket} procedure.
This procedure groups pairs of elements with similar value into (a constant number of) buckets.
Then, from each bucket, it takes the smallest pair (i.e. the pair that minimizes $c$) and, using the Modified$^2$ Greedy algorithm, extend into a candidate solution.
Finally it returns the best solution among the above candidates.

\begin{procedure}
	% Initialization
	\tcp{Initialization}
	$w_{\max} \leftarrow \max_{\{a, b\} \in U}f(\{a, b\})$
	\\
	$i_{\max} \leftarrow \ceil{\log_{1 + \epsilon}\frac{w_{\max}}{w_{\min}}\}}$
	\\
	\For{$i \in \{0,\dots,i_{\max}\}$}{
		$P_i \leftarrow \emptyset$
	}
	% Bucketing
	\tcp{Bucketing}
	\For{$\{a,b\} \in U$}{
		\label{line:p}
		$P \leftarrow \{a,b\}$
		\\
		\label{line:i}
		$i \leftarrow \floor{\log_{1 + \epsilon}\frac{f(P)}{w_{\min}}}$
		\\
		\eIf{$P_i = \emptyset$}{
			$P_i \leftarrow P$
		}{
			$P_i \leftarrow \displaystyle{\argmin_{Q \in \{P_i, P\}}c(Q)}$
		}
	}
	% Candidate Solutions
	\tcp{Candidate Solutions}
	\For{$i \in \{0,\dots,i_{\max}\}$}{
		$S_i \leftarrow P_i \cup \text{Modified}^2\text{Greedy}(U \setminus P_i, f_{P_i}, c, \beta - c(P_i))$
	}
	$B \leftarrow \arg\max_{S_i}f(S_i)$
	\\
	\Return{B}
	% 
	\caption{Bucket($U, f, c, \beta, \epsilon, w_{\min}$)}
	\label{procedure:bucket}
\end{procedure}

The value of $i$ in line~\ref{line:i} of the procedure is the \emph{bucket} of $P$ from line~\ref{line:p} of the procedure.
It is easy to see that the running time of the procedure is $O(i_{\max}n^2)$,
later on we will show that we can upper bound $i_{\max}$ with a small, absolute constant.

Our algorithm~\ref{algorithm:bucket} returns the best among the solution of the Bucket procedure and the output of the Modified$^2$ Greedy algorithm.

\begin{algorithm}
	$B \leftarrow \text{Bucket}(TODO)$
	$M \leftarrow \text{Modified}^2(TODO)$
	\Return $\argmax_{S \in \{B, M\}}f(S)$
	% 
	\caption{TODO($U, f, c, \beta, \epsilon, w_{\min}$)}
	\label{algorithm:bucket}
\end{algorithm}

We now analyze the performance of the TODO algorithm.
We start by showing that,  under certain conditions, the Modified$^2$ Greedy algorithm gives $(1 - e^{-2/3})$-approximation ratio.
From now on fix an optimal solution, $O$, and denote by $\pLarge$ its two largest elements.
We also assume that an optimal solution contains an element larger than $(1/3)c(O)$, otherwise, by Lemma~\ref{lemma:sub-main}, we can obtain such approximation ratio by the greedy algorithm.
Now consider the third largest element in $O$, and denote its size by $(1/3 - \delta)c(O)$.
Finally, let $S$ be the output of the greedy algorithm, then we have the following lemma

Now consider the set of elements $O \setminus \pLarge$.
Its value is at least $f(O|\pLarge)$ and its size is at most $1 - \frac{1}{3} - (\frac{1}{3} - \delta)$.
Observe also that the size of the set of elements considered by a Greedy algorithm before dropping an element from $O \setminus \pLarge$ is at least $\frac{2}{3} + \delta$.
Using Lemma~\ref{lemma:sub-main} we have the following corollary:

\begin{corollary}
	\label{eq:sub:greedy}
	If the largest element in $O$ is bigger than $1/3$ then
	\begin{align*}
		f(S) & \geq (1-e^{-\frac{2/3 + \delta}{1 - 1/3 - (1/3 - \delta)}})f(O|\pLarge)
		\\		& \geq (1 - e^{-3/2})f(O|\pLarge)
	\end{align*}
\end{corollary}

Now we want to find how large $f(O|\pLarge)$ compared to $f(O)$ needs to be so that the solution of the Greedy algorithm is good enough, i.e. a $(1-e^{-2/3})$-approximation.

\begin{corollary}
	\label{corollary:greedy-good}
	If $f(O|\pLarge) \geq \frac{1-e^{-2/3}}{1-e^{-3/2}}f(O)$ then $f(S) \geq (1 - e^{-2/3})f(O)$.
\end{corollary}

Now, let $\pValuable$ be the most valuable pair in the ground set.
We show that if the value of $\pValuable$ is large enough compared to the value of $\pLarge$ and if the greedy algorithm fails to find a good solution, i.e. the condition in Corollary~\ref{corollary:greedy-good} does not hold, then the value of $\pValuable$ is large enough, formally:

\begin{lemma}
	\label{lemma:sub:alpha}
	If $f(S) < (1 - e^{-2/3})f(O)$ then there is a constant $0 < \alpha < 1$ such that if $f(\pLarge) < \alpha f(\pValuable)$ then $f(\pValuable) \geq (1 - e^{-2/3})f(O)$.
\end{lemma}

\begin{proof}
	By the assumptions and by Corollary \ref{corollary:greedy-good} we know that 
	\begin{align}
		f(O)	& = f(\pLarge) + f(O|\pLarge) 
		\\ 		& < \alpha f(\pValuable) + \frac{1-e^{-2/3}}{1-e^{-3/2}}f(O)
	\end{align}
	So for $\alpha \leq \frac{e^{-2/3} - e^{-3/2}}{(1-e^{-2/3})(1-e^{-3/2})} \approx 0.77$ it holds that $f(\pValuable) \geq (1 - e^{-2/3})f(O)$.
\end{proof}

We show that at least one of the solutions - the one returned by the Modified$^2$ Greedy algorithm and the one returned by the Bucket procedure - is good enough, i.e. is a $(1 - e^{-2/3})$- approximation.

\begin{lemma}
	\label{lemma:mmgreedy-or-bucket}
	If $f(M) < (1 - e^{-2/3})f(O)$ then $f(B) \geq (1 - e^{-2/3})f(O)$
\end{lemma}

\begin{proof}
	Let $P_j$ be the pair considered by the algorithm from the bucket of $\pLarge$ then we know that:
	\begin{enumerate}
		\item $c(P_j) \leq c(\pLarge)$
		\item $f(P_j) \geq \frac{1}{(1 + \epsilon)}w(\pLarge)$
	\end{enumerate}
	Let $S_{j}$ be the solution produced by the Bucket procedure from $P_j$,
	then using Lemma~\ref{lemma:sub-main} and using the properties of $\pLarge$ we know that:
	\begin{align}
		f(S_{j}) 
		&
		\geq f(P_i) + f(\text{Modified}^2(O \setminus P_j), f_{P_j}, c, \beta - c(P_j))
		\\ & 
		\geq f(P_j) + (1-e^{-(1/2 + \eps)})f(O \setminus \pLarge | P_j)
		\\ & 
		\geq f(P_j) + (1-e^{-(1/2 + \eps)})(f(O) - f(\pLarge) - f(P_j))
		\\ & 
		\geq \frac{1}{(1 + \epsilon)}f(\pLarge) + (f(O)-e^{-(1/2 + \eps)})(1 - 2f(\pLarge))
		\label{eq:sub:bucket}
	\end{align}
	Setting $\epsilon$ to (say) $0.002$ and taking the maximum between~\ref{eq:sub:bucket} and Corollary~\ref{eq:sub:greedy} completes the proof as can be seen in Figure~\ref{fig:sub:bucket}.
\end{proof}

From the discussions above we have the following theorem:

\begin{theorem}
	The TODO algorithm is $(1 - e^{-2/3})$-approximation.
\end{theorem}

\paragraph{Running Time:}
The Bucket procedure dominate the running time of the TODO algorithm, and, as mentioned above, its running time is $O(i_{\max}n^2)$.
The proofs of Lemma~\ref{lemma:sub:alpha} and from the proof of Lemma~\ref{lemma:mmgreedy-or-bucket} implies that when running the Bucket procedure it is enough to set $w_min = 0.77f(\pValuable)$ and to set $\epsilon = 0.002$ which upper bound $i_{\max}$ by 132.

\begin{figure}
	\def\zeta{0.002}
	\caption{
		\label{fig:sub:bucket}
		Bucket Algorithm - Approximation Ratio
	}
	\begin{tikzpicture}
		\begin{axis}[
				width=\textwidth
				,domain=0:1
				,ymax=.55
				,ymin=0.3
				% 	,xmax=.51
				,xlabel=$\omega$
				,xtick distance=0.1
				,ytick distance=0.1
				,axis lines=left
				,grid=both
				,grid style={
						draw=gray!20
					}
				,minor tick num=5
				,legend pos=south west
				,legend entries={
						Bucket Algorithm
						,Greedy Algorithm
						,$(1 - e^{-2/3})$
					}
			]
			\addplot[blue, dashed]{(1 + \zeta)^(-2)*x + (1 - e^(-(1/2 + \eps))) * (1 - 2 * x)};
			\addplot[red, dotted]{(1 - exp(-1.5)) * (1 - x)};
			\addplot[green]{(1 - exp(-2/3))};
		\end{axis}
	\end{tikzpicture}
\end{figure}
