Agatz et al.~\cite{agatz2012optimization} outlined the optimization
challenges that arise when developing technology to support
ride-sharing and survey the related operations research models in the
academic literature.
%
Hartman et al.~\cite{hartman2014theory} designed several heuristic
algorithms for the \carpool problem and compared their performance on
real data.  Other heuristic algorithms were developed by Knapen et
al.~\cite{knapen2014exploiting}.
%
Hartman~\cite{hartman2013optimal} proved that the \carpool problem is
NP-hard even in the case where the weight function is binary and
$c(v) \leq 2$ for every $v \in V$.  In addition, Hartman presented a
natural integer linear program and showed that if the set of drivers
is known, then an optimal assignment of passengers to drivers can be
found in polynomial time using a reduction to \textsc{Network Flow}
(see also~\cite{kutiel2017}.)
%
Nguyen et al.~\cite{nguyen2008approximating} considered
the \textsc{Spanning Star Forest} problem.  A \emph{star forest} is a
graph consisting of node-disjoint star graphs.  In
the \textsc{Spanning Star Forest} problem, we are given an undirected
graph $G$, and the goal is to find a spanning subgraph which is a star
forest that maximizes the weight of edges that are covered by the star
forest.  Notice that this problem is equivalent to \carpool on
undirected and uncapacitated instances.  We also note that if all
weights leaving a node are the same, then the instance is referred to
as node-weighted.
%
Nguyen et al.~\cite{nguyen2008approximating} provided a PTAS for
unweighted planner graphs and a polynomial-time
$\frac{3}{5}$-approximation algorithm for unweighted graphs.  They
gave an exact optimization algorithm for weighted trees, and used it
on a maximum spanning tree of the input graph to obtain a
$\frac{1}{2}$-approximation algorithm for weighted graphs.  They also
shows that it is NP-hard to approximate unweighted \textsc{Spanning
Star Forest} within a ratio of $\frac{259}{260}+\eps$, for any
$\eps>0$.
%
%They also showed how to apply the spanning star forest model to
%aligning multiple genomic sequences over a tandem duplication region.
%
Chen et al.~\cite{CENRRS13} improved the approximation ratio for
unweighted graphs from $\frac{3}{5}$ to $0.71$ and gave a
$0.64$-approximation algorithm for node weighted graphs.  They also
showed that the edge- and node-weighted problem cannot be approximated
to within a factor of $\frac{19}{20} + \eps$, and $\frac{31}{32}
+ \eps$, resp., for any $\eps > 0$, assuming that
$\text{P} \neq \text{NP}$.
%
Chakrabarty and Goel~\cite{ChakrabartyGoel10} improved the lower bounds
to $\frac{10}{11} + \eps$ and $\frac{13}{14}$.

Athanassopoulos et al.~\cite{ACKK09} improve the ratio for the
unweighted case to $\frac{193}{240} \approx 0.804$.
%
They consider a natural family of \emph{local search} algorithms
for \textsc{Spanning Star Forest}.  Such an algorithm starts with the
solution where all node are star centers.  Then, it repeatedly tries
to turn $t \leq k$ from leaves to centers and $t+1$ centers to leaves.
A change is made if it results in a feasible solution, namely if each
leave is adjacent to at least one center.  The algorithm terminates
when such changes are no longer possible.
%
Athanassopoulos et al.~\cite{ACKK09} showed that, for any $k$ and
$\eps \in (0,\inv{2(k+2)}]$, there exists an instance $G$ and a local
optima whose size is smaller than $(\half + \eps) \textsc{opt}$,
where \textsc{opt} is the size of the optimal spanning star forest.
We note that, for a given $k$, the construction of the above result
requires that the maximum degree of $G$ is at least $2(k+2)$.  Hence,
this result does not hold in graphs with maximum degree $\Delta$.

Arkin et al.~\cite{arkin2004approximations} considered
the \textsc{Maximum Capacitated Star Packing} problem.  In this
problem the input consists of a complete undirected graph with
non-negative edge weights and a capacity vector $c
= \set{c_1,\ldots,c_p}$, where $\sum_{i=1}^p c_i = \abs{V} - p$.  The
goal is to find a set of vertex-disjoint stars in $G$ of size
$c_1,\ldots,c_p$ of maximum total weight.  Arkin et
al.~\cite{arkin2004approximations} provided a local search algorithm
whose approximation ratio is $\inv{3}$, and a matching-based
$\half$-approximation algorithm for the case where edge weights
satisfy the triangle inequality.

Bar-Noy et al.~\cite{bar2015improved} considered the
\textsc{Minimum $2$-Path Partition} problem.
In this problem the input is a complete graph on $3k$ vertices with
non-negative edge weights, and the goal is to partition the graph into
disjoint paths of length 2.  This problem is the special case of the
undirected carpool matching where $c(v) = 2$, for every $v \in V$.
They presented two approximation algorithms, one for the weighted case
whose ratio is $0.5833$, and another for the unweighted case whose
ratio is $\frac{3}{4}$.

Another related problem is the \textsc{$k$-Set Packing}, where one is
given a collection of weighted sets, each containing at most $k$
elements, and the goal is to find a maximum weight subcollection of
disjoint sets.  Chandra and Halld\'orsson~\cite{chandra2001greedy}
presented a $\frac{3}{2(k+1)}$-approximation algorithm for this
problem.
%
\carpool can be seen as a special case of \textsc{$k$-Set Packing}
with $k = \cmax + 1$.  Consider a subset of nodes $U$ of size at most
$k$.  Observe that each subset of nodes has an optimal internal
assignment of passenger to drivers.  Let the weight of this assignment
be the profit of $U$, denote $p(U)$.  If $k = O(1)$, $p(U)$ can be
computed for every $U$ of size at most $k$ in polynomial time.  The
outcome is a \textsc{$k$-Set Packing} instance.  This leads to a
$\frac{3}{2(\cmax+2)}$-approximation algorithm for the case where
$\cmax = O(1)$.


%%%%%
