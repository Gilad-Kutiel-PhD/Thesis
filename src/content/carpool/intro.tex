
As traveling costs become higher and parking becomes sparse it is only
natural to share rides or to \emph{carpool}.  Originally, carpooling
was an arrangement among a group of people by which they take turns
driving the others to and from a designated location.  However, taking
turns is not essential, instead passengers can share the cost of the
ride with the driver.  Carpooling has social advantages other than
reducing the costs: it reduces fuel consumption and road congestion
and frees parking space.  
%
While in the past carpooling was usually a fixed arrangement between
friends or neighbors, the emergence of social networks has made
carpooling more dynamic and wide scale.  These days applications like
Zimride~\cite{zimride}, BlaBlaCar~\cite{blablacar},
Moovit~\cite{moovit} and even Waze~\cite{waze} are matching passengers
to drivers.

The matching process of passengers to drivers entails more than
matching the route.
%
Passenger satisfaction also need to be taken into account.  Given
several riding options (including taking their own car), passengers
have preferences.  For example, a passenger may prefer to ride with a
co-worker or a friend.  A passenger may have an opinion on a driver
that she rode with in the past.  She may prefer a non-smoker, someone
who shares her taste in music, or someone who is recommended by
others.
%
Moreover, the matching process may take into account driver
preferences.  For instance, we would like to minimize the extra
distance that a driver has to take.
%
Preferences may also be computed using past information.  Knapen et
al.~\cite{knapen2013estimating} described an automatic service to
match commuting trips.  Users of the service register their personal
profile and a set of periodically recurring trips, and the service
advises registered candidates on how to combine their commuting trips
by carpooling.  The service estimates the probability that a person
$a$ traveling in person's $b$ car will be satisfied by the trip.  This
is done based on personal information and feedback from users on past
rides.

In this paper we assume that potential passenger-driver satisfactions
are given as input and the goal is to compute an assignment of
passengers to drivers so as to maximize the global satisfaction.
%
More formally, we are given a directed graph $G = (V, A)$, where each
vertex $v \in V$ corresponds to a user of the service, and an arc $(u,
v)$ exists if the user corresponding to vertex $u$ is willing to
commute with the user corresponding to vertex $v$.  We are given a
capacity function $c: V \rightarrow \N$ which bounds the number of
passengers each user can drive if she is selected as a driver.  A
non-negative weight function $w : A \rightarrow \R^+$ is used to model
the amount of satisfaction $w(u,v)$ of assigning $u$ to $v$.
%
If $(u,v) \in A$ implies that $(v,u) \in A$ and $w(u,v) = w(v,u)$, the
instance is \emph{undirected}.  If $w(v,u) = 1$, for every $(v,u) \in
A$, then the instance is \emph{unweighted}.  If $c(v) = \deg(v)$, for
every $v$, then the instance in \emph{uncapacitated}.

Given a directed graph $G$ and a subset $M \subseteq A$, define
$$\din[M](v) \eqdf \abs{\set{u :(u,v) \in M}}$$ 
and
$$\dout[M](v) \eqdf \abs{\set{u :(v,u) \in M}}$$
%
A feasible \emph{carpool matching} is a subset of arcs, $M \subseteq
A$, such that every $v \in V$ satisfies:%
\begin{inparaenum}[(i)]
\item $\din[M](v) \cdot \dout[M](v) = 0$,
\item $\din[M](v) \leq c(v)$, and 
\item $\dout[M](v) \leq 1$.
\end{inparaenum}
A feasible carpool matching $M$ partitions $V$ as follows:
\begin{align*}
P_M & \eqdf \set{v : \dout[M](v) = 1} &
\\
D_M & \eqdf \set{v : \din[M](v) \geq 1} &
\\
Z_M & \eqdf \set{v : \dout[M](v) = \din[M](c) = 0}
%~.
\end{align*}
where $P_M$ is the set of \emph{passengers}, $D_M$ is the set of
\emph{active drivers}, and $Z_M$ is the set of \emph{solo drivers}.
%
In the \carpool problem the goal is to find a matching $M$ of maximum
total weight, namely to maximize $w(M) \eqdf \sum_{(v,u) \in M}
w(v,u)$.  In other words, the \carpool problem is about finding a set
of (directed toward the center) vertex disjoint stars that maximizes
the total weight of the arcs.  
%
Figure~\ref{fig:carpool-new} contains an example of a \carpool instance.
%
Note that in the unweighted case the goal is to find a carpool
matching $M$ that maximizes $\abs{P_M}$.
%
Moreover, observe that if $G$ is undirected, $D_M \cup Z_M$ is
a \emph{dominating set}.  Hence, in this case, an optimal carpool
matching induces an optimal dominating set and vice versa.
Since \textsc{Minimum Dominating Set} is NP-hard, it follows
that \carpool is NP-hard even if the instance is undirected,
unweighted, and uncapacitated.

\begin{figure}
\centering
\add{fig-carpool}
\caption[]{
\label{fig:carpool-new}
A \carpool example.
%(\subref{carpool:input}) an instance containing a
%directed graph with capacities on the vertices and weights on the
%arcs.  (\subref{carpool:output}) a feasible matching with total weight
%of 23.  $P_M$ is the set of blue vertices, and $D_M$ is the set of
%red, dashed vertices, and $Z_M$ contains only the dotted, black
%vertex.
}
\end{figure}  

%In this paper
%
We also consider an extension of \carpool, called \gcp, in which each
vertex represents a group of passengers, and each group may have a
different size.  Such a group may represent a family or two friends
traveling together.  Formally, each vertex $u \in V$ has a size
$s(u) \in \N$, and the constraint $\din[M](v) \leq c(v)$ is replaced
with the constraint $\sum_{u:(u,v) \in M} s(u) \leq c(v)$.
%
Notice that \textsc{Knapsack} is the special case where only arcs
directed at a single vertex have non-zero (integral) weights.

%%%%%

\subparagraph{Related work.}
%
Agatz et al.~\cite{agatz2012optimization} outlined the optimization
challenges that arise when developing technology to support
ride-sharing and survey the related operations research models in the
academic literature.
%
Hartman et al.~\cite{hartman2014theory} designed several heuristic
algorithms for the \carpool problem and compared their performance on
real data.  Other heuristic algorithms were developed by Knapen et
al.~\cite{knapen2014exploiting}.
%
Hartman~\cite{hartman2013optimal} proved that the \carpool problem is
NP-hard even in the case where the weight function is binary and
$c(v) \leq 2$ for every $v \in V$.  In addition, Hartman presented a
natural integer linear program and showed that if the set of drivers
is known, then an optimal assignment of passengers to drivers can be
found in polynomial time using a reduction to \textsc{Network Flow}
(see also~\cite{kutiel2017}.)
%
Kutiel~\cite{kutiel2017} presented a $\frac{1}{3}$-approximation
algorithm for \carpool that is based on a \textsc{Minimum Cost Flow}
computation and a local search $\half$-approximation algorithm for the
unweighted variant of \carpool.  The latter starts with an empty
matching and tries to improve the matching by turning a single
passenger into a driver.

Nguyen et al.~\cite{nguyen2008approximating} considered
the \textsc{Spanning Star Forest} problem.  A \emph{star forest} is a
graph consisting of node-disjoint star graphs.  In
the \textsc{Spanning Star Forest} problem, we are given an undirected
graph $G$, and the goal is to find a spanning subgraph which is a star
forest that maximizes the weight of edges that are covered by the star
forest.  Notice that this problem is equivalent to \carpool on
undirected and uncapacitated instances.  We also note that if all
weights leaving a node are the same, then the instance is referred to
as node-weighted.
%
Nguyen et al.~\cite{nguyen2008approximating} provided a PTAS for
unweighted planner graphs and a polynomial-time
$\frac{3}{5}$-approximation algorithm for unweighted graphs.  They
gave an exact optimization algorithm for weighted trees, and used it
on a maximum spanning tree of the input graph to obtain a
$\frac{1}{2}$-approximation algorithm for weighted graphs.  They also
shows that it is NP-hard to approximate unweighted \textsc{Spanning
Star Forest} within a ratio of $\frac{259}{260}+\eps$, for any
$\eps>0$.
%
%They also showed how to apply the spanning star forest model to
%aligning multiple genomic sequences over a tandem duplication region.
%
Chen et al.~\cite{CENRRS13} improved the approximation ratio for
unweighted graphs from $\frac{3}{5}$ to $0.71$ and gave a
$0.64$-approximation algorithm for node weighted graphs.  They also
showed that the edge- and node-weighted problem cannot be approximated
to within a factor of $\frac{19}{20} + \eps$, and $\frac{31}{32}
+ \eps$, resp., for any $\eps > 0$, assuming that
$\text{P} \neq \text{NP}$.
%
Chakrabarty and Goel~\cite{ChakrabartyGoel10} improved the lower bounds
to $\frac{10}{11} + \eps$ and $\frac{13}{14}$.

Athanassopoulos et al.~\cite{ACKK09} improve the ratio for the
unweighted case to $\frac{193}{240} \approx 0.804$.
%
They consider a natural family of \emph{local search} algorithms
for \textsc{Spanning Star Forest}.  Such an algorithm starts with the
solution where all node are star centers.  Then, it repeatedly tries
to turn $t \leq k$ from leaves to centers and $t+1$ centers to leaves.
A change is made if it results in a feasible solution, namely if each
leave is adjacent to at least one center.  The algorithm terminates
when such changes are no longer possible.
%
Athanassopoulos et al.~\cite{ACKK09} showed that, for any $k$ and
$\eps \in (0,\inv{2(k+2)}]$, there exists an instance $G$ and a local
optima whose size is smaller than $(\half + \eps) \textsc{opt}$,
where \textsc{opt} is the size of the optimal spanning star forest.
We note that, for a given $k$, the construction of the above result
requires that the maximum degree of $G$ is at least $2(k+2)$.  Hence,
this result does not hold in graphs with maximum degree $\Delta$.

Arkin et al.~\cite{arkin2004approximations} considered
the \textsc{Maximum Capacitated Star Packing} problem.  In this
problem the input consists of a complete undirected graph with
non-negative edge weights and a capacity vector $c
= \set{c_1,\ldots,c_p}$, where $\sum_{i=1}^p c_i = \abs{V} - p$.  The
goal is to find a set of vertex-disjoint stars in $G$ of size
$c_1,\ldots,c_p$ of maximum total weight.  Arkin et
al.~\cite{arkin2004approximations} provided a local search algorithm
whose approximation ratio is $\inv{3}$, and a matching-based
$\half$-approximation algorithm for the case where edge weights
satisfy the triangle inequality.

Bar-Noy et al.~\cite{bar2015improved} considered the
\textsc{Minimum $2$-Path Partition} problem.
In this problem the input is a complete graph on $3k$ vertices with
non-negative edge weights, and the goal is to partition the graph into
disjoint paths of length 2.  This problem is the special case of the
undirected carpool matching where $c(v) = 2$, for every $v \in V$.
They presented two approximation algorithms, one for the weighted case
whose ratio is $0.5833$, and another for the unweighted case whose
ratio is $\frac{3}{4}$.

Another related problem is the \textsc{$k$-Set Packing}, where one is
given a collection of weighted sets, each containing at most $k$
elements, and the goal is to find a maximum weight sub-collection of
disjoint sets.  Chandra and Halldorsson~\cite{chandra2001greedy}
presented a $\frac{3}{2(k+1)}$-approximation algorithm for this
problem.
%
\carpool can be seen as a special case of \textsc{$k$-Set Packing}
with $k = \cmax + 1$.  Consider a subset of nodes $U$ of size at most
$k$.  Observe that each subset of nodes has an optimal internal
assignment of passenger to drivers.  Let the weight of this assignment
be the profit of $U$, denote $p(U)$.  If $k = O(1)$, $p(U)$ can be
computed for every $U$ of size at most $k$ in polynomial time.  The
outcome is a \textsc{$k$-Set Packing} instance.  This leads to a
$\frac{3}{2(\cmax+2)}$-approximation algorithm for the case where
$\cmax = O(1)$.


%%%%%

\subparagraph{Our contribution.}
%
Section~\ref{sec:approx} contains approximation algorithms
for the \carpool problem.  
First, in Section~\ref{sec:sub} we show that \carpool
can be formulated as an unconstrained submodular maximization problem,
thus it has a $\frac{1}{2}$-approximation algorithm due
to~\cite{BFNS15,buchbinder2016deterministic}.
%
We present a local search algorithm for \carpool which repeatedly
checks whether the current carpool matching can be improved by means
of a star centered at a vertex, and it terminates when such a step is
not possible.
%
The approximation ratio of this algorithm is $\half$ if weights are
polynomially bounded, and its ratio is $\half-\eps$ in general.

In Section~\ref{sec:cmax} we consider \carpool with bounded maximum
capacity.   
%
In Section~\ref{sec:hardness} we show that \carpool is APX-hard even
for undirected and unweighted instances with $\Delta \leq b$, for any
$b \geq 3$.
%
In Section~\ref{sec:local} we provide another local search algorithm,
whose approximation ratio is $\half + \inv{2\cmax} - \eps$, for any
$\eps>0$, for unweighted \carpool, where $\cmax \eqdf \max_{v \in V}
c(v)$.  Given a parameter $k$, our algorithm starts with the empty
carpool matching.  Then, it repeatedly tries to find a better matching
by replacing $t \leq k$ arcs in the current solution by $t+1$ arcs
that are not in the solution.
%
We show that our analysis is tight.
%
We also note that our algorithm falls within the local search family
defined in~\cite{ACKK09}.  However, on undirected and uncapacitated
instances we have that $\cmax = \Delta$, and as mentioned above the
result from~\cite{ACKK09} does not hold in bounded degree graphs.

Finally, Section~\ref{sec:group} discusses \gcp.  We show that the
unconstrained submodular maximization formulation for \carpool does
not work for \gcp.  We show, however, that this problem still admits a
$(\frac{1}{2} -\varepsilon)$-approximation algorithm by extending our
first local search algorithm.  In addition, we show that the second
local search algorithm generalizes to unweighted \gcp with the same
approximation ratio.
