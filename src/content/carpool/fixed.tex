\label{sec:fixed}
In the \textsc{\FIXEDCARPOOL{}} problem, $P$ and $D$ are given, 
and the goal is to find $M$ that maximizes the total weight. 
This variant of the problem can be solved efficiently
\footnote{A solution to this variant of the problem was already proposed in~\cite{hartman2014theory}.
For the sake of completeness, however, we describe a detailed solution for this variant. 
More importantly, 
the described solution helps us develop the intuition and understand the basic idea behind the
approximation algorithm described in Section~\ref{sec:cm}.   
},
by reducing it to a maximum weight flow (flow) problem as
follow:
Let $(G = (V, A), c, w)$ be a \CARPOOL{} instance,
let $(P, D)$ be a partition of $V$,
let  $N = (G' = (V', A'), s, t, c')$ be a flow network, 
and let $w' : A \rightarrow \N$ be a weight function, where 

\begin{align*}
V'			& = P \cup D \cup \{s, t\}										\\
A'			& = A_{sp} \cup A_{pd} \cup A_{dt}								\\
A_{sp}		& =	\{(s, p) : p \in P \}										\\
A_{pd}		& =	A \cap (P \times D)											\\
A_{dt}		& =	\{(d, t) : d \in D \}										\\
c'(u, v)	& = 
				\begin{cases}
				c(u) & \text{if } (u, v) \in A_{dt} 						\\
				1 & \text{otherwise}
				\end{cases}
																			\\
w'(e)			& = 
				\begin{cases}
				w(e) & \text{if } e \in A_{pd} 								\\
				0 & \text{otherwise}	
				\end{cases}
\end{align*}
The flow network is described in Figure~\ref{fig:flow}.
\begin{figure}
\centering
\add{fig-flow}
\caption{
\label{fig:flow}
Illustration of a flow network corresponding to a \FIXEDCARPOOL{} instance.}
\end{figure}

\begin{observation}
For every integral flow $f$ in $N$, there is a carpool matching $M$ on $G$ with
the same weight.
\end{observation}

\begin{proof}
Consider the carpool matching $(P, D, M^f)$, where 
$$ M^f = \{(p, d) \in A_{pd} : f(p, d) = 1\} $$
one can verify that this is indeed a matching with the same weight as $f$.
\end{proof}

\begin{observation}
For every carpool matching $(P, D, M)$ on $G$, there exists a flow $f$ on $N$
with the same weight.
\end{observation}

\begin{proof}
Consider the flow function
\begin{align*}
f(s, p_i)		& = \dout(p_i)		 				\\
f(p_i, d_j)		& = 
				\begin{cases}
				1 & \text{if } (p_i, d_j) \in M		\\
				0 & \text{otherwise}
				\end{cases}						\\
f(d_j, t) 	& = \din(d_j) 
\end{align*}

It is easy to verify, that $f$ is indeed a flow function.
Also, observe, that by construction,
the weight of $f$ equals to the weight of the matching.
\end{proof}

As we mentioned, 
the maximum weight flow problem can be solved efficiently, 
and so is the \FIXEDCARPOOL{} problem.
It is worth mentioning, that it is possible that in a maximum weight flow, 
some of the arcs will have no flow at all, 
that is, it is possible that in a \FIXEDCARPOOL{}
some of the passengers and drivers will be unmatched.  