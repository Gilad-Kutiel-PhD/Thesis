We now present a 3-approximation algorithm for the \textsc{\CARPOOL{}} problem.
This algorithm acts in two phases.
In the first phase it computes a maximum super-matching of $G$, 
in the second phase it decomposes the super-matching into 3 feasible
carpool matching and output the best of them.

We now describe how a super-matching can be decomposed into 3 feasible carpool
matching.
First, consider the graph obtained by an optimal super-matching.
Recall that in a super matching the out degree of every vertex is at most 1,
that is, the graph obtained by an optimal super matching is a pseudoforest -
every connected component has at most one cycle.
We now eliminate cycles from the super-matching by removing one edge from every
connected component.
It is easy to see that the resulting graph is a forest of anti-arborescences.
Each of these anti-arborescences can be, in turn, decomposed into two disjoint
feasible carpool matching.
This can be done, for example, by coloring each such anti-arborescences with two
colors, say red and green, and then consider the two solutions: one where the
green nodes are the drivers, and the other where the red nodes are the drivers.
We describe the algorithm in Algorithm~\ref{alg:cm3}, 
and illustrate it in Figure~\ref{fig:spanning-bipartite-graph}.   

\begin{algorithm}[t]
\KwIn{$G = (V, A), c : V \rightarrow \N, w : A \rightarrow \R$}											 
\KwOut{$(M \subseteq A)$}

$M \leftarrow \emptyset$								\\
$G' = (V, A') \leftarrow \text{superMatching}(G)$				\\

\For{every connected component $C_i = (V_i, A_i) \in G'$}{	
	Eliminate the cycle in $C_i$ by removing an arc $a_i$							\\
	Decompose the remains anti-arborescences into two solutions, $M^i_1$, $M^i_2$	\\
	$M \gets M \cup \argmax_{F \in \{ \{e\}, M_1, M_2 \}}{w(F)}$	
}

\Return $M$
\caption{
\label{alg:cm3}
SuperMatching}
\end{algorithm}

\begin{figure}
\centering
\add{fig-super-matching-alg}
\caption[]{
\label{fig:spanning-bipartite-graph}
Illustration of the SuperMatching algorithm:
\subref{subfloat:graph} a directed graph. 
\subref{subfloat:super} a maximum super-matching.  
\subref{subfloat:spanning} an anti-arborescences:
$M_1$ is the set of arcs exiting red, dashed vertices, 
and $M_2$ is the set of arcs exiting blue vertices.
\subref{subfloat:matching} a feasible carpool matching with total value of 6.   
}
\end{figure}

\begin{theorem}
Algorithm~\ref{alg:cm3} achieves a 3-approximation ratio.
\end{theorem}

\begin{proof}
Let $M_a = \bigcup_i \{a_i\}$ be the set of all removed arcs in the cycle
elimination phase.
Let $M_1 = \bigcup_i M^i_1$, and $M_2 = \bigcup_i M^i_2$.
Clearly, $M_a \cup M_1 \cup M_2 = A'$, 
and that $\max(w(M_a), w(M_1), w(M_2)) \geq
\frac{w(A')}{3}$.
The observation that the weight of a maximum super-matching is an upper bound on
the weight of a maximum carpool matching finishes the proof. 
\end{proof}

To see that our analysis is tight, consider the example in
Figure~\ref{fig:3cm-tight-fig}. 
Assume, for the given graph in the figure, 
that all weights are 1 and that there is no capacity constraint.
The maximum matching, then, is 3 ($\{(1,4), (2,4), (3,4)\}$), 
but the algorithm can return the super matching $\{(1,2), (2,3), (3,1)\}$ from which
only one arc can survive.  

\begin{figure}
\centering
\add{fig-3cm-tight}
\caption{
\label{fig:3cm-tight-fig}
Super Matching Algorithm, worst case example
}
\end{figure}
