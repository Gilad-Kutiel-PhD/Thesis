In this thesis we develop approximation algorithms for four NP-hard combinatorial optimization problems.
Here we briefly (and sometimes informally) describe the problems and our results, a more detailed description to each of the problems appears at the beginning of each chapter.

\subsection*{Convex Recoloring} 
In Chapter~\ref{chapter:convex-recoloring} we study the Convex Recoloring problem.
A \emph{coloring} of an undirected graph, $G(V, E)$, is a function $f:V \to \mathbb{N}$.
We say that a coloring is \emph{convex} if the vertices of each color induce a connected sub-graph of $G$.
In the Convex Recoloring problem we are given an undirected graph $G$ and a (non-convex) coloring of this graph and our goal is to find a convex coloring of the graph that recolor the minimum number of vertices.
In this thesis we consider a special case of this problem, namely the 2-Convex Recoloring problem in which the original coloring colors at most two vertices with the same color.
For this special case we develop a greedy $3/2$-approximation algorithm.
We show that if the input graph is a path then this is in fact a $5/4$-approximation algorithm.


\subsection*{Service Chain Placement in SDNs} 
Traditional networks use ad-hoc controller to achieve various functionality and to provide various services to the clients of the network.
An emerging paradigm aim to construct networks of general, smart controllers that can programed to achieve the same goal. 
This new paradigm offers much more flexibility to the network administrator that can now construct many virtual networks on top of the physical one.
The question that arises is how to allocate resource efficiently in order to satisfy clients demand.
In Chapter~\ref{chapter:service-chain-placement-in-sdns} we model the problem and, using a rounding and non-trivial dynamic programming technique, we give a fully polynomial approximation scheme for the special case where the physical network is directed acyclic graph. 
For general graphs we give a parameterized algorithm.
This algorithm find optimal solution efficiently when the cactus network.

\subsection*{Monotone Submodular Function Maximization} 
In Chapter~\ref{chapter:monotone-submodular-function-maximization} we consider the problem of maximizing a monotone, submodular function under knapsack constraint.
There is a well known $(1-e^{-1})$-approximation algorithm for this problem that runs in $O(n^5)$ time.
This is the best approximation ratio that can be achieved by a polynomial algorithm unless $\text{P} = \text{NP}$.
A (asymptotic) faster $1 - e^{-1} - \epsilon$-approximation algorithm is also known.
This algorithm runs in $1/\epsilon^{O(1/\epsilon^4)}n\log^2n$ time for any choice of $\epsilon$ and, as the authors of this algorithm mention, is impractical.
In this thesis we develop a $(1-e^{-2/3})$-approximation algorithm that runs in time $O(n^2)$ where the big O hides sensible constant.

\subsection*{Maximum Carpool Matching}
In Chapter~\ref{chapter:maximum-carpool-matching} we consider the Maximum Carpool Matching problem.
In this problem we are given an edge weighted directed graph, $G=(V, E)$, and capacity constraints on the vertices, $c:V \to \mathbb{N}$.
Our goal is to find a matching, $M \subseteq E$, with maximum total weight such that in $H = (V, M)$ the out degree of each vertex is at most one, the in degree of each vertex, $v$, is at most $c(v)$, and either the out degree or the in degree of each vertex is zero.
We develop several approximation algorithms for several special and more general cases of the problem.
We also show that this problem can be formalized as a non-monotone, unconstrained, submodular function maximization and, thus, admits a 2-approximation algorithm.