In this thesis we develop and analyze approximation algorithms, while focusing on two families of problems: network design and submodular optimization.
For the former family we consider the problems of \textsc{Convex Recoloring} and the \textsc{Service Chain Placement in SDNs}.
For the latter family we consider the problems of maximizing a monotone submodular function given a knapsack constraint and the \textsc{Maximum Carpool Matching} problem.
Let us now elaborate on the above.

\subsection*{Network Design Problems}
In a typical instance of a \emph{network design} problem we are given a collection of resources and our goal is to construct a desired network that satisfies some requirements while utilizing the minimum possible amount of resources.
One of the classical network design problems (and probably the most well known) is the \textsc{Minimum Spanning Tree} problem where, given an undirected weighted graph, our goal is to find the minimum cost set of edges that induces a connected graph.
Network design problems have numerous applications, e.g., the construction of efficient communication and traffic networks, the design of small and cheap VLSI chips and the reconstruction of phylogenetic trees, are just a few.

Many of the practical network design problems are NP-hard and thus likely intractable. 
In this thesis we focus on approximation algorithms for two network design problems which we now briefly (and somewhat informally) describe\footnote{A formal, detailed, description of each problem will be given at the relevant chapters.}:


\subsubsection*{Convex Recoloring} 
In Chapter~\ref{chapter:convex-recoloring} we study the \textsc{Convex Recoloring} problem.
Let $\mathcal{C}$ be a set of colors, a \emph{coloring} of an undirected graph, $G(V, E)$, is a function $\chi:V \to \mathcal{C}$.
We say that a coloring is \emph{convex} if the vertices of each color induce a connected sub-graph of $G$.
In the \textsc{Convex Recoloring} problem we are given an undirected graph $G$ and a (non-convex) coloring of this graph and our goal is to find a convex coloring of the graph that recolors the minimum number of vertices.
In this thesis we consider a special case of this problem, namely the \textsc{2-Convex Recoloring} problem, in which the original coloring uses each color in $\mathcal{C}$ to color at most two vertices.
For this special case we develop a greedy $3/2$-approximation algorithm.
We show that if the input graph is a path then this is in fact a $5/4$-approximation algorithm.


\subsubsection*{Service Chain Placement in SDNs} 
Traditional communication networks use ad-hoc controllers to achieve various functionality and to provide various services for the clients of the network.
An emerging paradigm aims to construct networks of general, smart controllers that can be programed to achieve the same goal. 
This new paradigm offers much more flexibility to the network administrator that can now construct many virtual networks on top of the physical one.
The question that arises is how to allocate resource efficiently in order to satisfy clients demand.
In Chapter~\ref{chapter:service-chain-placement-in-sdns} we model the problem and, using scaling and non-trivial dynamic programming techniques, we give a fully polynomial time approximation scheme for the special case where the physical network is a directed acyclic graph. 
For general graphs we give a parameterized algorithm.
This algorithm finds an optimal solution efficiently for cactus network.

\subsection*{Submodular Function Maximization}
Submodularity is a fundamental mathematical notion that captures the concept of economy of scale and is prevalent in many areas of science and technology.
Given a ground set $U$ a set function $f:2^U \to \mathbb{R}_+$ over $U$ is called \emph{submodular} if it has the \emph{diminishing returns} property:
$f(A \cup \{a\}) - f(A) \geq f(B \cup \{a\}) - f(B)$ for every $A \subseteq B \subseteq U$ and $a \in U \setminus B$.\footnote{
    An equivalent definition is: $f(A) + f(B) \geq f(A \cup B) + f(A \cup B)$ for every $A,B \in U$.
}
Submodular functions naturally arise in different disciplines such as combinatorics, graph theory, probability, game theory, and economics.
Some well known examples include coverage functions, cuts in graphs and hypergraphs, matroid rank functions, entropy, and budget additive functions.
Additionally, submodular functions play a major role in many real world applications, {\em e.g.}, the spread of influence in networks \cite{KKT03,KKT05,KKT15,MR10}, recommender systems \cite{EG11,EVSG09}, document summarization \cite{DKR13,LB10,LB11}, and information gathering \cite{GKS05,KG11,KGGK06,KGGK11,KSG08}, are just a few such examples.

Combinatorial optimization problems with a submodular objective have been the focus of intense research in the last decade as such problems provide a unifying framework that captures many fundamental problems in the theory of algorithms and numerous real world practical applications.
Examples of the former include, {\em e.g.}, Max-CUT and Max-DiCUT \cite{FG95,GW95,HZ01,H01,K72,KKMO07,LLZ02,TSSW00}, Max-$k$-Coverage \cite{F98,SW11,V01}, Max-Bisection \cite{ABG13,FJ97,HZ02,Y01}, Generalized-Assignment \cite{CK05,CKR06,FGMS06,FV06}, and Max-Facility-Location \cite{AS99,CFN77a,CFN77b}\footnote{Many of the above mentioned problems can also be found in introductory books to approximation algorithms \cite{SW11,V01}.}, whereas examples of the latter include, {\em e.g.}, pollution detection \cite{KLGVF08}, gang violence reduction \cite{SSPB14}, outbreak detection in networks \cite{LKGFFVG07}, exemplar based clustering \cite{GK10}, image segmentation \cite{KXFK11}, and recommendation diversification \cite{YG11}.

A main driving force behind the above research is the need for algorithms that not only provide provable approximation guarantees, but are also fast and  simple to implement in practice.
This need stems from the sheer scale of the applicability of submodular maximization problems in diverse disciplines, and is further amplified by the fact that many of the practical applications arise in areas such as machine learning and data mining where massive data sets and inputs are ubiquitous.\footnote{Refer to the recent book \cite{B13} and survey \cite{KG14} for additional examples and applications of submodularity in machine learning.}

In this thesis we consider maximizing a monotone submodular function given a Knapsack constant, we also consider the Maximum Carpool Matching optimization problem and show that this problem can be formulated as a submodular function maximization problem.
We now briefly describe these problems\footnote{A formal, detailed, description of each problem will be given at the relevant chapters.}:

\subsubsection*{Submodular Knapsack} 
In Chapter~\ref{chapter:monotone-submodular-function-maximization} we consider the problem of maximizing a monotone, submodular function given a knapsack constraint.
There is a well known $(1-e^{-1})$-approximation algorithm for this problem that runs in $O(n^5)$ time.
This is the best approximation ratio that can be achieved by a polynomial time algorithm~\cite{khuller1999budgeted}.
A (asymptotic) faster $1 - e^{-1} - \epsilon$-approximation algorithm is also known.
This algorithm runs in $1/\epsilon^{O(1/\epsilon^4)}n\log^2n$ time for any choice of $\epsilon$ and, as the authors of this algorithm mention, is impractical.
In this thesis we develop a framework that for any $\alpha < \beta < \frac{1}{2}$ and an $\alpha$-approximation algorithm $A$ with a running time $t(n)$ produces a $\beta$-approximation algorithm $B$.
The running time of $B$ is $O(n^2) + O(t(n))$ where the constants hidden by the big $O$ depend on $\alpha$ and $\beta$.
This framework can be used, for example, to develop a $(1-e^{-2/3})$-approximation algorithm that runs in time $O(n^2)$ withe very small constant.

\subsubsection*{Maximum Carpool Matching}
In Chapter~\ref{chapter:maximum-carpool-matching} we consider the Maximum Carpool Matching problem.
In this problem we are given an edge weighted directed graph, $G=(V, E)$, and capacity constraints on the vertices, $c:V \to \mathbb{N}$.
Our goal is to find a matching, $M \subseteq E$, with maximum total weight such that in $H = (V, M)$ the out degree of each vertex is at most one, the in degree of each vertex, $v$, is at most $c(v)$, and either the out degree or the in degree of each vertex is zero.
We develop approximation algorithms for several special cases of the problem. 
We also show that \textsc{Maximum Carpool Matching} can be formalized as a non-monotone, unconstrained, submodular function maximization and, thus, admits a $1/2$-approximation algorithm.
Finally we develop a $(1/2 - \epsilon)$-approximation algorithm for a more general problem that captures \textsc{Maximum Carpool Matching}.