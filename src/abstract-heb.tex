\begin{hebrew}
אופטימיזציה קומבינטורית הוא תחום שבו אנו מעוניינים לבחור את האובייקט הכי טוב מבין מספר סופי של אובייקטים נתונים.
טיב האובייקט מתואר על ידי פונקציה שמקצה לכל אובייקט ערך מספרי ואנו מעוניינים (כתלות בסוג הבעיה) למצוא את האובייקט עם הערך המקסימלי או עם הערך המינימלי.
האתגר בבעיות כאלו הוא שהאובייקטים מתוארים בדרך כלל בצורה קומפקטית.
למשל, בבעיית
\emph{עץ פורש מינימום}
נתון לנו גרף לא מכוון עם משקלות על הקשתות ואנו מעוניינים למצוא עץ שפורש את צמתי הגרף עם סך משקל קטן ככל האפשר.
פוטנציאלית, ישנו מספר אקספוננציאלי (בגודל הגרף) של עצים פורשים.
האלגוריתם הנאיבי יעבור על כל העצים הפורשים ויחזיר את הזול ביותר, אולם זהו אלגוריתם לא יעיל ולא מעשי.
לשמחתנו, ידוע כי לבעיה זאת האלגוריתם החמדן - כזה שבונה עץ פורש באופן איטרטיבי על ידי בחירת הקשת הקלה ביותר שניתן להוסיף לעץ שנבנה עד כה - מוצא תמיד עץ פורש קל ביותר.
לאופטימיזציה קומבינטורית יישומים רבים בייעול של תהליכים וניצול יעיל משאבים וגם בתחומים אחרים כמו למידת מכונה, בינה מלאכותית ועיבוד תמונה.

מספר רב של בעיות אופטימיזציה קומבינטורית שייכות למחלקה $NP$-שלמות.
ידוע כי אלגוריתם יעיל לאחת הבעיות במחלקה הנ"ל מהווה למעשה אלגוריתם יעיל לכל הבעיות במחלקה זאת.
עם זאת, על אף מאמצים רבים במשך שנים רבות, לא נמצא אלגוריתם יעיל לאף אחת מהבעיות במחלקה ואף יתכן כי אלגוריתם כזה לא קיים.
אלגוריתם נאיבי, כאמור, הוא לא מעשי וקיימות מספר דרכים להתמודד עם בעיות כאלו.

\begin{description}
\item[\texthebrew{היוריסטיקה}]
אלגוריתם היוריסטי פועל על פי כלל "אצבע" פשוט שמבוסס על אינטואיציה.
אלגוריתמים כאלו בדרך כלל לא מבטיחים פתרון איכותי או זמן ריצה יעיל אבל בדרך כלל הם קלים למימוש ולפעמים הם מספקים פתרונות טובים מספיק.
\item[\texthebrew{מקרים פרטיים}]
לעתים, למרות שלבעיה הכללית לא ידוע אלגוריתם יעיל, קיימים אלגוריתמים יעילים למקרים פרטיים של הבעיה, למשל לקלטים מסוימים או לפונקציית משקל מסוימת.
\item[\texthebrew{סיבוכיות פרמטרית}]
לחלק מהבעיות ניתן למצוא אלגוריתמים יעילים כתלות בפרמטר מסוים של הבעיה (בדרך כלל גודל הפתרון) אלגוריתמים כאלו יעילים ומוצאים פתרון אופטימלי עבור קלטים מסוימים,
אך בדרך כלל אין לנו אפשרות לדעת אם אכן הקלט שברשותנו מקיים את התנאים הדרושים.
\item[\texthebrew{אלגוריתמי קירוב}]
אלגוריתמי קירוב הם אלגוריתמים יעילים שפועלים על כל קלט של הבעיה ומחזירים פתרון "מקורב".
אלגוריתם נקרא $\alpha$-קירוב אם תמיד הוא מחזיר פתרון שערכו אינו גדול (או קטן) מ-$\alpha$ פעמים ערך הפתרון האופטימלי.
נציין שלעתים $\alpha$ יכול להיות פונקציה של הקלט, למשל גודלו.
\end{description}

בתזה זאת אנו מציגים אלגוריתמי קירוב למספר בעיות אופטימיזציה
$\text{NP}$%
-קשיות.
התזה עוסקת בארבע בעיות וכעת נתאר כל בעיה ובעיה.

\section{\texthebrew{בעיית הצביעה הקמורה}}
צביעה של גרף (לא מכוון),
$G = (V, E)$,
 היא פונקציה
$c(V) \to \mathbb{N}$.
נאמר שצביעה היא
\emph{קמורה}
אם תת הגרף שמושרה על ידי כל אחד מהצבעים הוא קשיר.
בבעיית הצביעה הקמורה אנו מקבלים גרף לא מכוון וצבוע (בצביעה שאינה קמורה באופן כללי) והמטרה היא למצוא צביעה קמורה של הגרף שמשנה את הצבע של מספר מינימלי של צמתים (ביחס לצביעה המקורית).
לבעיית הצביעה הקמורה יישומים בביולוגיה חישובית ובתכנון יעיל של רשתות תקשורת.
בתזה זאת אנו מתמקדים במקרה פרטי של הבעיה שבו בצביעה המקורית נעשה שימוש בכל צבע פעמיים לכל היותר.
אנחנו מראים שבמקרה זה האלגוריתם החמדן - כזה שצובע בכל פעם את המסלול הקצר ביותר בין שני צמתים בעלי אותו צבע (שאינם שכנים) - הוא למעשה אלגוריתם 1.5-קירוב לבעיה.
במקרה שהגרף המקורי הוא מסלול אותו אלגוריתם הוא למעשה 1.2-קירוב לבעיה.


\section{\texthebrew{הקצאה של שירותים ברשתות מבוססות תוכנה}}
בניגוד לרשתות תקשורת מסורתיות שבהן כל בקר ברשת הוא בקר יעודי ברשתות תקשורת מבוססות תוכנה ישנם בקרים חכמים שיכולים לספק, באמצעות תוכנה, מספר שירותים מגוון.
המעבר לרשתות מבוססי תוכנה יאפשר למנהלי הרשתות להקצות רשתות וירטואליות על גבי הרשתות הפיזיות וזאת בהתאם לדרישות הלקוחות.
אחד האתגרים שעומד בפני מנהלי הרשתות הוא כיצד להקצות משאבים בצורה יעילה וחסכונית.

בתזה זאת אנו ממדלים את הבעיה במונחים של תורת הגרפים.
נתונה לנו רשת פיזי שממודלת על ידי גרף מכוון,
$G=(V,E)$,
ואוסף אלטרנטיבות של שירותים וירטואליים שמיוצגים על ידי גרף מכוון חסר מעגלים,
$\mathcal{G} = (\mathcal{V}, \mathcal{E})$.
המטרה היא לבחור מסלול של שירותים וירטואליים ולשבץ אותו על גבי הרשת הפיזי.
שיבוץ של מסלול וירטואלי על גבי הרשת הפיזי יעשה בהתאם למגבלות.
לכל צומת פיזי,
$v$,
 יש נפח עיבוד פנוי,
$p(v)$,
ולכל שירות וירטואלי,
$\alpha$,
ישנה דרישת עיבוד,
$\alpha$,
בשיבוץ חוקי נפח העיבוד הפנוי על כל צומת פיזי חייב להיות לפחות סך העיבוד הדרוש לשירותים שמשובצים בצומת זה.
בנוסף, לכל קשת פיזי,
$e$,
נתון רוחב פס פנוי,
$b(e)$,
ולכל קשת וירטואלית,
$\epsilon$,
נתונה דרישת מינימום של רוחב פס,
$b(\epsilon)$.
בשיבוץ חוקי רוחב הפס של קשת פיזי חייב להיות גדול או שווה לרוחב הפס שנדרש על ידי הקשת הווירטואלית  המתאימה.
בנוסף, לכל שירות,
$\alpha$,
וצומת פיזי,
$v$,
נתונה עלות של שיבוץ השירות על גבי הצומת,
$c(\alpha, v)$.
המטרה היא למצוא את השיבוץ הזול ביותר תחת האילוצים הנ"ל.

ראשית, אנחנו מראים שגם אם מגבילים את הבעיה למקריים פרטיים הבעיה נשארת
$\text{NP}$%
-קשה.
לאחר מכן, אנחנו מראים, על ידי טכניקת עיגול ותכנון דינמי מורכב, שקיימת לבעיה סכמת קירוב פולינומית, למקרה שבו גם הרשת הפיזית היא גרף מכוון חסר מעגלים.
בנוסף, אנו מציעים היוריסטיקה למקרה הכללי.
ההיוריסטיקה המוצעת  מוצאת פתרון אופטימלי אם הרשת הפיזית היא גרף קקטוס.








\end{hebrew}
