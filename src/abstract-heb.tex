\begin{hebrew}
אופטימיזציה קומבינטורית הוא תחום שבו אנו מעוניינים לבחור את האובייקט הכי טוב מבין מספר סופי של אובייקטים נתונים.
טיב האובייקט מתואר על ידי פונקציה שמקצה לכל אובייקט ערך מספרי ואנו מעוניינים (כתלות בסוג הבעיה) למצוא את האובייקט עם הערך המקסימלי או עם הערך המינימלי.
האתגר בבעיות כאלו הוא שהאובייקטים מתוארים בדרך כלל בצורה קומפקטית.  
למשל, בבעיית 
\emph{עץ פורש מינימום}
נתון לנו גרף לא מכוון עם משקלות על הקשתות ואנו מעוניינים למצוא עץ שפורש את צמתי הגרף עם סך משקל קטן ככל האפשר.
פוטנציאלית, ישנו מספר אקספוננציאלי (בגודל הגרף) של עצים פורשים.
האלגוריתם הנאיבי יעבור על כל העצים הפורשים ויחזיר את הזול ביותר, אולם זהו אלגוריתם לא יעיל ולא מעשי.
לשמחתנו, ידוע כי לבעיה זאת האלגוריתם החמדן - כזה שבונה עץ פורש באופן איטרטיבי על ידי בחירת הקשת הקלה ביותר שניתן להוסיף לעץ שנבנה עד כה - מוצא תמיד עץ פורש קל ביותר.

בבעיית 
\emph{קבוצה שולטת קלה ביותר}
נתון לנו גרף לא מכוון ואנו רוצים למצוא קבוצה קטנה ביותר של צמתים כך שלכל יתר הצמתים בגרף ישנו שכן אחד לפחות בקבוצה הנבחרת.
לבעיה זאת לא ידוע אלגוריתם יעיל (פולינומי בגודל הקלט) ויתכן כי אף אלגוריתם כזה אינו קיים.
בעיה זאת שייכת למחלקה 
$\text{NP}$%
-שלמות.
\end{hebrew}
