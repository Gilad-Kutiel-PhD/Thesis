\begin{hebrew}
\section*{\texthebrew{בעיות תכנון רשתות}}
בבעיות של
\emph{תכנון רשתות}
נתון לנו אוסף של משאבים והמטרה שלנו היא לבנות באמצעותם רשת שמקיימת תנאים מסוימים תוך שימוש מינימלי במשאבים.
דוגמה לבעיה כזאת היא בעיית
\emph{עץ פורש מינימום}
בה נתון לנו גרף לא מכוון עם משקלים על הקשתות והמטרה שלנו היא למצוא עץ שפורש את כל צמתי הגרף במשקל מינימלי.
לבעיות תכנון רשתות ישנן יישומים רבים: תכנון רשתות תקשורת ותחבורה, תכנון מעגלים חשמליים, תכנון שרשראות הספקה וכן יישומים  בביולוגיה חישובית ובתחומים אחרים.

בעיות רבות בתכנון רשתות הן
$\text{NP}$
-קשות, ולכן לא מעשי לנסות לפתור בעיות כאלה בצורה אופטימלית.
בתזה זאת אנו מתמקדים במציאת אלגוריתמי קירוב לשתי בעיות בתכנון רשתות שכעת נתאר:

\subsection*{\texthebrew{בעיית הצביעה הקמורה}}
צביעה של גרף (לא מכוון),
$G = (V, E)$,
 היא פונקציה
$c(V) \to \mathbb{N}$.
נאמר שצביעה היא
\emph{קמורה}
אם תת הגרף שמושרה על ידי כל אחד מהצבעים הוא קשיר.
בבעיית הצביעה הקמורה אנו מקבלים גרף לא מכוון וצבוע (בצביעה שאינה קמורה בדרך כלל) והמטרה היא למצוא צביעה קמורה של הגרף שמשנה את הצבע של מספר מינימלי של צמתים (ביחס לצביעה המקורית).
לבעיית הצביעה הקמורה יישומים בביולוגיה חישובית ובתכנון יעיל של רשתות תקשורת.
בתזה זאת אנו מתמקדים במקרה פרטי של הבעיה שבו בצביעה המקורית נעשה שימוש בכל צבע פעמיים לכל היותר.
אנחנו מראים שבמקרה זה האלגוריתם החמדן - כזה שצובע בכל פעם את המסלול הקצר ביותר בין שני צמתים בעלי אותו צבע (שאינם שכנים) - הוא למעשה אלגוריתם 1.5-קירוב לבעיה.
במקרה שהגרף המקורי הוא מסלול אותו אלגוריתם הוא למעשה 1.2-קירוב לבעיה.


\subsection*{\texthebrew{הקצאה של שירותים ברשתות מבוססות תוכנה}}
בניגוד לרשתות תקשורת מסורתיות שבהן כל בקר ברשת הוא בקר יעודי ברשתות תקשורת מבוססות תוכנה ישנם בקרים חכמים שיכולים לספק, באמצעות תוכנה, מספר שירותים מגוון.
המעבר לרשתות מבוססי תוכנה יאפשר למנהלי הרשתות להקצות רשתות וירטואליות על גבי הרשתות הפיזיות וזאת בהתאם לדרישות הלקוחות.
אחד האתגרים שעומד בפני מנהלי הרשתות הוא כיצד להקצות משאבים בצורה יעילה וחסכונית.

בתזה זאת אנו ממדלים את הבעיה במונחים של תורת הגרפים.
נתונה לנו רשת פיזי שממודלת על ידי גרף מכוון,
$G=(V,E)$,
ואוסף מסלולים (שרשראות) אלטרנטיבים של שירותים וירטואליים שמיוצגים על ידי גרף מכוון חסר מעגלים,
$\mathcal{G} = (\mathcal{V}, \mathcal{E})$.
המטרה היא לבחור מסלול של שירותים וירטואליים ולשבץ אותו על גבי הרשת הפיזית.
שיבוץ של מסלול וירטואלי על גבי הרשת הפיזית יעשה בהתאם למגבלות.
לכל צומת פיזי,
$v$,
 יש נפח עיבוד פנוי,
$p(v)$,
ולכל שירות וירטואלי,
$\alpha$,
ישנה דרישת עיבוד,
$p(\alpha)$,
בשיבוץ חוקי נפח העיבוד הפנוי על כל צומת פיזי חייב להיות לפחות סך העיבוד הדרוש לשירותים שמשובצים בצומת זה.
בנוסף, לכל קשת פיזית,
$e$,
נתון רוחב פס פנוי,
$b(e)$,
ולכל קשת וירטואלית,
$\epsilon$,
נתונה דרישת מינימום של רוחב פס,
$b(\epsilon)$.
בשיבוץ חוקי רוחב הפס של קשת פיזית חייב להיות גדול או שווה לרוחב הפס שנדרש על ידי הקשת הווירטואלית  המתאימה.
בנוסף, לכל שירות,
$\alpha$,
וצומת פיזי,
$v$,
נתונה עלות של שיבוץ השירות על גבי הצומת,
$c(\alpha, v)$.
המטרה היא למצוא את השיבוץ הזול ביותר תחת האילוצים הנ"ל.

ראשית, אנחנו מראים שגם אם מגבילים את הבעיה למקריים פרטיים הבעיה נשארת
$\text{NP}$%
-קשה.
לאחר מכן, אנחנו מראים, על ידי טכניקת עיגול ותכנון דינמי מורכב, שקיימת לבעיה סכמת קירוב פולינומית, למקרה שבו גם הרשת הפיזית היא גרף מכוון חסר מעגלים.
בנוסף, אנו מציעים היוריסטיקה למקרה הכללי.
ההיוריסטיקה המוצעת  מוצאת פתרון אופטימלי באופן יעיל אם הרשת הפיזית היא גרף קקטוס.


% SUBMODULAR
\section*{\texthebrew{מקסום פונקציה תת מודולרית}}
תת-מודולריות היא מושג בסיסי במתמטיקה שלוכד, בין היתר, את מושג התפוקה השולית הפוחתת והוא נפוץ בתחומים רבים במדעים.
פונקציה,
$f$,
שפועלת על קבוצות של אלמנטים היא תת-מודולרית אם היא מקיימת את התכונה
$f(A \cup B) + f(A \cap B) \leq f(A) + f(B)$
לכל שתי קבוצות $A$ ו-$B$ בתחום של הפונקציה.
מקסום של פונקציה תת מודולרית היא הכללה, למשל, של בעיית חתך מקסימלי בגרף, בעיה
$\text{NP}$
-קשה.

בתזה זאת נתמקד באלגוריתם קירוב מהיר ופשוט למקסום של פונקציה תת מודולרית.
בנוסף נעסוק בבעיית השידוך המקסימלי בנסיעה משותפת ונראה כי ניתן לתאר בעיה זאת על ידי מקסום של פונקציה תת מודולרית.
כעת נתאר את שתי הבעיות הנ"ל:

\subsection*{\texthebrew{מקסום פונקציה תת-מודולרית תחת אילוצי תרמיל גב}}
פונקציה שפועלת על קבוצות  של אלמנטים נקראת
\emph{מונוטונית}
אם מתקיים ש-%
$f(A) \leq f(B)$
לכל שתי קבוצות,
$A \subseteq B$,
בתחום של הפונקציה.
בהינתן קבוצה של אלמנטים, $U$, פונקציית מחיר,
$c:U \to \mathbb{R_+}$,
ותקציב, $b$, נאמר שתת קבוצה,
$S \subseteq U$,
מקיימת את אילוץ תרמיל הגב אם סך המחירים של האלמנטים ב-$S$ קטן או שווה לתקציב.
מקסום של פונקציה תת-מודולרית תחת אילוץ תרמיל גב היא הכללה של בעיית הכיסוי המקסימלי תחת אילוץ תרמיל גב,
ויש לבעיה יישומים בלמידה חישובית, תקצור אוטומטי של מאמרים, בחירת מיקום יעיל של חיישנים ועוד.


לבעיה זאת ידוע אלגוריתם
$(1-e^{-1})$-%
קירוב שרץ בזמן
$O(n^5)$.
זהו הקירוב הטוב ביותר שניתן להגיע אליו בעזרת אלגוריתם פולינומי בהינתן ש-%
$\text{P} \neq \text{NP}$.
ניתן להראות שיחס הקירוב של האלגוריתם החמדן - כזה שבכל איטרציה בוחר את האלמנט שהיחס בין התועלת השולית שלו לבין העלות שלו גדול ביותר - יכול להיות קטן כרצוננו במקרה הכללי.
עם זאת, על ידי שינוי קטן באלגוריתם מקבלים אלגוריתם
$(1 - e^{-1/2})$-%
קירוב מבלי לפגוע בזמן הריצה -
$O(n^2)$.
קיים גם אלגוריתם, שרץ בזמן
$1/\epsilon^{O(1/\epsilon^4)}n\log^2n$
ונותן קירוב
$1 - e^{-1} -\epsilon$
לכל בחירה של
$\epsilon$.
למרות שזמן הריצה האסימפטוטי של האלגוריתם האחרון הוא קרוב ללינארי, זמן הריצה בפועל של אלגוריתם כזה, כמו שצוין על ידי אחד המחברים, הוא לא מעשי.

בתזה זאת, אנו מציגים אלגוריתם חדש שרץ בזמן
$O(n^2)$
ונותן קירוב של
$1 - e^{-2/3}$
לבעיה.




\subsection*{\texthebrew{בעיית שידוך מקסימלי בנסיעה משותפת}}

בנסיעה משותפת מספר אנשים חולקים את אותו הרכב ובכך תורמים להפחתת העומס בכבישים וחסכון בעלויות דלק תחזוקה ועוד.
כאשר קבוצה של אנשים רוצים מעוניינים לבצע נסיעה משותפת (למשל קבוצת אנשים שגרה באותה שכונה ועובדת באותו אזור) נשאלת השאלה כיצד עליהם לשתף את הנסיעה.
לחברים בקבוצה ישנן העדפות עם מי הם מעוניינים לנסוע ובאיזה רכב (שלהם או של מישהו אחר בקבוצה).
העדפות אלו יכולות להיות מבוססות על טעם זהה במוזיקה הרגלי עישון (או חוסר עישון), נושאי שיחה משותפים, הערכה של טיב הנהג וכדומה.
את ההעדפות הללו ניתן למדל כגרף מכוון עם משקלים על הקשתות שבו כל צומת מייצג חבר בקבוצה ואת כלי הרכב שברשותו ומשקל הקשת מצומת א' לצומת ב' מייצג את רמת ההעדפה של חבר א' לנסוע ברכב של חבר ב'.

בבעיית השידוך מקסימלי בנסיעה משותפת אנחנו מעוניינים לקבוע כיצד חברי הקבוצה צריכים לבצע את הנסיעה המשותפת כך שסך ההעדפות שסופקו גדול ככל הניתן, פורמלית:
בהינתן גרף מכוון,
$G = (V, E)$,
פונקציית קיבול (מספר המקומות הפנויים),
$c:V \to \mathbb{N}$,
ופונקציית משקל (העדפות),
$w:E \to \mathbb{R}_+$,
אנו מעוניינים למצוא תת קבוצה של קשתות,
$M \subset E$,
שממקסמת את סך המשקלים על הקשתות בקבוצה, כך שלכל צומת דרגת יציאה 0 או דרגת כניסה 0 ולכל צומת,
$v$,
דרגת כניסה קטנה או שווה ל-%
$c(V)$.

בתזה זאת אנו מראים שאלגוריתם חיפוש מקומי פשוט הוא חצי קירוב לבעיה הלא ממושקלת.
אנחנו מראים שחיפוש מקומי יותר מורכב נותן קירוב של
$1/2 - \epsilon$
לבעיה הכללית לכל בחירה של
$\epsilon$.
לבסוף אנחנו מראים שאת הבעיה ניתן למדל כבעיית מקסום של פונקציה תת-מודולרית לא מאולצת (ולא מונוטונית) ולכן קיים אלגוריתם חצי קירוב לבעיה.




\end{hebrew}
