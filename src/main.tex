\nonstopmode
\documentclass[a4paper,UKenglish,cleveref, autoref]{lipics-v2019}

\usepackage{bibentry}
\tikzset{every picture/.style={
	very thick,
	>=Latex,
}}


% NODE
\tikzset{default node/.style={
	draw, 
	circle,
	inner sep=0mm,
	minimum size=5mm,
	very thick,
	font=\small
}}

\tikzset{
colored node/.style={line width=1.6pt, star, minimum size=9mm,inner sep=0pt,scale=0.9,}, 
red node/.style={colored node, draw=red, star points=2},
blue node/.style={colored node, draw=blue, star points=3},
green node/.style={colored node, draw=green, star points=4},
black node/.style={colored node, draw=black, star points=5},
orange node/.style={colored node, draw=orange, star points=6},
brown node/.style={colored node, draw=brown, star points=8},
teal node/.style={colored node, draw=teal, star points=10},
violet node/.style={colored node, draw=violet, star points=12},
olive node/.style={colored node, draw=olive, star points=14},
cyan node/.style={colored node, draw=cyan, star points=18},
}


% LABELS
\tikzset{
	label/.style={
		rectangle,
		draw=none,
		sloped,
		midway,
		fill=white,
		inner sep=2pt,
		minimum height=0,
		minimum width=0,
	},
	label above/.style={
		label,
		above,
	},
	label below/.style={
		label,
		below,
	},
	label inside/.style={
		label,
		fill=white,
		draw=black,
	},
}

% EDGES
\tikzset{
	light/.style={
		thin,
		gray,
	},
	path/.style={
		light,
		decorate, 
		decoration={snake, segment length=18pt},
	},
	brace/.style={
		decorate,
		decoration={brace, amplitude=5},
		-,
	},
}

% MISC
\tikzset{
	cloud/.style={
		-,
		decorate, 
		decoration={
			snake, 
			segment length=3mm,
			amplitude=.5mm,
		},
	}
	,packing/.style={
		dotted
		,purple
		,very thick	
		,-
	}
}




% \usepackage{lineno}
% \linenumbers

\def\CRP{CR}
\def\TWOCR{2-CR}

\newtheorem{observation}{Observation}
\newtheorem{lemma}{Lemma}
\newtheorem{theorem}{Theorem}

\tikzset{
hide text/.style={
	text opacity=0,
},
hide/.style={
	draw=none,
	opacity=0,
	hide text,
},	
/tikz/graphs/default graph/.style={
	random seed=1,
	spring layout,
	spring constant=1,
	node distance=12mm,	
},
packing/.style={
	dashed, 
	very thick,	
	looseness=.7,
},
mapping/.style={
	-{Triangle[]}, 
	dotted,
	thick,	
},
colored node/.style={
	line width=1.6pt, 
	star, 
	minimum size=9mm,
	inner sep=0pt,
	scale=0.9,
}, 
red node/.style={colored node, draw=red, star points=2},
blue node/.style={colored node, draw=blue, star points=3},
green node/.style={colored node, draw=green, star points=4},
black node/.style={colored node, draw=black, star points=5},
orange node/.style={colored node, draw=orange, star points=6},
brown node/.style={colored node, draw=brown, star points=8},
teal node/.style={colored node, draw=teal, star points=10},
violet node/.style={colored node, draw=violet, star points=12},
olive node/.style={colored node, draw=olive, star points=14},
cyan node/.style={colored node, draw=cyan, star points=18},
}


\title{A Fast and Simple Algorithm for Submodular Maximization with a Knapsack Constraint}
\titlerunning{A Fast and Simple Algorithm for Submodular Knapsack}

\author{Ariel Kulik}{Department of Computer Science, Technion, Haifa, Israel}{kulik@cs.technion.ac.il}{}{}
\author{Gilad Kutiel}{Department of Computer Science, Technion, Haifa, Israel}{gkutiel@cs.technion.ac.il}{}{}
\author{Roy Schwartz}{Department of Computer Science, Technion, Haifa, Israel}{schwartz@cs.technion.ac.il}{}{}
\authorrunning{A.\,Kulik and G.\,Kutiel and R.\,Schwartz}

\Copyright{Ariel Kulik and Gilad Kutiel and Roy Schwartz}
\ccsdesc{}

%\ccsdesc[100]{ {\color{red}{TBD}} }
%\ccsdesc[100]{ {\color{red}{TBD}} }

\ccsdesc[100]{Theory of computation~Submodular optimization and polymatroids}
\ccsdesc[100]{Theory of computation~Approximation algorithms analysis}

\keywords{knapsack, submodular function, approximation algorithm}

%\category{}

%\relatedversion{}

%\supplement{}

\nolinenumbers



\newcommand{\SK}{{\textsc{Submodular Knapsack}}\xspace}



\begin{document}
\maketitle

\begin{abstract}
We consider the problem of maximizing a monotone submodular function with a knapsack constraint.
In this work we aim at finding fast and simple algorithms for the problem.
Previously, the best fast and simple algorithm is that of Khuller {\em et. al.} [IPL`99] which runs in time $O(n^2)$ and achieves an approximation guarantee of $(1-e^{-\nicefrac[]{1}{2}})\approx 0.393 $.
We present a new fast and simple algorithm which retains the same running time and has an improved approximation guarantee of $0.4536$.
%At the heart of our analysis lies a method that enables us to better analyze the greedy ``bang per buck'' algorithm in the presence of elements with varying cost.
Moreover, we present a general method for ``amplifying'' the approximation factor of any algorithm for the problem, while losing little in the constants of the running time.
Applying this amplification to our new algorithm enables us to further improve the results obtained.
We believe that our amplification method might be of independent interest.

%    We consider the problem of maximizing a monotone submodular function under a knapsack constraint.
%    We propose a simple and fast algorithm to the problem that runs in $O(n^2)$ time (with very small constants) and achieves a $(1-e^{-2/3})$ approximation.
%    As a side effect of our work we fix a logical flaw in the analysis of the ModifiedGreedy algorithm for this problem as appeared in \cite{khuller1999budgeted} and \cite{krause2005note}.
%    We also analyze the Modified$^2$Greedy algorithm and show that it might achieves a better approximation than the ModifiedGreedy algorithm.
%
%    {\color{red}{TBD}}
\end{abstract}

\newcommand{\defpath}[1]{\def\path{content/#1}}
\newcommand{\add}[1]{\input{\path/#1}}

\chapter{Introduction}
\defpath{introduction}
\add{introduction}

% \chapter{Convex Recoloring}
% \defpath{2cr}
% \add{2cr}
% 
% \chapter{Service Chain Placement in SDNs}
% \defpath{vpn}
% \add{vpn}
% 
% \chapter{Maximum Carpool Matching}
% \defpath{carpool}
% \add{carpool}

\chapter{Maximum Submodular Function Maximization}
\defpath{knapsack}
\add{knapsack}

\chapter{Conclusion}
\defpath{conclusion}
\add{conclusion}

\newpage

\bibliographystyle{plainurl}
\bibliography{bib/knapsack}

\appendix
\section{Omitted Proofs}
\label{appendix:omitted}
\begin{proof}[Proof of Lemma \ref{lemma:sub-main}]
	

		Consider $a_i$ and observe that:
		\begin{align}
		\frac{f(a_i|A_{i-1})}{c(a_i)}c(B)
		& = \sum_{e \in B} \frac{f(a_i|A_{i-1})}{c(a_i)}c(e)
		\nonumber
		\\ 	& \geq \sum_{e \in B} \frac{f(e|A_{i-1})}{c(e)}c(e)
		\label{ineq:main:cond}
		\\	& \geq f(B|A_{i-1})
		\label{ineq:main:sub}
		\\ 	& \geq f(B) - f(A_{i-1}).
		\label{ineq:main:mon}
		\end{align}
		Inequality \eqref{ineq:main:cond} follows from the condition in the lemma, inequality \eqref{ineq:main:sub} follows from the submodularity of $f$, and inequality \eqref{ineq:main:mon} follows from the monotonicity of $f$.
		Thus, from the above we can conclude that:
		$$
		f(B) - f(A_i)  \leq (f(B) - f(A_{i - 1}))
		\left(1 - \frac{c(a_i)}{c(B)}\right).
		$$
		Hence,
		$$
		f(B) - f(A_i)  \leq f(B) \prod_{j = 1}^{i}
		\left(1 - \frac{c(a_j)}{c(B)}\right).
		$$
		Applying the inequality $1 - x \leq e^{-x}$ we get that:
		$$
		f(B) - f(A_i)  \leq f(B)\cdot
		e^{-\frac{c(A_i)}{c(B)}}.
		$$
		Rearranging the terms and setting $i = k$ completes the proof.

	
	
	
\end{proof}





\end{document}
